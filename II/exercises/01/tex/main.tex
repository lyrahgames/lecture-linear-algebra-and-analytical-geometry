% documentclass: article used for scientific journals, short reports, program documentation, etc
% options: fontsize 11, generate document for double sided printing, a4-paper
\documentclass[10pt, twoside, a4paper, fleqn]{article}

% package for changing page layout
\usepackage{geometry}
\geometry{a4paper, lmargin=40mm, rmargin=45mm, tmargin=40mm, bmargin=45mm}
% set indentation
\setlength{\parindent}{1em}
% set factor for line spacing
% \linespread{1.0}\selectfont
% set (dynamic) additional line spacing
% \setlength{\parskip}{1ex plus 0.5ex minus 0.3ex}

% rigorous formatting (not too much hyphens)
% \fussy
% \sloppy

% package for changing page layout (used to indent whole paragraphs with adjustwidth)
\usepackage{changepage}

% input encoding for special characters (e.g. ä,ü,ö,ß), only for non english text
% options: utf8 as encoding standard, latin1
\usepackage[utf8]{inputenc}
% package for font encoding
\usepackage[T1]{fontenc}
% package for changing used language (especially for more than one language)
% options: ngerman (new spelling) or default: english
\usepackage[ngerman]{babel}
% package for times font
% \usepackage{times}
% package for latin modern fonts
\usepackage{lmodern}

% package for math symbols, functions and environments from ams(american mathematical society)
\usepackage{amsmath}
\usepackage{mathtools}
% package for extended symbols from ams
\usepackage{amssymb}
% package for math black board symbols (e.g. R,Q,Z,...)
\usepackage{bbm}
% package used for calligraphic math symbols
\usepackage{mathrsfs}
% package for extended symbols from stmaryrd(st mary road)
\usepackage{stmaryrd}
% package for more math blackboard symbols
\usepackage{dsfont}

% pack­age im­ple­ments scal­ing of the math ex­ten­sion font cmex; used for scaling math signs
\usepackage{exscale}

% package for including extern graphics plus scaling and rotating
\usepackage{graphicx}
%package for positioning figures
\usepackage{float}
% package for changing color of font and paper
% options: using names of default colors (e.g red, black)
% \usepackage[usenames]{color}
\usepackage[dvipsnames]{xcolor}
\definecolor{shadecolor}{gray}{0.9}
% package for customising captions
\usepackage[footnotesize, hang]{caption}
% package for customising enumerations (e.g. axioms)
\usepackage{enumitem}
% calc package reimplements \setcounter, \addtocounter, \setlength and \addtolength: commands now accept an infix notation expression
\usepackage{calc}
% package for creating framed, shaded, or differently highlighted regions that can break across pages; environments: framed, oframed, shaded, shaded*, snugshade, snugshade*, leftbar, titled-frame
\usepackage{framed}
% package for creating custom "list of"
% options: titles: do not intefere with standard headings for "list of"
\usepackage[titles]{tocloft}
% change enumeration style of equations
% \renewcommand\theequation{\thesection.\arabic{equation}}

% init list of math for definitions and theorems
\newcommand{\listofmathcall}{Verzeichnis der Definitionen und Sätze}
\newlistof{math}{mathlist}{\listofmathcall}
% add parentheses around argument
\newcommand{\parent}[1]{ \ifx&#1&\else (#1) \fi }
% unnumerated mathematical definition environment definiton
\newenvironment{mathdef*}[2]{
	\medskip
	\begin{tcolorbox}[colback=white, boxrule=0.5pt, colframe=black, breakable]
	\noindent
	{ \fontfamily{ppl}\selectfont \textbf{\textsc{#1:}} } ~ #2 
	\par \hfill\\ 
	\fontfamily{lmr}\selectfont \itshape
}{
	\end{tcolorbox}
	\medskip
}
% definitions for numerated mathematical definition environment
\newcounter{mathdefc}[section]
\newcommand*{\mathdefnum}{\thesection.\arabic{mathdefc}}
\renewcommand{\themathdefc}{\mathdefnum}
\newenvironment{mathdef}[2]{
	\refstepcounter{mathdefc}
	\addcontentsline{mathlist}{figure}{\protect{\numberline{\mathdefnum}#1 ~ #2}}
	\begin{mathdef*}{#1 \mathdefnum}{#2}
}{
	\end{mathdef*}
}
% standard mathdef calls
\newcommand{\definitioncall}{Definition}
\newenvironment{definition*}[1][]{ \begin{mathdef*}{\definitioncall}{\parent{#1}} }{ \end{mathdef*} }
\newenvironment{definition}[1][]{ \begin{mathdef}{\definitioncall}{\parent{#1}} }{ \end{mathdef} }
% unnumerated theorem environment definition
\newenvironment{maththeorem*}[2]{
	\begin{leftbar}%[boxrule=0pt, leftrule=3pt, arc=0pt, colback=white, colframe=black, enhanced jigsaw]
	\noindent
	{ \fontfamily{ppl}\selectfont \textbf{\textsc{#1:}} } ~ #2
	\par \hfill\\ 
	\fontfamily{lmr} \fontshape{it} \selectfont
}{ 
	\end{leftbar}
}
% definitions for numerated theorem environment
\newcounter{maththeoremc}[section]
\newcommand*\maththeoremnum{\thesection.\arabic{maththeoremc}}
\renewcommand{\themaththeoremc}{\maththeoremnum}
\newenvironment{maththeorem}[2]{
	\refstepcounter{maththeoremc}
	\addcontentsline{mathlist}{figure}{\protect{\qquad\numberline{\maththeoremnum}#1 ~ #2}}
	\begin{maththeorem*}{#1 \maththeoremnum}{#2}
}{
	\end{maththeorem*}
}
% standard maththeorem calls
\newcommand{\theoremcall}{Theorem}
\newenvironment{theorem*}[1][]{ \begin{maththeorem*}{\theoremcall}{\parent{#1}} }{ \end{maththeorem*} }
\newenvironment{theorem}[1][]{ \begin{maththeorem}{\theoremcall}{\parent{#1}} }{ \end{maththeorem} }
\newcommand{\lemmacall}{Lemma}
\newenvironment{lemma*}[1][]{ \begin{maththeorem*}{\lemmacall}{\parent{#1}} }{ \end{maththeorem*} }
\newenvironment{lemma}[1][]{ \begin{maththeorem}{\lemmacall}{\parent{#1}} }{ \end{maththeorem} }
\newcommand{\propositioncall}{Proposition}
\newenvironment{proposition*}[1][]{ \begin{maththeorem*}{\propositioncall}{\parent{#1}} }{ \end{maththeorem*} }
\newenvironment{proposition}[1][]{ \begin{maththeorem}{\propositioncall}{\parent{#1}} }{ \end{maththeorem} }
\newcommand{\corollarycall}{Korollar}
\newenvironment{corollary*}[1][]{ \begin{maththeorem*}{\corollarycall}{\parent{#1}} }{ \end{maththeorem*} }
\newenvironment{corollary}[1][]{ \begin{maththeorem}{\corollarycall}{\parent{#1}} }{ \end{maththeorem} }
% q.e.d. definition
\newcommand{\qed}{ \par \hfill \fontfamily{lmr} \fontshape{it} \selectfont \mbox{q.e.d.} \\}
\newcommand{\qedbox}{ \par \hfill $\Box$ \\ }
% proof environment definition for theorems
\newenvironment{mathproof}[1]{
	\par\hfill\\
	\noindent
	{ \fontfamily{lmr}\selectfont \small \textsc{#1:}}
	\normalfont
	\small
	\begin{adjustwidth}{1em}{}
	\medskip
}{ 
	\end{adjustwidth} 
	% \qed
	\qedbox
}
% standard mathproof calls
\newcommand{\proofcall}{Beweis}
\newenvironment{proof}{ \begin{mathproof}{\proofcall} }{ \end{mathproof} }
\newcommand{\proofideacall}{Beweisidee}
\newenvironment{proofidea}{ \begin{mathproof}{\proofideacall} }{ \end{mathproof} }

% new displaymath command, so that equations will not be stretched
\newcommand{\D}[1]{\mbox{$ #1 $}}
% make unnumerated equation
\newcommand{\E}[1]{\[ #1 \]}
% command for curly brackets
\newcommand{\curlb}[1]{\left\{ #1 \right\}}
% command for box brackets
\newcommand{\boxb}[1]{\left[ #1 \right]}
% command for parentheses/curved brackets
\newcommand{\curvb}[1]{\left( #1 \right)}
% command for angle brackets
\newcommand{\angleb}[1]{\left\langle #1 \right\rangle}
% command for floor brackets
\newcommand{\floorb}[1]{\left\lfloor #1 \right\rfloor}
% command for ceil brackets
\newcommand{\ceilb}[1]{\left\lceil #1 \right\rceil}
% command for creating sets
\newcommand{\set}[2]{ \left\{ #1 \enspace \middle\vert \enspace #2 \right\} }
% command for absolute value
\newcommand{\abs}[1]{\left\vert #1 \right\vert}
\newcommand{\norm}[1]{\left\| #1 \right\|}
% commands for writing limits
\newcommand{\limit}[3]{\, \longrightarrow \, #1, \ #2 \longrightarrow #3}
\newcommand{\Limit}[2]{\lim_{#1 \rightarrow #2}}
% command for differential
\newcommand{\diff}{\mathrm{d}}
\newcommand{\Diff}{\mathrm{D}}
% command for derivative
\newcommand{\Deriv}[3][]{\Diff_{#2}^{#1}#3}
\newcommand{\deriv}[3][]{\dfrac{\diff^{#1}#2(#3)}{\diff #3^{#1}}}
% command for integral
\newcommand{\integral}[4]{\int_{#1}^{#2} #3\ \diff #4}
\newcommand{\Integral}[4]{\int\limits_{#1}^{#2} #3\ \diff #4}
\newcommand{\iintegral}[2]{\int #1\ \diff #2} % indefinite integral
% mathematical definitions (standard sets)
\newcommand{\SR}{\mathds{R}} % real numbers
\newcommand{\SC}{\mathds{C}} % complex numbers
\newcommand{\SN}{\mathds{N}} % natural numbers
\newcommand{\SZ}{\mathds{Z}} % integral numbers
\newcommand{\SQ}{\mathds{Q}} % rational numbers
\newcommand{\SP}{\mathcal{P}} % power set
\newcommand{\SFP}{\mathds{P}} % polynom functions
\newcommand{\SFC}{\mathrm{C}} % complex valued functions (continous or differentiable)
\newcommand{\SFL}{\mathcal{L}} % space of integrable functions
\newcommand{\SFLL}{\mathrm{L}} % space of integrable function classes
\newcommand{\SH}{\mathcal{H}} % hilbert space
% mathematical standard functions
\DeclareMathOperator{\real}{Re} % real part
\DeclareMathOperator{\imag}{Im} % imaginary part
\DeclareMathOperator{\diag}{diag}
\DeclareMathOperator{\id}{Id}
\newcommand{\FF}{\mathcal{F}} % fourier transform
\newcommand{\FE}{\mathbb{E}} % expectation
\DeclareMathOperator{\var}{var} % variance
\newcommand{\FN}{\mathcal{N}} % normal distribution

\newcommand{\m}[1]{\mathrm{#1}}

% command for physical units
\newcommand{\unit}[1]{\, \mathrm{#1}}


% package for init listings(non-formatted  text) e.g. different source codes
\usepackage{listings}


% definitions for listing colors
\definecolor{codeDarkGray}{gray}{0.2}
\definecolor{codeGray}{gray}{0.4}
\definecolor{codeLightGray}{rgb}{0.94,0.94,0.91}
\definecolor{codeBorder}{rgb}{0.34,0.24,0.21}
% predefinitions for listings
\newcommand{\listingcall}{Listing}
\newlength{\listingframemargin}
\setlength{\listingframemargin}{1em}
\newlength{\listingmargin}
\setlength{\listingmargin}{0.08\textwidth}
% \newlength{\listingwidth}
% \setlength{\listingwidth}{ ( \textwidth - \listingmargin * \real{2} + \listingframemargin * \real{2} ) }
% definitions for list of listings
\newcommand{\listoflistingscall}{\listingcall -Verzeichnis}
\newlistof{listings}{listinglist}{\listoflistingscall}
% style definition for standard code listings
\lstdefinestyle{std}{
	belowcaptionskip=0.5\baselineskip,
	breaklines=true,
	frameround=tttt,
	% frame=false,
	xleftmargin=0em,
	xrightmargin=0em,
	showstringspaces=false,
	showtabs=false,
	% tab=\smash{\rule[-.2\baselineskip]{.4pt}{\baselineskip}\kern.5em},
	basicstyle= \fontfamily{pcr}\selectfont\footnotesize\bfseries,
	keywordstyle= \bfseries\color{MidnightBlue}, %\color{codeDarkGray},
	commentstyle= \itshape\color{codeGray},
	identifierstyle=\color{codeDarkGray},
	stringstyle=\color{BurntOrange}, %\color{codeDarkGray},
	numberstyle=\tiny\ttfamily,
	% numbers=left,
	numbersep = 1em,
	% stepnumber = 1,
	% captionpos=t,
	tabsize=4,
	% backgroundcolor=\color{codebLightGray},
	rulecolor=\color{codeBorder},
	framexleftmargin=\listingframemargin,
	framexrightmargin=\listingframemargin
}
% definition for unnumerated listing
\newcommand{\inputlistingn}[3][]{
	\begin{center}
		\begin{adjustwidth}{\listingmargin}{\listingmargin}
			\centerline{ {\fontfamily{lmr}\selectfont \footnotesize \listingcall:}\quad {\footnotesize #2} }
			\lstinputlisting[style=std, #1]{#3}
		\end{adjustwidth}
	\end{center}
}
% definition for numerated listing
\newcounter{listingc}[section]
\newcommand*\listingnum{\thesection.\arabic{listingc}}
\renewcommand{\thelistingc}{\listingnum}
\newcommand{\inputlisting}[3][]{
	\refstepcounter{listingc}
	\addcontentsline{listinglist}{figure}{\protect{\numberline{\listingnum:} #2 } }
	% \inputlistingn[#1]{#2}{#3}
	\begin{center}
		\begin{adjustwidth}{\listingmargin}{\listingmargin}
			\centerline{ {\fontfamily{lmr}\selectfont \footnotesize \listingcall~\listingnum:}\quad {\footnotesize #2} }
			\lstinputlisting[style=std, #1]{#3}
		\end{adjustwidth}
	\end{center}
}


% package for including csv-tables from file
% \usepackage{csvsimple}
% package for creating, loading and manipulating databases
\usepackage{datatool}

% package for converting eps-files to pdf-files and then include them
\usepackage{epstopdf}
% use another program (ps2pdf) for converting
% !!! important: set shell_escape=1 in /etc/texmf/texmf.cnf (Linux/Ubuntu 12.04) for allowing to use other programs
% !!!			or use the command line with -shell-escape
% \epstopdfsetup{outdir=./}
% \epstopdfDeclareGraphicsRule{.eps}{pdf}{.pdf}{
% ps2pdf -dEPSCrop #1 \OutputFile
% }


% package for reference to last page (output number of last page)
\usepackage{lastpage}
% package for using header and footer
% options: automate terms of right and left marks
% \usepackage[automark]{scrpage2}
% \setlength{\headheight}{4\baselineskip}
% set style for footer and header
% \pagestyle{scrheadings}
% \pagestyle{headings}
% clear pagestyle for redefining
% \clearscrheadfoot
% set header and footer: use <xx>head/foot[]{Text} (i...inner, o...outer, c...center, o...odd, e...even, l...left, r...right)

% use that for mark to last page: \pageref{LastPage}
% set header separation line
% \setheadsepline[\textwidth]{0.5pt}
% set foot separation line
% \setfootsepline[\textwidth]{0.5pt}



\usepackage{tcolorbox}
% \usepackage{tikz}
% \tcbuselibrary{listings}
\tcbuselibrary{many}
\tcbset{fonttitle=\footnotesize}

\usepackage{array}

\allowdisplaybreaks

% \usepackage{epic, eepic}
\usepackage{epic}

\usepackage{natbib}
\bibliographystyle{plain}
\usepackage{url}

\usepackage{indentfirst}

\title{Lineare Algebra und Analytische Geometrie II - Übungsserie 01}
\author{Markus Pawellek}

\ihead{Lineare Algebra und Analytische Geometrie II \\ Übungsserie 01}
\ohead{Markus Pawellek - 144645 \\ markuspawellek@gmail.com}
\cfoot{\newline\newline\newline\pagemark/\pageref{LastPage} }

\begin{document}
	
	\section*{\centering Lineare Algebra und Analytische Geometrie II \\ Übungsserie 01} % (fold)
	\label{sec:lineare_algebra_und_analytische_geometrie_ii}

		\subsection*{Aufgabe 1} % (fold)
		\label{sub:aufgabe_1}
		
			(a): Sei $A\in \m{O}_3(\SR)$ mit $\det A = -1$. 
			Es ist nun zu zeigen, dass $-1$ ein Eigenwert von $A$ ist.
			Dafür betrachtet man die folgende Rechnung.
			\begin{alignat*}{3}
				\det(A+\m{I}) &= \det\boxb{(A+\m{I})^\m{T}} \\
					&= \det\curvb{A^\m{T} + \m{I}} \\
					\text{($A$ ist orthogonal)}\qquad &= \det\curvb{A^\m{T} + A^\m{T}A} \\
					&= \det A^\m{T} \det(\m{I}+A) \\
					\curvb{\det A = \det A^\m{T} = -1}\qquad &= - \det(A+\m{I}) \\
			\end{alignat*}
			\[
				\Rightarrow \quad \det(A+\m{I}) = 0 \quad \Rightarrow \quad -1\in\lambda(A)
			\]
			\qedbox

			(b): Sei nun $A\in\m{O}_2(\SR)$ mit $\det A = -1$.
			Um nun zu beweisen, dass $A$ immer die Spiegelung an einer Achse darstellt, soll als Erstes eine Matrix eingeführt werden, welche eine solche Spiegelung beschreibt.
			Diese soll dann auf Orthogonalität und Determinante untersucht werden.
			Im Anschluss wird gezeigt, dass sich jede Matrix $A$ mit den angegebenen Eigenschaften als eine solche Matrix schreiben lässt.

			Einer der beiden normierten Richtungsvektoren der Achse sei mit $e\in\SR^2, \norm{e} = 1$ bezeichnet.
			Es sei dann
			\[
				e := \curvb{
					\begin{matrix}
						e_x \\ e_y
					\end{matrix}
				},
				\qquad
				e^\perp := \curvb{
					\begin{matrix}
						-e_y \\ e_x
					\end{matrix}
				}
			\]
			Folglich ergibt sich $\angleb{e,e^\perp} = 0$ und damit bildet $\curlb{e,e^\perp}$ eine Orthonormalbasis.
			Sei nun $S\in \m{M}_2(\SR)$ mit
			\[
				S := \curvb{e,e^\perp} = \curvb{
					\begin{matrix}
						e_x & e_y \\ -e_y & e_x
					\end{matrix}
				}\qquad
				\Rightarrow \ \det S = e_x^2 + e_y^2 = \norm{e}^2 = 1
			\]
			Daraus folgt nun $S \in \m{SO}_2$.
			$S$ muss also gerade eine Drehung auf die Orthonormalbasis $\curlb{e,e^\perp}$ beschreiben.
			Für diese Orthonormalbasis ist die Spiegelachse gerade die Abszisse.
			Im Koordinatensystem ergibt sich also die Spiegelmatrix $S_x$ zu
			\[
				S_x :=\curvb{
					\begin{matrix}
						1 & 0 \\ 0 & -1	
					\end{matrix}	
				}
			\]
			Nach dem Ausführen der Spiegelung transformiert man das System nun durch $S^\m{T}$ zurück in das ursprüngliche.
			Für die Spiegelmatrix $S_e$ der beliebigen Achse ergibt sich also
			\[
				S_e = S^\m{T}S_xS =
				\curvb{
					\begin{matrix}
						e_x & -e_y \\ e_y & e_x
					\end{matrix}
				}
				\curvb{
					\begin{matrix}
						1 & 0 \\ 0 & -1	
					\end{matrix}	
				}\curvb{
					\begin{matrix}
						e_x & e_y \\ -e_y & e_x
					\end{matrix}
				} = \curvb{
					\begin{matrix}
						e_x^2 - e_y^2 & 2e_xe_y \\ 2e_xe_y & -(e_x^2 - e_y^2)
					\end{matrix}
				}
			\]
			Dabei ist $S_e\in\m{O}_2(\SR)$ mit $\det S_e = 1$, da $S,S_x\in\m{O}_2(\SR)$ mit $\det S = -\det S_x = 1$.

			Die Spaltenvektoren $a,a^\perp$ von $A$ müssen nun nach der Charakterisierung über orthogonale Matrizen eine Orthonormalbasis des $\SR^2$ bilden.
			Sei nun $a\in\SR^2$ mit $\norm{a}=1$.
			Dann gibt es genau zwei Möglichkeiten $a_1^\perp,a_2^\perp\in\SR$ einen orthogonalen Einheitsvektor zu $a$ zu wählen.
			\[
				a_1^\perp := \curvb{
					\begin{matrix}
						-a_y \\ a_x
					\end{matrix}
				},
				\qquad
				a_2^\perp := \curvb{
					\begin{matrix}
						a_y \\ -a_x
					\end{matrix}
				}
			\]
			Dabei gilt
			\[ \det(a,a_1^\perp) = a_x^2 + a_y^2 = 1 ,\qquad \det(a,a_2^\perp) = -\curvb{a_x^2 + a_y^2} = -1 \]
			Nach der Voraussetzung $\det A = -1$ muss also $a^\perp = a_2^\perp$ sein und $A$ damit die folgende Form haben.
			\[
				A = \curvb{
					\begin{matrix}
						a_x & a_y \\ a_y & -a_x	
					\end{matrix}	
				}
			\]
			Es ist nun zu zeigen, dass es ein $e\in\SR^2, \norm{e}=1$ gibt, sodass $A = S_e$.
			Dafür ist das folgende Gleichungssystem zu lösen.
			\begin{alignat*}{3}
				a_x &= e_x^2 - e_y^2 \\
				a_y &= 2e_xe_y
			\end{alignat*}
			\begin{alignat*}{3}
				\Rightarrow \quad 1+a_x = (e_x^2 + e_y^2) + e_x^2 - e_y^2 = 2e_x^2 \\
				1-a_x = (e_x^2 + e_y^2) - e_x^2 + e_y^2 = 2e_y^2
			\end{alignat*}
			\[ \Rightarrow \quad \abs{e_x} = \sqrt{\frac{1+a_x}{2}},\qquad \abs{e_y} = \sqrt{\frac{1-a_x}{2}} \]
			Die Vorzeichen bestimmen sich nun aus der zweiten Gleichung.
			Die Lösung besteht damit aus den zwei Vektoren $e$ und $-e$ mit
			\[ e := \curvb{ \sqrt{\frac{1+a_x}{2}} \quad \sqrt{\frac{1-a_x}{2}} } \]
			Dies ist logisch, da $S_e = S_{-e}$ sein muss. \qedbox

		% subsection aufgabe_1 (end)

		% \newpage

		\subsection*{Aufgabe 2} % (fold)
		\label{sub:aufgabe_2}
		
			Seien $A\in\m{M}_2(\SR)$ und $b_A$ die zugehörige Bilinearformen mit
			\[
				A = \curvb{
					\begin{matrix}
						2 & 1 \\ 1 & 2
					\end{matrix}	
				}
			\]
			Dann ist $A$ symmetrisch und positiv definit, da für alle $x\in\SR^2, x\neq 0$ gilt
			\[ x^\m{T}Ax = 2x_1^2 + x_1x_2 + x_1x_2 + 2x_2^2 = x_1^2 + x_2^2 + (x_1+x_2)^2 > 0 \]
			Damit lässt sich nun das Gram-Schmidtsche Orthonormierungsverfahren anwenden.
			Die gegebene (frei gewählte) Orthonormalbasis bestehe hier aus den kartesischen Einheitsvektoren $e_1,e_2$.
			Dann sei
			\[ v_1 := \frac{e_1}{\norm{e_1}_A} = \frac{e_1}{\sqrt{b_A(e_1,e_1)}} = \frac{e_1}{\sqrt{2}} \quad \Rightarrow \quad \norm{v_1}_A = 1 \]
			Weiterhin sei dann
			\[ v_2 := \frac{e_2 - b_A(e_2,v_1)v_1}{\norm{e_2 - b_A(e_2,v_1)v_1}_A} = \frac{\curvb{-\frac{1}{2} \quad 1}^\m{T}}{\norm{\curvb{-\frac{1}{2} \quad 1}^\m{T}}_A} = \frac{\sqrt{2}}{\sqrt{3}} \begin{pmatrix}-\frac{1}{2}\\1\end{pmatrix} \ \Rightarrow \quad \norm{v_2}_A=1 \]
			Es folgt nach den Verfahrensregeln
			\[ b_A(v_1,v_2) = 0 \]
			$\curlb{v_1,v_2}$ stellt damit eine Orthonormalbasis in $\SR^2$ bezüglich $b_A$ dar.

		% subsection aufgabe_2 (end)

		% \newpage

		\subsection*{Aufgabe 3} % (fold)
		\label{sub:aufgabe_3}
		
			Sei $n\in\SN$.
			Sei weiterhin $b:\SR^n\times\SR^n\longrightarrow\SR$ eine beliebige Bilinearform.
			Dann definiert man $s,a:\SR^n\times\SR^n\longrightarrow\SR$ durch die folgenden wohldefinierten Ausdrücke.
			\[ s(x,y):=\frac{1}{2}\boxb{b(x,y) + b(y,x)} ,\qquad a(x,y):=\frac{1}{2}\boxb{b(x,y)-b(y,x)} \]
			Sowohl $s$ als $a$ sind Linearkombinationen von $b$ und damit wieder Bilinearformen.
			Es gilt nun für alle $x,y\in\SR^n$
			\begin{alignat*}{4}
				s(x,y) &= \frac{1}{2}\boxb{b(x,y) + b(y,x)} &&= &\frac{1}{2}\boxb{b(y,x) + b(x,y)} &=& s(y,x) \\
				a(x,y) &= \frac{1}{2}\boxb{b(x,y) - b(y,x)} &&= -&\frac{1}{2}\boxb{b(y,x) - b(x,y)} &=& -a(y,x)
			\end{alignat*}
			Damit ist $s$ also symmetrisch und $a$ schiefsymmetrisch.
			Weiterhin folgt nun für alle $x,y\in\SR^n$
			\[ s(x,y) + a(x,y) = \frac{1}{2}\boxb{b(x,y) + b(y,x)} + \frac{1}{2}\boxb{b(x,y) - b(y,x)} = b(x,y) \]
			oder auch $b=s+a$. \qedbox

		% subsection aufgabe_3 (end)

		% \newpage

		\subsection*{Aufgabe 4} % (fold)
		\label{sub:aufgabe_4}
		
			Sei $A\in\m{M}_n(\SR)$ für ein $n\in\SN$. Es ist nun folgende Aussage zu zeigen.
			\[ A \text{ ist positiv definit} \quad \Rightarrow \quad \det A \neq 0 \]
			Nach Aussagenlogik ergibt sich eine äquivalente Aussage zu
			\[ \det A = 0 \quad \Rightarrow \quad A \text{ ist nicht positiv definit} \]
			Sei also nun $\det A = 0$.
			Dann gibt es ein $x\in\SR$ mit $x\neq 0$, sodass $Ax = 0$ gilt.
			Für dieses $x$ gilt dann aber auch
			\[ x^\m{T}Ax = 0 \not>0 \quad \Rightarrow \quad A \text{ ist nicht positiv definit} \]
			\qedbox

		% subsection aufgabe_4 (end)
	
	% section lineare_algebra_und_analytische_geometrie_ii (end)

\end{document}