\input{pre}

\usepackage{titling}
\title{Lineare Algebra und Analytische Geometrie II \\ Übungsserie 02}
\author{Markus Pawellek \\ 144645}
% \email{markuspawellek@gmail.com}
\newcommand{\email}{markuspawellek@gmail.com}


\newcommand{\equivalent}{\Longleftrightarrow}

\DeclareMathOperator{\tr}{tr}



\usepackage{fancyhdr}

\fancypagestyle{titlestyle}{
	\fancyhf{}
	\fancyfoot[C]{\footnotesize\bigskip\thepage/\pageref{LastPage}}
	\renewcommand{\footrulewidth}{0.4pt}
	\renewcommand{\headrulewidth}{0pt}
}

\fancypagestyle{mainstyle}{
	\fancyhf{}
	\fancyfoot[C]{\footnotesize\bigskip\thepage/\pageref{LastPage}}
	\fancyhead[LO,RE]{\footnotesize \thetitle} %left
	\fancyhead[RO,LE]{\footnotesize \theauthor} %right
	\renewcommand{\footrulewidth}{0.5pt}
	\renewcommand{\headrulewidth}{0.5pt}
}

\pagestyle{mainstyle}


\newcommand{\articletitle}{
	\thispagestyle{titlestyle}
	\hrule
	\section*{\centering \thetitle} % (fold)
	\noindent
	\parbox[b][][c]{0.5\textwidth}{\raggedright{\theauthor}}\hfill\parbox[b][][c]{0.5\textwidth}{\raggedleft{\email}}\\
	\hrule
	\bigskip
}


\begin{document}

	\articletitle

		\subsection*{Aufgabe 1} % (fold)
		\label{sub:aufgabe_1}
		
			Sei $K$ ein Körper, $V = \m{M}_n(K)$ für ein $n\in\SN$ und $b:V\times V\longrightarrow K$ eine Bilinearform auf $V$.

			\textbf{(a):}
			Es gilt für alle $A,B\in \m{M}_n(K)$
			\[ b(A,B) = \tr(AB) = \sum_{i=1}^{n}\sum_{j=1}^{n}a_{ij}b_{ji} = \sum_{j=1}^{n}\sum_{i=1}^{n}b_{ji}a_{ij} = \tr(BA) = b(B,A) \]
			$b$ ist also symmetrisch.
			Es reicht also die Linearität im ersten Argument zu zeigen, um die Bilinearität nachzuweisen.
			Für alle $A,B,C\in\m{M}_n(K)$ und alle $\lambda\in K$ gilt
			\begin{alignat*}{3}
				b(A+\lambda B,C) &= \sum_{i,j=1}^{n} (a_{ij} + \lambda b_{ij})c_{ji} = \sum_{i,j=1}^{n} \curvb{a_{ij}c_{ji} + \lambda b_{ij}c_{ji}} \\
					&= \sum_{i,j=1}^{n} a_{ij}c_{ji} + \lambda \sum_{i,j=1}^{n} b_{ij}c_{ji} = b(A,C) + \lambda b(B,C)
			\end{alignat*}
			$b$ ist also eine Bilinearform.

			\textbf{(b):}
			Es gilt für alle $A,B\in \m{M}_n(K)$
			\[ b(A,B) = \tr(AB-BA) = \tr(AB) - \tr(BA) = \tr(AB) - \tr(AB) = 0 \]
			$b$ ist damit die triviale Bilinearform und besitzt damit die Nullmatrix als darstellende Matrix bezüglich einer beliebigen Basis.

			\textbf{(c):}
			Es seien $b(A,B):=\det(AB)$ für alle $A,B\in V$ und 
			\[
				A:=
				\begin{pmatrix}
					1 & -2 \\ -2 & 1
				\end{pmatrix}
				\quad \implies \quad \det A = -3,\quad -1\in\lambda(A)
			\]
			\[ \implies \quad b(A+\m{I},\m{I}) = \det(A+\m{I}) = 0 \neq -2 = \det A + \det \m{I} = b(A,\m{I}) + b(\m{I},\m{I}) \]
			Damit ist $b$ keine Bilinearform.

			\textbf{(d):}
			Sei $i,j\in\SN$ mit $i,j\leq n$.
			Es gilt für alle $A,B\in \m{M}_n(K)$
			\[ b(A,B) = (AB)_{ij} = \sum_{k=1}^n a_{ik}b_{kj} \]
			Der Nachweis, dass $b$ bilinear ist kann vollkommen analog zu Teilaufgabe (a) behandelt werden.

		% subsection aufgabe_1 (end)
	
		\subsection*{Aufgabe 2} % (fold)
		\label{sub:aufgabe_2}
		
			\textbf{(a):}
			Es sei $b$ die Bilinearform auf $\SR^3$, welche bezüglich der Basis $B:=\curlb{e_1,e_2,e_3}$ durch die Matrix $A\in\m{M}_3(\SR)$ gegeben ist.
			Weiterhin sei $\tilde{B}:=\curlb{\tilde{e}_1, \tilde{e}_2, \tilde{e}_3}$ eine weitere Basis mit der zugehörigen Basistransformationsmatrix $C\in\m{GL}_3(\SR)$.
			\[
				A =
				\begin{pmatrix}
					1 & 2 & 3 \\
					4 & 5 & 6 \\
					7 & 8 & 9
				\end{pmatrix},\qquad
				C =
				\begin{pmatrix}
					1 & 1 & 1 \\
					-1 & 0 & 1 \\
					0 & 1 & 1
				\end{pmatrix}
			\]
			Aus der Vorlesung ist dann bekannt, dass sich die Matrix $A$ der Bilinearform $b$ gemäß der folgenden Formel transformiert.
			\[ \tilde{A} = C^\m{T} A C \]
			\[
				\tilde{A} =
				\begin{pmatrix}
					1 & -1 & 0 \\
					1 & 0 & 1 \\
					1 & 1 & 1
				\end{pmatrix}
				\begin{pmatrix}
					1 & 2 & 3 \\
					4 & 5 & 6 \\
					7 & 8 & 9
				\end{pmatrix}
				\begin{pmatrix}
					1 & 1 & 1 \\
					-1 & 0 & 1 \\
					0 & 1 & 1
				\end{pmatrix}
				=
				\begin{pmatrix}
					0 & -6 & -9 \\
					-2 & 20 & 30 \\
					-3 & 30 & 45
				\end{pmatrix}
			\]

			\textbf{(b):}
			Diese Lösung dieser Aufgabe kann analog zu (a) berechnet werden.
			\[
				A =
				\begin{pmatrix}
					0 & 2 & 1 \\
					-2 & 2 & 0 \\
					-1 & 0 & 3
				\end{pmatrix},\qquad
				C =
				\begin{pmatrix}
					1 & 0 & -1 \\
					2 & 1 & 1 \\
					-1 & -1 & -3
				\end{pmatrix}
			\]
			\[ \tilde{A} = 
				\begin{pmatrix}
					1 & 2 & -1 \\
					0 & 1 & -1 \\
					-1 & 1 & -3
				\end{pmatrix}
				\begin{pmatrix}
					0 & 2 & 1 \\
					-2 & 2 & 0 \\
					-1 & 0 & 3
				\end{pmatrix}
				\begin{pmatrix}
					1 & 0 & -1 \\
					2 & 1 & 1 \\
					-1 & -1 & -3
				\end{pmatrix}
				=
				\begin{pmatrix}
					11 & 8 & 15 \\
					6 & 5 & 12 \\
					11 & 10 & 29
				\end{pmatrix}
			\]

		% subsection aufgabe_2 (end)

		\subsection*{Aufgabe 3} % (fold)
		\label{sub:aufgabe_3}
		
			Sei $K$ ein Körper und $V=K^n$ für ein $n\in\SN$.
			Seien weiterhin $W_1,W_2,W\subset V$ Unterräume und $b$ eine symmetrische Bilinearform auf $V$.

			\textbf{(a):}
			Sei $v\in V$ beliebig. Dann gilt
			\begin{alignat*}{3}
				&v\in (W_1+W_2)_b^\perp \\
				\equivalent \quad &b(v,w) = 0 \text{ für alle } w\in W_1+W_2 \\
				\equivalent \quad &b(v,x+y) = 0 \text{ für alle } x\in W_1,y\in W_2 \\
				\stackrel{(\star)}{\equivalent} \quad &b(v,x) = b(v,y) = 0 \text{ für alle } x\in W_1,y\in W_2 \\
				\equivalent \quad &v\in (W_1)_b^\perp \wedge v\in (W_2)_b^\perp \\
				\equivalent \quad &v\in (W_1)_b^\perp \cap (W_2)_b^\perp
			\end{alignat*}
			Die Rückrichtung von $(\star)$ ist klar aufgrund der Bilinearität von $b$.
			Für die vorhandene Richtung kann man, da es sich bei $W_1$ und $W_2$ um Unterräume handelt, entweder $x=0$ oder $y=0$ setzen.
			\qedbox

			\textbf{(b):}
			Sei $v\in W$ beliebig.
			\begin{alignat*}{3}
				&\implies \quad b(v,w) = 0 \text{ für alle } w\in W_b^\perp \\
				&\equivalent \quad v \in \curvb{W_b^\perp}_b^\perp
			\end{alignat*}
			Dies bedeutet gerade $W\subset \curvb{W_b^\perp}_b^\perp$.
			\qedbox

			\textbf{(c):}
			Es seien nun $W_1\subset W_2$ und $v\in (W_2)_b^\perp$ beliebig.
			\begin{alignat*}{3}
				&\implies \quad b(v,w) = 0 \text{ für alle } w\in W_2 \\
				&\stackrel{\mathclap{(W_1\subset W_2)}}{\implies} \quad b(v,w) = 0 \text{ für alle } w\in W_1\subset W_2 \\
				&\implies \quad v\in (W_1)_b^\perp
			\end{alignat*}
			Es folgt also $(W_2)_b^\perp \subset (W_1)_b^\perp$.
			\qedbox

		% subsection aufgabe_3 (end)

		\subsection*{Aufgabe 4} % (fold)
		\label{sub:aufgabe_4}
		
			Seien $V=\m{M}_2(\SR)$ und
			\[ b:V\times V\longrightarrow \SR,\qquad b(A,B) := \det(A+B)-\det A - \det B \]
			Dann ist $b$ symmetrisch, da Folgendes für alle $A,B\in\m{M}_2(\SR)$ gilt.
			\begin{alignat*}{3}
				b(A,B) &= \det(A+B)-\det A - \det B \\
					&= \det(B+A) - \det B - \det A \\
					&= b(B,A)
			\end{alignat*}
			Seien nun $A,B\in\m{M}_2(\SR)$ beliebig mit
			\[
				A:=
				\begin{pmatrix}
					a_{11} & a_{12} \\ a_{21} & a_{22}
				\end{pmatrix} ,\qquad
				B:=
				\begin{pmatrix}
					b_{11} & b_{12} \\ b_{21} & b_{22}					
				\end{pmatrix}
			\]
			Dann folgt durch Berechnung der Determinanten
			\begin{alignat*}{3}
				b(A,B) &=\ && (a_{11} + b_{11})(a_{22}+b_{22}) - (a_{12}+b_{12})(a_{21}+b_{21}) \\
					& &&- a_{11}a_{22} + a_{12}a_{21} - b_{11}b_{22} + b_{12}b_{21} \\
					&=\ && a_{11}b_{22} - a_{12}b_{21} - a_{21}b_{12} + a_{22}b_{11}
			\end{alignat*}
			Es ist aufgrund der bereits gezeigten Symmetrie ausreichend die Linearität im ersten Argument zu zeigen, um Bilinearität nachzuweisen.
			Für alle $A,B,C\in\m{M}_2(\SR)$ und alle $\lambda\in\SR$ gilt
			\begin{alignat*}{3}
				b(A+\lambda B,C) &= (a_{11} + \lambda b_{11})c_{22} - (a_{12} + \lambda b_{12})c_{21} \\
					&- (a_{21} + \lambda b_{21})c_{12} + (a_{22} + \lambda b_{22})c_{11} \\
					&= a_{11}c_{22} + \lambda b_{11}c_{22} - a_{12}c_{21} + \lambda b_{12}c_{21} \\
					&- a_{21}c_{12} + \lambda b_{21}c_{12} + a_{22}c_{11} + \lambda b_{22}c_{11} \\
					&= b(A,C) + \lambda b(B,C)
			\end{alignat*}
			
			Es soll nun jede $2\times 2$-Matrix $A\in\m{M}_2(\SR)$ mit einem vierdimensionalen Vektor $a\in\SR^4$ identifiziert werden.
			\[ a_1 := a_{11},\quad a_2 := a_{12},\quad a_3 := a_{21},\quad a_4:=a_{22} \]
			Dann lässt sich $b$ für $x,y\in\SR^4$ schreiben als
			\[ b(x,y) = x_1y_4 - x_2y_3 - x_3y_2 + x_4y_1 \]
			Bezüglich der kanonischen Basis ergibt sich damit auch sofort die zu $b$ zugehörige Matrix $B$
			\[
				B = 
				\begin{pmatrix}
					0 & 0 & 0 & 1 \\
					0 & 0 & -1 & 0 \\
					0 & -1 & 0 & 0 \\
					1 & 0 & 0 & 0
				\end{pmatrix}
				,\qquad b(x,y) = x^\m{T}By
			\]

			Aus der Vorlesung ist bekannt, dass sich die Signatur einer Bilinearform gerade aus der Anzahl positiver und negativer Eigenwerte der zugehörigen Matrix ergibt.
			Für das charakteristische Polynom von $B$ folgt
			\[ \det (B-\lambda\m{I}) = \curvb{1-\lambda^2}^2 = \curvb{1-\lambda}^2\curvb{1+\lambda}^2 \]
			Damit besitzt das charakteristische Polynom für $\lambda = \pm 1$ jeweils eine doppelte Nullstelle.
			Es gibt damit zwei positive und zwei negative Eigenwerte.
			Die Signatur von $b$ ist also $(2,2)$.


			Seien nun $x,y\in\SR^4$.
			Dann ergibt sich deren Spur durch
			\[ \tr x = x_1 + x_4,\qquad \tr y = y_1 + y_4 \]
			Dann folgt damit direkt (analoges gilt auch für $y$)
			\[ \tr x = 0 \quad \equivalent \quad x_1=-x_4 \]
			Es seien jetzt $x,y$ spurfrei und $S:=\set{x\in\SR^4}{\tr x = 0}$.
			Dann gilt
			\[ b\vert_S(x,y) = -2x_1y_1 - x_2y_3 - x_3y_2 \]
			und es folgt auch hier direkt die zugehörige Matrix
			\[
				B\vert_S = 
				\begin{pmatrix}
					-2 & 0 & 0 & 0 \\
					0 & 0 & -1 & 0 \\
					0 & -1 & 0 & 0 \\
					0 & 0 & 0 & 0
				\end{pmatrix}
			\]

		% subsection aufgabe_4 (end)

	% section lineare_algebra_und_analytische_geometrie_ii_\_bungsserie_02 (end)

\end{document}