\input{pre}

\usepackage{titling}
\title{Lineare Algebra und Analytische Geometrie II \\ Übungsserie 03}
\author{Markus Pawellek \\ 144645}
% \email{markuspawellek@gmail.com}
\newcommand{\email}{\url{markuspawellek@gmail.com}}


\newcommand{\equivalent}{\Longleftrightarrow}

\DeclareMathOperator{\tr}{tr}



\usepackage{fancyhdr}

\fancypagestyle{titlestyle}{
	\fancyhf{}
	\fancyfoot[C]{\footnotesize\bigskip\thepage/\pageref{LastPage}}
	\renewcommand{\footrulewidth}{0.4pt}
	\renewcommand{\headrulewidth}{0pt}
}

\fancypagestyle{mainstyle}{
	\fancyhf{}
	\fancyfoot[C]{\footnotesize\bigskip\thepage/\pageref{LastPage}}
	\fancyhead[LO,RE]{\footnotesize \thetitle} %left
	\fancyhead[RO,LE]{\footnotesize \theauthor} %right
	\renewcommand{\footrulewidth}{0.5pt}
	\renewcommand{\headrulewidth}{0.5pt}
}

\pagestyle{mainstyle}


\newcommand{\articletitle}{
	\thispagestyle{titlestyle}
	\hrule
	\section*{\centering \thetitle} % (fold)
	\noindent
	\parbox[b][][c]{0.5\textwidth}{\raggedright{\theauthor}}\hfill\parbox[b][][c]{0.5\textwidth}{\raggedleft{\email}}\\
	\hrule
	\bigskip
}

\newcommand{\transp}[1]{ {#1}^\m{T} }
\newcommand{\inv}[1]{ {#1}^{-1} }
\newcommand{\conj}[1]{ \overline{#1} }
\newcommand{\idmat}{\m{I}}

\begin{document}

	\articletitle

	\subsection*{Aufgabe 1} % (fold)
	\label{sub:aufgabe_1}
	
		Sei $n\in\SN$.
		Aus Linearer Algebra I ist bereits bekannt, dass die Menge $\m{M}_n(\SC)$ der $n\times n$-Matrizen über den komplexen Zahlen ein reeller Vektorraum ist.
		Für die Menge der hermiteschen Matrizen gilt offenbar
		\[ \m{H}_n := \set{ A\in \m{M}_n(\SC) }{ \transp{\conj{A}} = A } \subset \m{M}_n(\SC) \]
		Es reicht damit zu zeigen, dass $\m{H}_n$ einen linearen Unterraum von $\m{M}_n(\SC)$ darstellt.

		Es ist klar, dass $\m{H}_n \neq \emptyset$, da $\idmat\in\m{H}_n$.
		Seien nun $A,B\in\m{H}_n$ und $\lambda\in\SR$ (das heißt $\conj{\lambda}=\lambda$).
		\begin{alignat*}{3}
			\transp{\conj{\curvb{A+\lambda B}}} &= \transp{ \curvb{\conj{A} + \conj{\lambda}\, \conj{B}} } = \transp{\conj A} + \lambda \transp{\conj{B}} = A + \lambda B
		\end{alignat*}
		\[ \implies \quad A+\lambda B \in \m{H}_n \]
		$\m{H}_n$ ist also abgeschlossen unter Linearkombinationen und damit ein linearer Unterraum von $\m{M}_n(\SC)$ bezüglich der reellen Zahlen. $\hfill \Box$

		Seien jetzt $E_{ij}:=\curvb{e^{ij}_{pq}}_{1\leq p,q\leq n}$ für $i,j\in\SN,\ i,j\leq n, j\geq i$ mit
		\[ e^{ij}_{pq} := \delta_{ip}\delta_{jq} + \delta_{iq}\delta_{jp} - \delta_{ij}\delta_{ip}\delta_{jq} \]
		und $\tilde{E}_{km}:=\curvb{\tilde{e}^{km}_{pq}}_{1\leq p,q\leq n}$ für $k,m\in\SN,\ k,m\leq n,\ m>k$ mit
		\[ \tilde{e}^{km}_{pq} := \delta_{kp}\delta_{mq} - \delta_{kq}\delta_{mp} \]
		Dann bildet die folgende Menge ($i$ steht hier für die imaginäre Einheit) eine Basis von $\m{H}_n$.
		\[ \set{ E_{km} }{ k,m\in\SN,\ k,m\leq n,\ m\geq k } \cup \set{ i\tilde{E}_{km} }{ k,m\in\SN,\ k,m\leq n,\ m>k } \]

	% subsection aufgabe_1 (end)

	\subsection*{Aufgabe 2} % (fold)
	\label{sub:aufgabe_2}
	
		Sei $n\in\SN$.
		Seien $A,B\in\m{M}_n(\SC)$ positiv definite hermitesche Matrizen.
		
		\textbf{Matrix $A+B$:}\\
		In Aufgabe 1 wurde gezeigt, dass dann auch $A+B$ eine hermitesche Matrix bildet (Abgeschlossenheit der Linearkombination).
		Weiterhin gilt für alle $v\in\SC^n\setminus \curlb{0}$
		\[ \transp{\conj{v}}(A+B)v = \transp{\conj{v}}(Av + Bv) = \underbrace{\transp{\conj{v}}Av}_{>0} + \underbrace{\transp{\conj{v}}Bv}_{>0} > 0 \]
		$A+B$ ist damit auch eine positiv hermitesche Matrix. $\hfill \Box$ \\

		\textbf{Matrix $A^{-1}$:}\\
		Für die Inverse einer Matrix (sofern diese existiert) gilt
		\[ \idmat = \inv{A}A = A\inv{A} = \transp{\conj{\curvb{ A\inv{A} }}} = \transp{\curvb{\conj{A}\ \conj{\inv{A}}}} = \transp{\conj{\inv{A}}} \transp{\conj{A}} = \transp{\conj{\inv{A}}} A \]
		Die Inverse ist eindeutig bestimmt.
		Es muss also $\transp{\conj{\inv{A}}} = \inv{A}$ gelten.
		Da die Inverse von $A$ existiert, ist die zugehörige lineare Abbildung eine bijektive Abbildung.
		Für alle $v\in\SC^n$ gibt es also ein eindeutig bestimmtes $w\in\SC^n$ mit $v=Aw$.
		Es folgt also für alle $v\in\SC^n\setminus\curlb{0}$ mit dem zugehörigen $w\in\SC^n\setminus\curlb{0}$ (das heißt $Aw = v$)
		\[ \transp{\conj{v}}\inv{A}v = \transp{\conj{\curvb{Aw}}}\inv{A}\curvb{Aw} = \transp{\conj{w}} \underbrace{\transp{\conj{A}}}_{=A} \underbrace{\inv{A}A}_{=\idmat} w = \transp{\conj{w}}Aw > 0 \]
		$\inv{A}$ ist hermitesch und positiv definit. $\hfill \Box$\\

		\textbf{Matrix $A^2$:}\\
		Es folgt direkt, dass $A^2$ hermitesch ist.
		\[ \transp{\conj{A^2}} = \transp{\curvb{\conj{A}\,\conj{A}}} = \transp{\conj{A}}\transp{\conj{A}} = AA = A^2 \]
		Weiterhin gilt nach bekannten Rechenregeln
		\[ \transp{\conj{v}}A^2 v = \transp{\curvb{ \transp{A}\conj{v} }}(Av) = \transp{\conj{\curvb{\transp{\conj{A}}v}}}(Av) = \transp{\conj{\curvb{Av}}}\curvb{Av} \]
		Im Allgemeinen muss $A$ nicht invertierbar sein.
		Es gibt also für bestimmte $A$ ein $v\in\SC^n\setminus\curlb{0}$, sodass $Av=0$.
		Für dieses $v$ folgt dann auch
		\[ \transp{\conj{v}}A^2 v = \transp{\conj{\curvb{Av}}}\curvb{Av} = 0 \]
		$A^2$ ist im Allgemeinen also nicht positiv definit. $\hfill \Box$\\

		\textbf{Matrix $AB$:}\\
		Seien $A,B$ so gewählt, dass $AB\neq BA$, wie zum Beispiel
		\[
			A=
			\begin{pmatrix}
				1 & i \\ -i & 1
			\end{pmatrix}
			,\qquad 
			B = 
			\begin{pmatrix}
				1 & 1 \\ 1& 1
			\end{pmatrix}
		\]
		Nach bekannten Rechenregeln gilt
		\[ \transp{\conj{AB}} = \transp{\conj{B}}\transp{\conj{A}} = BA \neq AB \]
		$AB$ ist also nicht hermitesch (theoretisch gesehen, ist nun nicht sicher gestellt, ob positive Definitheit überhaupt Sinn ergibt).
		$AB$ muss auch nicht positiv definit sein.
		Dafür wählt man $A$ mit $\det A = 0$ und $B=A$.
		Nach der vorherigen Aussage $AB=A^2$ dann nicht positiv definit. $\hfill \Box$\\

	% subsection aufgabe_2 (end)

	\subsection*{Aufgabe 3} % (fold)
	\label{sub:aufgabe_3}
	
		Sei $A\in\m{M}_2(\SC)$ die folgende hermitesche Matrix und $\chi$ das zugehörige charakteristische Polynom.
		\[
			A =
			\begin{pmatrix}
				1 & i \\
				-i & 1
			\end{pmatrix}
			,\quad \implies \quad \chi(\lambda) = \det(A-\lambda\idmat) = (1-\lambda)^2 -1
		\]
		Für die Eigenwerte $\lambda_1,\lambda_2\in\lambda(A)$ folgt also (durch Setzen von $\chi(\lambda)=0$)
		\[ \lambda_1=2, \qquad \lambda_2=0 \]
		Die Eigenvektoren $x\in\SC^2$ bezüglich $\lambda_1$ ergeben sich durch
		\[ x_1 + ix_2 = 2x_1 \quad \implies \quad x_1 = ix_2 \]
		Man wähle nun $x_2=1$ und normiere den dadurch entstehenden Eigenvektor.
		Dieser Vektor soll hier $b_1$ genannt werden.
		Analog geht man für einen normierten Eigenvektor bezüglich $\lambda_2$ vor.
		\[ \implies \quad b_1 = \frac{1}{\sqrt{2}} \begin{pmatrix} i \\ 1 \end{pmatrix},\qquad b_2 = \frac{1}{\sqrt{2}}\begin{pmatrix} -i \\ 1 \end{pmatrix} \]
		Nach dem bekannten Spektralsatz sind diese beiden Vektoren orthogonal zueinander.
		Die Matrix $C:=(b_1,b_2)$ stellt dem zufolge eine unitäre Matrix dar.
		Man bestätigt nun durch Rechnung
		\[
			\transp{\conj{C}}AC = \frac{1}{2}
			\begin{pmatrix}
				-i & 1 \\
				i & 1
			\end{pmatrix}
			\begin{pmatrix}
				1 & i \\
				-i & 1
			\end{pmatrix}
			\begin{pmatrix}
				i & -i \\
				1 & 1
			\end{pmatrix}
			= \frac{1}{2}
			\begin{pmatrix}
				-i & 1 \\
				i & 1
			\end{pmatrix}
			\begin{pmatrix}
				2i & 0 \\
				2 & 0
			\end{pmatrix}
			=
			\begin{pmatrix}
				2 & 0 \\
				0 & 0
			\end{pmatrix}
		\]

	% subsection aufgabe_3 (end)

	\subsection*{Aufgabe 4} % (fold)
	\label{sub:aufgabe_4}
	
		Das Vorgehen in dieser Aufgabe ist vollkommen analog zu Aufgabe 3.
		Aus diesem Grund sei auch die Benennung aller Variablen die gleiche.\\

		\textbf{(a):}
		\[
			A :=
			\begin{pmatrix}
				1 & 2 \\
				2 & 1
			\end{pmatrix}
			,\quad \implies \quad \lambda_1 = 3, \quad \lambda_2 = -1
		\]
		\[
			\implies \quad
			b_1 = \frac{1}{\sqrt{2}}
			\begin{pmatrix}
				1 \\ 1
			\end{pmatrix}
			,\qquad
			b_2 = \frac{1}{\sqrt{2}}
			\begin{pmatrix}
				1 \\ -1
			\end{pmatrix}
		\]
		\[
			\transp{\conj{C}}AC = \frac{1}{2}
			\begin{pmatrix}
				1 & 1 \\
				1 & -1
			\end{pmatrix}
			\begin{pmatrix}
				1 & 2 \\
				2 & 1
			\end{pmatrix}
			\begin{pmatrix}
				1 & 1 \\
				1 & -1
			\end{pmatrix}
			=
			\begin{pmatrix}
				3 & 0 \\
				0 & -1 
			\end{pmatrix}
		\]

		\textbf{(b):}
		\[
			A :=
			\begin{pmatrix}
				1 & 1 & 1 \\
				1 & 1 & 1 \\
				1 & 1 & 1
			\end{pmatrix}
			,\quad \implies \quad 3\lambda^2-\lambda^3=\lambda^2(3-\lambda) \stackrel{!}{=}0
		\]
		\[ \implies \quad \lambda_1=\lambda_2=0,\qquad \lambda_3 = 3 \]
		\[
			\implies \quad
			b_1 = \frac{1}{\sqrt{2}}
			\begin{pmatrix}
				0 \\ -1 \\ 1
			\end{pmatrix}
			,\quad
			b_2 = \frac{1}{\sqrt{6}}
			\begin{pmatrix}
				-2 \\ 1 \\ 1
			\end{pmatrix}
			,\quad
			b_3 = \frac{1}{\sqrt{3}}
			\begin{pmatrix}
				1 \\ 1 \\ 1
			\end{pmatrix}
		\]
		\[
			\transp{\conj{C}}AC = \frac{1}{2}
			\transp{\conj{C}}
			\begin{pmatrix}
				1 & 1 & 1 \\
				1 & 1 & 1 \\
				1 & 1 & 1
			\end{pmatrix}
			\begin{pmatrix}
				0 & -2/\sqrt{6} & 1/\sqrt{3} \\
				-1/\sqrt{2} & 1/\sqrt{6} & 1/\sqrt{3} \\
				1/\sqrt{2} & 1/\sqrt{6} & 1/\sqrt{3}
			\end{pmatrix}
			=
			\begin{pmatrix}
				0 & 0 & 0 \\
				0 & 0 & 0 \\
				0 & 0 & 3 
			\end{pmatrix}
		\]

		\textbf{(c):}
		\[
			A :=
			\begin{pmatrix}
				1 & 0 & 1 \\
				0 & 1 & 0 \\
				1 & 0 & 0
			\end{pmatrix}
			,\quad \implies \quad -1 + 2\lambda^2-\lambda^3=(\lambda - 1)\curvb{-\lambda^2+\lambda+1} \stackrel{!}{=}0
		\]
		\[ \implies \quad \lambda_1=1,\qquad \lambda_2=\frac{1+\sqrt{5}}{2},\qquad \lambda_3 = \frac{1-\sqrt{5}}{2} \]
		\[
			\implies \quad
			b_1 =
			\begin{pmatrix}
				0 \\ 1 \\ 0
			\end{pmatrix}
			,\quad
			b_2 = \frac{2}{\sqrt{5+\sqrt{5}}}
			\begin{pmatrix}
				\lambda_2 \\ 0 \\ 1
			\end{pmatrix}
			,\quad
			b_3 = \frac{2}{\sqrt{5-\sqrt{5}}}
			\begin{pmatrix}
				\lambda_3 \\ 0 \\ 1
			\end{pmatrix}
		\]
		\[
			\transp{\conj{C}}AC = \frac{1}{2}
			\transp{\conj{C}}
			\begin{pmatrix}
				1 & 0 & 1 \\
				0 & 1 & 0 \\
				1 & 0 & 0
			\end{pmatrix}
			\begin{pmatrix}
				0 & b_{21} & b_{31} \\
				1 & b_{22} & b_{32} \\
				0 & b_{23} & b_{33}
			\end{pmatrix}
			=
			\begin{pmatrix}
				1 & 0 & 0 \\
				0 & \lambda_2 & 0 \\
				0 & 0 & \lambda_3 
			\end{pmatrix}
		\]
	% subsection aufgabe_4 (end)

	% section lineare_algebra_und_analytische_geometrie_ii_\_bungsserie_02 (end)

\end{document}