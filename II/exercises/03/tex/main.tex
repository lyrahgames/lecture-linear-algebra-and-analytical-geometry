% documentclass: article used for scientific journals, short reports, program documentation, etc
% options: fontsize 11, generate document for double sided printing, a4-paper
\documentclass[10pt, twoside, a4paper, fleqn]{article}

% package for changing page layout
\usepackage{geometry}
\geometry{a4paper, lmargin=40mm, rmargin=45mm, tmargin=40mm, bmargin=45mm}
% set indentation
\setlength{\parindent}{1em}
% set factor for line spacing
% \linespread{1.0}\selectfont
% set (dynamic) additional line spacing
% \setlength{\parskip}{1ex plus 0.5ex minus 0.3ex}

% rigorous formatting (not too much hyphens)
% \fussy
% \sloppy

% package for changing page layout (used to indent whole paragraphs with adjustwidth)
\usepackage{changepage}

% input encoding for special characters (e.g. ä,ü,ö,ß), only for non english text
% options: utf8 as encoding standard, latin1
\usepackage[utf8]{inputenc}
% package for font encoding
\usepackage[T1]{fontenc}
% package for changing used language (especially for more than one language)
% options: ngerman (new spelling) or default: english
\usepackage[ngerman]{babel}
% package for times font
% \usepackage{times}
% package for latin modern fonts
\usepackage{lmodern}

% package for math symbols, functions and environments from ams(american mathematical society)
\usepackage{amsmath}
\usepackage{mathtools}
% package for extended symbols from ams
\usepackage{amssymb}
% package for math black board symbols (e.g. R,Q,Z,...)
\usepackage{bbm}
% package used for calligraphic math symbols
\usepackage{mathrsfs}
% package for extended symbols from stmaryrd(st mary road)
\usepackage{stmaryrd}
% package for more math blackboard symbols
\usepackage{dsfont}

% pack­age im­ple­ments scal­ing of the math ex­ten­sion font cmex; used for scaling math signs
\usepackage{exscale}

% package for including extern graphics plus scaling and rotating
\usepackage{graphicx}
%package for positioning figures
\usepackage{float}
% package for changing color of font and paper
% options: using names of default colors (e.g red, black)
% \usepackage[usenames]{color}
\usepackage[dvipsnames]{xcolor}
\definecolor{shadecolor}{gray}{0.9}
% package for customising captions
\usepackage[footnotesize, hang]{caption}
% package for customising enumerations (e.g. axioms)
\usepackage{enumitem}
% calc package reimplements \setcounter, \addtocounter, \setlength and \addtolength: commands now accept an infix notation expression
\usepackage{calc}
% package for creating framed, shaded, or differently highlighted regions that can break across pages; environments: framed, oframed, shaded, shaded*, snugshade, snugshade*, leftbar, titled-frame
\usepackage{framed}
% package for creating custom "list of"
% options: titles: do not intefere with standard headings for "list of"
\usepackage[titles]{tocloft}
% change enumeration style of equations
% \renewcommand\theequation{\thesection.\arabic{equation}}

% init list of math for definitions and theorems
\newcommand{\listofmathcall}{Verzeichnis der Definitionen und Sätze}
\newlistof{math}{mathlist}{\listofmathcall}
% add parentheses around argument
\newcommand{\parent}[1]{ \ifx&#1&\else (#1) \fi }
% unnumerated mathematical definition environment definiton
\newenvironment{mathdef*}[2]{
	\medskip
	\begin{tcolorbox}[colback=white, boxrule=0.5pt, colframe=black, breakable]
	\noindent
	{ \fontfamily{ppl}\selectfont \textbf{\textsc{#1:}} } ~ #2 
	\par \hfill\\ 
	\fontfamily{lmr}\selectfont \itshape
}{
	\end{tcolorbox}
	\medskip
}
% definitions for numerated mathematical definition environment
\newcounter{mathdefc}[section]
\newcommand*{\mathdefnum}{\thesection.\arabic{mathdefc}}
\renewcommand{\themathdefc}{\mathdefnum}
\newenvironment{mathdef}[2]{
	\refstepcounter{mathdefc}
	\addcontentsline{mathlist}{figure}{\protect{\numberline{\mathdefnum}#1 ~ #2}}
	\begin{mathdef*}{#1 \mathdefnum}{#2}
}{
	\end{mathdef*}
}
% standard mathdef calls
\newcommand{\definitioncall}{Definition}
\newenvironment{definition*}[1][]{ \begin{mathdef*}{\definitioncall}{\parent{#1}} }{ \end{mathdef*} }
\newenvironment{definition}[1][]{ \begin{mathdef}{\definitioncall}{\parent{#1}} }{ \end{mathdef} }
% unnumerated theorem environment definition
\newenvironment{maththeorem*}[2]{
	\begin{leftbar}%[boxrule=0pt, leftrule=3pt, arc=0pt, colback=white, colframe=black, enhanced jigsaw]
	\noindent
	{ \fontfamily{ppl}\selectfont \textbf{\textsc{#1:}} } ~ #2
	\par \hfill\\ 
	\fontfamily{lmr} \fontshape{it} \selectfont
}{ 
	\end{leftbar}
}
% definitions for numerated theorem environment
\newcounter{maththeoremc}[section]
\newcommand*\maththeoremnum{\thesection.\arabic{maththeoremc}}
\renewcommand{\themaththeoremc}{\maththeoremnum}
\newenvironment{maththeorem}[2]{
	\refstepcounter{maththeoremc}
	\addcontentsline{mathlist}{figure}{\protect{\qquad\numberline{\maththeoremnum}#1 ~ #2}}
	\begin{maththeorem*}{#1 \maththeoremnum}{#2}
}{
	\end{maththeorem*}
}
% standard maththeorem calls
\newcommand{\theoremcall}{Theorem}
\newenvironment{theorem*}[1][]{ \begin{maththeorem*}{\theoremcall}{\parent{#1}} }{ \end{maththeorem*} }
\newenvironment{theorem}[1][]{ \begin{maththeorem}{\theoremcall}{\parent{#1}} }{ \end{maththeorem} }
\newcommand{\lemmacall}{Lemma}
\newenvironment{lemma*}[1][]{ \begin{maththeorem*}{\lemmacall}{\parent{#1}} }{ \end{maththeorem*} }
\newenvironment{lemma}[1][]{ \begin{maththeorem}{\lemmacall}{\parent{#1}} }{ \end{maththeorem} }
\newcommand{\propositioncall}{Proposition}
\newenvironment{proposition*}[1][]{ \begin{maththeorem*}{\propositioncall}{\parent{#1}} }{ \end{maththeorem*} }
\newenvironment{proposition}[1][]{ \begin{maththeorem}{\propositioncall}{\parent{#1}} }{ \end{maththeorem} }
\newcommand{\corollarycall}{Korollar}
\newenvironment{corollary*}[1][]{ \begin{maththeorem*}{\corollarycall}{\parent{#1}} }{ \end{maththeorem*} }
\newenvironment{corollary}[1][]{ \begin{maththeorem}{\corollarycall}{\parent{#1}} }{ \end{maththeorem} }
% q.e.d. definition
\newcommand{\qed}{ \par \hfill \fontfamily{lmr} \fontshape{it} \selectfont \mbox{q.e.d.} \\}
\newcommand{\qedbox}{ \par \hfill $\Box$ \\ }
% proof environment definition for theorems
\newenvironment{mathproof}[1]{
	\par\hfill\\
	\noindent
	{ \fontfamily{lmr}\selectfont \small \textsc{#1:}}
	\normalfont
	\small
	\begin{adjustwidth}{1em}{}
	\medskip
}{ 
	\end{adjustwidth} 
	% \qed
	\qedbox
}
% standard mathproof calls
\newcommand{\proofcall}{Beweis}
\newenvironment{proof}{ \begin{mathproof}{\proofcall} }{ \end{mathproof} }
\newcommand{\proofideacall}{Beweisidee}
\newenvironment{proofidea}{ \begin{mathproof}{\proofideacall} }{ \end{mathproof} }

% new displaymath command, so that equations will not be stretched
\newcommand{\D}[1]{\mbox{$ #1 $}}
% make unnumerated equation
\newcommand{\E}[1]{\[ #1 \]}
% command for curly brackets
\newcommand{\curlb}[1]{\left\{ #1 \right\}}
% command for box brackets
\newcommand{\boxb}[1]{\left[ #1 \right]}
% command for parentheses/curved brackets
\newcommand{\curvb}[1]{\left( #1 \right)}
% command for angle brackets
\newcommand{\angleb}[1]{\left\langle #1 \right\rangle}
% command for floor brackets
\newcommand{\floorb}[1]{\left\lfloor #1 \right\rfloor}
% command for ceil brackets
\newcommand{\ceilb}[1]{\left\lceil #1 \right\rceil}
% command for creating sets
\newcommand{\set}[2]{ \left\{ #1 \enspace \middle\vert \enspace #2 \right\} }
% command for absolute value
\newcommand{\abs}[1]{\left\vert #1 \right\vert}
\newcommand{\norm}[1]{\left\| #1 \right\|}
% commands for writing limits
\newcommand{\limit}[3]{\, \longrightarrow \, #1, \ #2 \longrightarrow #3}
\newcommand{\Limit}[2]{\lim_{#1 \rightarrow #2}}
% command for differential
\newcommand{\diff}{\mathrm{d}}
\newcommand{\Diff}{\mathrm{D}}
% command for derivative
\newcommand{\Deriv}[3][]{\Diff_{#2}^{#1}#3}
\newcommand{\deriv}[3][]{\dfrac{\diff^{#1}#2(#3)}{\diff #3^{#1}}}
% command for integral
\newcommand{\integral}[4]{\int_{#1}^{#2} #3\ \diff #4}
\newcommand{\Integral}[4]{\int\limits_{#1}^{#2} #3\ \diff #4}
\newcommand{\iintegral}[2]{\int #1\ \diff #2} % indefinite integral
% mathematical definitions (standard sets)
\newcommand{\SR}{\mathds{R}} % real numbers
\newcommand{\SC}{\mathds{C}} % complex numbers
\newcommand{\SN}{\mathds{N}} % natural numbers
\newcommand{\SZ}{\mathds{Z}} % integral numbers
\newcommand{\SQ}{\mathds{Q}} % rational numbers
\newcommand{\SP}{\mathcal{P}} % power set
\newcommand{\SFP}{\mathds{P}} % polynom functions
\newcommand{\SFC}{\mathrm{C}} % complex valued functions (continous or differentiable)
\newcommand{\SFL}{\mathcal{L}} % space of integrable functions
\newcommand{\SFLL}{\mathrm{L}} % space of integrable function classes
\newcommand{\SH}{\mathcal{H}} % hilbert space
% mathematical standard functions
\DeclareMathOperator{\real}{Re} % real part
\DeclareMathOperator{\imag}{Im} % imaginary part
\DeclareMathOperator{\diag}{diag}
\DeclareMathOperator{\id}{Id}
\newcommand{\FF}{\mathcal{F}} % fourier transform
\newcommand{\FE}{\mathbb{E}} % expectation
\DeclareMathOperator{\var}{var} % variance
\newcommand{\FN}{\mathcal{N}} % normal distribution

\newcommand{\m}[1]{\mathrm{#1}}

% command for physical units
\newcommand{\unit}[1]{\, \mathrm{#1}}


% package for init listings(non-formatted  text) e.g. different source codes
\usepackage{listings}


% definitions for listing colors
\definecolor{codeDarkGray}{gray}{0.2}
\definecolor{codeGray}{gray}{0.4}
\definecolor{codeLightGray}{rgb}{0.94,0.94,0.91}
\definecolor{codeBorder}{rgb}{0.34,0.24,0.21}
% predefinitions for listings
\newcommand{\listingcall}{Listing}
\newlength{\listingframemargin}
\setlength{\listingframemargin}{1em}
\newlength{\listingmargin}
\setlength{\listingmargin}{0.08\textwidth}
% \newlength{\listingwidth}
% \setlength{\listingwidth}{ ( \textwidth - \listingmargin * \real{2} + \listingframemargin * \real{2} ) }
% definitions for list of listings
\newcommand{\listoflistingscall}{\listingcall -Verzeichnis}
\newlistof{listings}{listinglist}{\listoflistingscall}
% style definition for standard code listings
\lstdefinestyle{std}{
	belowcaptionskip=0.5\baselineskip,
	breaklines=true,
	frameround=tttt,
	% frame=false,
	xleftmargin=0em,
	xrightmargin=0em,
	showstringspaces=false,
	showtabs=false,
	% tab=\smash{\rule[-.2\baselineskip]{.4pt}{\baselineskip}\kern.5em},
	basicstyle= \fontfamily{pcr}\selectfont\footnotesize\bfseries,
	keywordstyle= \bfseries\color{MidnightBlue}, %\color{codeDarkGray},
	commentstyle= \itshape\color{codeGray},
	identifierstyle=\color{codeDarkGray},
	stringstyle=\color{BurntOrange}, %\color{codeDarkGray},
	numberstyle=\tiny\ttfamily,
	% numbers=left,
	numbersep = 1em,
	% stepnumber = 1,
	% captionpos=t,
	tabsize=4,
	% backgroundcolor=\color{codebLightGray},
	rulecolor=\color{codeBorder},
	framexleftmargin=\listingframemargin,
	framexrightmargin=\listingframemargin
}
% definition for unnumerated listing
\newcommand{\inputlistingn}[3][]{
	\begin{center}
		\begin{adjustwidth}{\listingmargin}{\listingmargin}
			\centerline{ {\fontfamily{lmr}\selectfont \footnotesize \listingcall:}\quad {\footnotesize #2} }
			\lstinputlisting[style=std, #1]{#3}
		\end{adjustwidth}
	\end{center}
}
% definition for numerated listing
\newcounter{listingc}[section]
\newcommand*\listingnum{\thesection.\arabic{listingc}}
\renewcommand{\thelistingc}{\listingnum}
\newcommand{\inputlisting}[3][]{
	\refstepcounter{listingc}
	\addcontentsline{listinglist}{figure}{\protect{\numberline{\listingnum:} #2 } }
	% \inputlistingn[#1]{#2}{#3}
	\begin{center}
		\begin{adjustwidth}{\listingmargin}{\listingmargin}
			\centerline{ {\fontfamily{lmr}\selectfont \footnotesize \listingcall~\listingnum:}\quad {\footnotesize #2} }
			\lstinputlisting[style=std, #1]{#3}
		\end{adjustwidth}
	\end{center}
}


% package for including csv-tables from file
% \usepackage{csvsimple}
% package for creating, loading and manipulating databases
\usepackage{datatool}

% package for converting eps-files to pdf-files and then include them
\usepackage{epstopdf}
% use another program (ps2pdf) for converting
% !!! important: set shell_escape=1 in /etc/texmf/texmf.cnf (Linux/Ubuntu 12.04) for allowing to use other programs
% !!!			or use the command line with -shell-escape
% \epstopdfsetup{outdir=./}
% \epstopdfDeclareGraphicsRule{.eps}{pdf}{.pdf}{
% ps2pdf -dEPSCrop #1 \OutputFile
% }


% package for reference to last page (output number of last page)
\usepackage{lastpage}
% package for using header and footer
% options: automate terms of right and left marks
% \usepackage[automark]{scrpage2}
% \setlength{\headheight}{4\baselineskip}
% set style for footer and header
% \pagestyle{scrheadings}
% \pagestyle{headings}
% clear pagestyle for redefining
% \clearscrheadfoot
% set header and footer: use <xx>head/foot[]{Text} (i...inner, o...outer, c...center, o...odd, e...even, l...left, r...right)

% use that for mark to last page: \pageref{LastPage}
% set header separation line
% \setheadsepline[\textwidth]{0.5pt}
% set foot separation line
% \setfootsepline[\textwidth]{0.5pt}



\usepackage{tcolorbox}
% \usepackage{tikz}
% \tcbuselibrary{listings}
\tcbuselibrary{many}
\tcbset{fonttitle=\footnotesize}

\usepackage{array}

\allowdisplaybreaks

% \usepackage{epic, eepic}
\usepackage{epic}

\usepackage{natbib}
\bibliographystyle{plain}
\usepackage{url}

\usepackage{indentfirst}

\usepackage{titling}
\title{Lineare Algebra und Analytische Geometrie II \\ Übungsserie 03}
\author{Markus Pawellek \\ 144645}
% \email{markuspawellek@gmail.com}
\newcommand{\email}{\url{markuspawellek@gmail.com}}


\newcommand{\equivalent}{\Longleftrightarrow}

\DeclareMathOperator{\tr}{tr}



\usepackage{fancyhdr}

\fancypagestyle{titlestyle}{
	\fancyhf{}
	\fancyfoot[C]{\footnotesize\bigskip\thepage/\pageref{LastPage}}
	\renewcommand{\footrulewidth}{0.4pt}
	\renewcommand{\headrulewidth}{0pt}
}

\fancypagestyle{mainstyle}{
	\fancyhf{}
	\fancyfoot[C]{\footnotesize\bigskip\thepage/\pageref{LastPage}}
	\fancyhead[LO,RE]{\footnotesize \thetitle} %left
	\fancyhead[RO,LE]{\footnotesize \theauthor} %right
	\renewcommand{\footrulewidth}{0.5pt}
	\renewcommand{\headrulewidth}{0.5pt}
}

\pagestyle{mainstyle}


\newcommand{\articletitle}{
	\thispagestyle{titlestyle}
	\hrule
	\section*{\centering \thetitle} % (fold)
	\noindent
	\parbox[b][][c]{0.5\textwidth}{\raggedright{\theauthor}}\hfill\parbox[b][][c]{0.5\textwidth}{\raggedleft{\email}}\\
	\hrule
	\bigskip
}

\newcommand{\transp}[1]{ {#1}^\m{T} }
\newcommand{\inv}[1]{ {#1}^{-1} }
\newcommand{\conj}[1]{ \overline{#1} }
\newcommand{\idmat}{\m{I}}

\begin{document}

	\articletitle

	\subsection*{Aufgabe 1} % (fold)
	\label{sub:aufgabe_1}
	
		Sei $n\in\SN$.
		Aus Linearer Algebra I ist bereits bekannt, dass die Menge $\m{M}_n(\SC)$ der $n\times n$-Matrizen über den komplexen Zahlen ein reeller Vektorraum ist.
		Für die Menge der hermiteschen Matrizen gilt offenbar
		\[ \m{H}_n := \set{ A\in \m{M}_n(\SC) }{ \transp{\conj{A}} = A } \subset \m{M}_n(\SC) \]
		Es reicht damit zu zeigen, dass $\m{H}_n$ einen linearen Unterraum von $\m{M}_n(\SC)$ darstellt.

		Es ist klar, dass $\m{H}_n \neq \emptyset$, da $\idmat\in\m{H}_n$.
		Seien nun $A,B\in\m{H}_n$ und $\lambda\in\SR$ (das heißt $\conj{\lambda}=\lambda$).
		\begin{alignat*}{3}
			\transp{\conj{\curvb{A+\lambda B}}} &= \transp{ \curvb{\conj{A} + \conj{\lambda}\, \conj{B}} } = \transp{\conj A} + \lambda \transp{\conj{B}} = A + \lambda B
		\end{alignat*}
		\[ \implies \quad A+\lambda B \in \m{H}_n \]
		$\m{H}_n$ ist also abgeschlossen unter Linearkombinationen und damit ein linearer Unterraum von $\m{M}_n(\SC)$ bezüglich der reellen Zahlen. $\hfill \Box$

		Seien jetzt $E_{ij}:=\curvb{e^{ij}_{pq}}_{1\leq p,q\leq n}$ für $i,j\in\SN,\ i,j\leq n, j\geq i$ mit
		\[ e^{ij}_{pq} := \delta_{ip}\delta_{jq} + \delta_{iq}\delta_{jp} - \delta_{ij}\delta_{ip}\delta_{jq} \]
		und $\tilde{E}_{km}:=\curvb{\tilde{e}^{km}_{pq}}_{1\leq p,q\leq n}$ für $k,m\in\SN,\ k,m\leq n,\ m>k$ mit
		\[ \tilde{e}^{km}_{pq} := \delta_{kp}\delta_{mq} - \delta_{kq}\delta_{mp} \]
		Dann bildet die folgende Menge ($i$ steht hier für die imaginäre Einheit) eine Basis von $\m{H}_n$.
		\[ \set{ E_{km} }{ k,m\in\SN,\ k,m\leq n,\ m\geq k } \cup \set{ i\tilde{E}_{km} }{ k,m\in\SN,\ k,m\leq n,\ m>k } \]

	% subsection aufgabe_1 (end)

	\subsection*{Aufgabe 2} % (fold)
	\label{sub:aufgabe_2}
	
		Sei $n\in\SN$.
		Seien $A,B\in\m{M}_n(\SC)$ positiv definite hermitesche Matrizen.
		
		\textbf{Matrix $A+B$:}\\
		In Aufgabe 1 wurde gezeigt, dass dann auch $A+B$ eine hermitesche Matrix bildet (Abgeschlossenheit der Linearkombination).
		Weiterhin gilt für alle $v\in\SC^n\setminus \curlb{0}$
		\[ \transp{\conj{v}}(A+B)v = \transp{\conj{v}}(Av + Bv) = \underbrace{\transp{\conj{v}}Av}_{>0} + \underbrace{\transp{\conj{v}}Bv}_{>0} > 0 \]
		$A+B$ ist damit auch eine positiv hermitesche Matrix. $\hfill \Box$ \\

		\textbf{Matrix $A^{-1}$:}\\
		Für die Inverse einer Matrix (sofern diese existiert) gilt
		\[ \idmat = \inv{A}A = A\inv{A} = \transp{\conj{\curvb{ A\inv{A} }}} = \transp{\curvb{\conj{A}\ \conj{\inv{A}}}} = \transp{\conj{\inv{A}}} \transp{\conj{A}} = \transp{\conj{\inv{A}}} A \]
		Die Inverse ist eindeutig bestimmt.
		Es muss also $\transp{\conj{\inv{A}}} = \inv{A}$ gelten.
		Da die Inverse von $A$ existiert, ist die zugehörige lineare Abbildung eine bijektive Abbildung.
		Für alle $v\in\SC^n$ gibt es also ein eindeutig bestimmtes $w\in\SC^n$ mit $v=Aw$.
		Es folgt also für alle $v\in\SC^n\setminus\curlb{0}$ mit dem zugehörigen $w\in\SC^n\setminus\curlb{0}$ (das heißt $Aw = v$)
		\[ \transp{\conj{v}}\inv{A}v = \transp{\conj{\curvb{Aw}}}\inv{A}\curvb{Aw} = \transp{\conj{w}} \underbrace{\transp{\conj{A}}}_{=A} \underbrace{\inv{A}A}_{=\idmat} w = \transp{\conj{w}}Aw > 0 \]
		$\inv{A}$ ist hermitesch und positiv definit. $\hfill \Box$\\

		\textbf{Matrix $A^2$:}\\
		Es folgt direkt, dass $A^2$ hermitesch ist.
		\[ \transp{\conj{A^2}} = \transp{\curvb{\conj{A}\,\conj{A}}} = \transp{\conj{A}}\transp{\conj{A}} = AA = A^2 \]
		Weiterhin gilt nach bekannten Rechenregeln
		\[ \transp{\conj{v}}A^2 v = \transp{\curvb{ \transp{A}\conj{v} }}(Av) = \transp{\conj{\curvb{\transp{\conj{A}}v}}}(Av) = \transp{\conj{\curvb{Av}}}\curvb{Av} \]
		Im Allgemeinen muss $A$ nicht invertierbar sein.
		Es gibt also für bestimmte $A$ ein $v\in\SC^n\setminus\curlb{0}$, sodass $Av=0$.
		Für dieses $v$ folgt dann auch
		\[ \transp{\conj{v}}A^2 v = \transp{\conj{\curvb{Av}}}\curvb{Av} = 0 \]
		$A^2$ ist im Allgemeinen also nicht positiv definit. $\hfill \Box$\\

		\textbf{Matrix $AB$:}\\
		Seien $A,B$ so gewählt, dass $AB\neq BA$, wie zum Beispiel
		\[
			A=
			\begin{pmatrix}
				1 & i \\ -i & 1
			\end{pmatrix}
			,\qquad 
			B = 
			\begin{pmatrix}
				1 & 1 \\ 1& 1
			\end{pmatrix}
		\]
		Nach bekannten Rechenregeln gilt
		\[ \transp{\conj{AB}} = \transp{\conj{B}}\transp{\conj{A}} = BA \neq AB \]
		$AB$ ist also nicht hermitesch (theoretisch gesehen, ist nun nicht sicher gestellt, ob positive Definitheit überhaupt Sinn ergibt).
		$AB$ muss auch nicht positiv definit sein.
		Dafür wählt man $A$ mit $\det A = 0$ und $B=A$.
		Nach der vorherigen Aussage $AB=A^2$ dann nicht positiv definit. $\hfill \Box$\\

	% subsection aufgabe_2 (end)

	\subsection*{Aufgabe 3} % (fold)
	\label{sub:aufgabe_3}
	
		Sei $A\in\m{M}_2(\SC)$ die folgende hermitesche Matrix und $\chi$ das zugehörige charakteristische Polynom.
		\[
			A =
			\begin{pmatrix}
				1 & i \\
				-i & 1
			\end{pmatrix}
			,\quad \implies \quad \chi(\lambda) = \det(A-\lambda\idmat) = (1-\lambda)^2 -1
		\]
		Für die Eigenwerte $\lambda_1,\lambda_2\in\lambda(A)$ folgt also (durch Setzen von $\chi(\lambda)=0$)
		\[ \lambda_1=2, \qquad \lambda_2=0 \]
		Die Eigenvektoren $x\in\SC^2$ bezüglich $\lambda_1$ ergeben sich durch
		\[ x_1 + ix_2 = 2x_1 \quad \implies \quad x_1 = ix_2 \]
		Man wähle nun $x_2=1$ und normiere den dadurch entstehenden Eigenvektor.
		Dieser Vektor soll hier $b_1$ genannt werden.
		Analog geht man für einen normierten Eigenvektor bezüglich $\lambda_2$ vor.
		\[ \implies \quad b_1 = \frac{1}{\sqrt{2}} \begin{pmatrix} i \\ 1 \end{pmatrix},\qquad b_2 = \frac{1}{\sqrt{2}}\begin{pmatrix} -i \\ 1 \end{pmatrix} \]
		Nach dem bekannten Spektralsatz sind diese beiden Vektoren orthogonal zueinander.
		Die Matrix $C:=(b_1,b_2)$ stellt dem zufolge eine unitäre Matrix dar.
		Man bestätigt nun durch Rechnung
		\[
			\transp{\conj{C}}AC = \frac{1}{2}
			\begin{pmatrix}
				-i & 1 \\
				i & 1
			\end{pmatrix}
			\begin{pmatrix}
				1 & i \\
				-i & 1
			\end{pmatrix}
			\begin{pmatrix}
				i & -i \\
				1 & 1
			\end{pmatrix}
			= \frac{1}{2}
			\begin{pmatrix}
				-i & 1 \\
				i & 1
			\end{pmatrix}
			\begin{pmatrix}
				2i & 0 \\
				2 & 0
			\end{pmatrix}
			=
			\begin{pmatrix}
				2 & 0 \\
				0 & 0
			\end{pmatrix}
		\]

	% subsection aufgabe_3 (end)

	\subsection*{Aufgabe 4} % (fold)
	\label{sub:aufgabe_4}
	
		Das Vorgehen in dieser Aufgabe ist vollkommen analog zu Aufgabe 3.
		Aus diesem Grund sei auch die Benennung aller Variablen die gleiche.\\

		\textbf{(a):}
		\[
			A :=
			\begin{pmatrix}
				1 & 2 \\
				2 & 1
			\end{pmatrix}
			,\quad \implies \quad \lambda_1 = 3, \quad \lambda_2 = -1
		\]
		\[
			\implies \quad
			b_1 = \frac{1}{\sqrt{2}}
			\begin{pmatrix}
				1 \\ 1
			\end{pmatrix}
			,\qquad
			b_2 = \frac{1}{\sqrt{2}}
			\begin{pmatrix}
				1 \\ -1
			\end{pmatrix}
		\]
		\[
			\transp{\conj{C}}AC = \frac{1}{2}
			\begin{pmatrix}
				1 & 1 \\
				1 & -1
			\end{pmatrix}
			\begin{pmatrix}
				1 & 2 \\
				2 & 1
			\end{pmatrix}
			\begin{pmatrix}
				1 & 1 \\
				1 & -1
			\end{pmatrix}
			=
			\begin{pmatrix}
				3 & 0 \\
				0 & -1 
			\end{pmatrix}
		\]

		\textbf{(b):}
		\[
			A :=
			\begin{pmatrix}
				1 & 1 & 1 \\
				1 & 1 & 1 \\
				1 & 1 & 1
			\end{pmatrix}
			,\quad \implies \quad 3\lambda^2-\lambda^3=\lambda^2(3-\lambda) \stackrel{!}{=}0
		\]
		\[ \implies \quad \lambda_1=\lambda_2=0,\qquad \lambda_3 = 3 \]
		\[
			\implies \quad
			b_1 = \frac{1}{\sqrt{2}}
			\begin{pmatrix}
				0 \\ -1 \\ 1
			\end{pmatrix}
			,\quad
			b_2 = \frac{1}{\sqrt{6}}
			\begin{pmatrix}
				-2 \\ 1 \\ 1
			\end{pmatrix}
			,\quad
			b_3 = \frac{1}{\sqrt{3}}
			\begin{pmatrix}
				1 \\ 1 \\ 1
			\end{pmatrix}
		\]
		\[
			\transp{\conj{C}}AC = \frac{1}{2}
			\transp{\conj{C}}
			\begin{pmatrix}
				1 & 1 & 1 \\
				1 & 1 & 1 \\
				1 & 1 & 1
			\end{pmatrix}
			\begin{pmatrix}
				0 & -2/\sqrt{6} & 1/\sqrt{3} \\
				-1/\sqrt{2} & 1/\sqrt{6} & 1/\sqrt{3} \\
				1/\sqrt{2} & 1/\sqrt{6} & 1/\sqrt{3}
			\end{pmatrix}
			=
			\begin{pmatrix}
				0 & 0 & 0 \\
				0 & 0 & 0 \\
				0 & 0 & 3 
			\end{pmatrix}
		\]

		\textbf{(c):}
		\[
			A :=
			\begin{pmatrix}
				1 & 0 & 1 \\
				0 & 1 & 0 \\
				1 & 0 & 0
			\end{pmatrix}
			,\quad \implies \quad -1 + 2\lambda^2-\lambda^3=(\lambda - 1)\curvb{-\lambda^2+\lambda+1} \stackrel{!}{=}0
		\]
		\[ \implies \quad \lambda_1=1,\qquad \lambda_2=\frac{1+\sqrt{5}}{2},\qquad \lambda_3 = \frac{1-\sqrt{5}}{2} \]
		\[
			\implies \quad
			b_1 =
			\begin{pmatrix}
				0 \\ 1 \\ 0
			\end{pmatrix}
			,\quad
			b_2 = \frac{2}{\sqrt{5+\sqrt{5}}}
			\begin{pmatrix}
				\lambda_2 \\ 0 \\ 1
			\end{pmatrix}
			,\quad
			b_3 = \frac{2}{\sqrt{5-\sqrt{5}}}
			\begin{pmatrix}
				\lambda_3 \\ 0 \\ 1
			\end{pmatrix}
		\]
		\[
			\transp{\conj{C}}AC = \frac{1}{2}
			\transp{\conj{C}}
			\begin{pmatrix}
				1 & 0 & 1 \\
				0 & 1 & 0 \\
				1 & 0 & 0
			\end{pmatrix}
			\begin{pmatrix}
				0 & b_{21} & b_{31} \\
				1 & b_{22} & b_{32} \\
				0 & b_{23} & b_{33}
			\end{pmatrix}
			=
			\begin{pmatrix}
				1 & 0 & 0 \\
				0 & \lambda_2 & 0 \\
				0 & 0 & \lambda_3 
			\end{pmatrix}
		\]
	% subsection aufgabe_4 (end)

	% section lineare_algebra_und_analytische_geometrie_ii_\_bungsserie_02 (end)

\end{document}