\input{pre}

\usepackage{titling}
\title{Lineare Algebra und Analytische Geometrie II \\ Übungsserie 05}
\author{Markus Pawellek \\ 144645}
% \email{markuspawellek@gmail.com}
\newcommand{\email}{markuspawellek@gmail.com}


\newcommand{\equivalent}{\Longleftrightarrow}

\DeclareMathOperator{\tr}{tr}



\usepackage{fancyhdr}

\fancypagestyle{titlestyle}{
	\fancyhf{}
	\fancyfoot[C]{\footnotesize\bigskip\thepage/\pageref{LastPage}}
	\renewcommand{\footrulewidth}{0.4pt}
	\renewcommand{\headrulewidth}{0pt}
}

\fancypagestyle{mainstyle}{
	\fancyhf{}
	\fancyfoot[C]{\footnotesize\bigskip\thepage/\pageref{LastPage}}
	\fancyhead[LO,RE]{\footnotesize \thetitle} %left
	\fancyhead[RO,LE]{\footnotesize \theauthor} %right
	\renewcommand{\footrulewidth}{0.5pt}
	\renewcommand{\headrulewidth}{0.5pt}
}

\pagestyle{mainstyle}


\newcommand{\articletitle}{
	\thispagestyle{titlestyle}
	\hrule
	\section*{\centering \thetitle} % (fold)
	\noindent
	\parbox[b][][c]{0.5\textwidth}{\raggedright{\theauthor}}\hfill\parbox[b][][c]{0.5\textwidth}{\raggedleft{\email}}\\
	\hrule
	\bigskip
}

\newcommand{\transp}[1]{ {#1}^\m{T} }
\newcommand{\inv}[1]{ {#1}^{-1} }
\newcommand{\conj}[1]{ \overline{#1} }
\newcommand{\idmat}{\m{I}}

\begin{document}

	\articletitle

	\section*{Aufgabe 1} % (fold)
	\label{sec:aufgabe_1}
	
		Seien $a,b,c\in\SR^3$ beliebig.
		Dann gilt nach der Formel des Kreuzproduktes
		\[
			a\times(b\times c) =
			a \times
			\begin{pmatrix}
				b_2c_3 - b_3c_2 \\
				b_3c_1 - b_1c_3 \\
				b_1c_2 - b_2c_1
			\end{pmatrix}
			=
			\begin{pmatrix}
				a_2(b_1c_2 - b_2c_1) - a_3(b_3c_1 - b_1c_3) \\
				a_3(b_2c_3 - b_3c_2) - a_1(b_1c_2 - b_2c_1) \\
				a_1(b_3c_1 - b_1c_3) - a_2(b_2c_3 - b_3c_2)
			\end{pmatrix}
		\]
		\[
			=
			\begin{pmatrix}
				b_1a_2c_2 + b_1a_3c_3 - c_1a_2b_2 + c_1a_3b_3 \\
				b_2a_1c_1 + b_2a_3c_3 - c_2a_1b_1 + c_2a_3b_3 \\
				b_3a_2c_2 + b_3a_1c_1 - c_3a_2b_2 + c_3a_1b_1
			\end{pmatrix}
			+
			\begin{pmatrix}
				a_1b_1c_1 \\
				a_2b_2c_2 \\
				a_3b_3c_3
			\end{pmatrix}
			-
			\begin{pmatrix}
				a_1b_1c_1 \\
				a_2b_2c_2 \\
				a_3b_3c_3
			\end{pmatrix}
		\]
		\[
			=
			\begin{pmatrix}
				b_1(a_1c_1 + a_2c_2 + a_3c_3) - c_1(a_1b_1 + a_2b_2 + a_3b_3) \\
				b_2(a_1c_1 + a_2c_2 + a_3c_3) - c_2(a_1b_1 + a_2b_2 + a_3b_3) \\
				b_3(a_1c_1 + a_2c_2 + a_3c_3) - c_3(a_1b_1 + a_2b_2 + a_3b_3)
			\end{pmatrix}
			=
			\angleb{a,c}b - \angleb{a,b}c
		\]
		Verwendet man nun, dass das Kreuzprodukt alternierend ist, dann erhält man die Jacobi-Identität direkt durch Einsetzen.
		\begin{alignat*}{3}
			(a\times b)\times c + (b\times c)\times a + (c\times a)\times b &= &&\angleb{a,c}b - \angleb{b,c}a \\
			&+ &&\angleb{a,b}c - \angleb{a,c}b \\
			&+ &&\angleb{b,c}a - \angleb{a,b}c = 0
		\end{alignat*}
		\qedbox

	% section aufgabe_1 (end)

	\section*{Aufgabe 2} % (fold)
	\label{sec:aufgabe_2}
	
		

	% section aufgabe_2 (end)

\end{document}