\input{pre}

\usepackage{titling}
\title{Lineare Algebra und Analytische Geometrie II \\ Übungsserie 07}
\author{Markus Pawellek \\ 144645}
% \email{markuspawellek@gmail.com}
\newcommand{\email}{markuspawellek@gmail.com}


\newcommand{\equivalent}{\Longleftrightarrow}

\DeclareMathOperator{\tr}{tr}



\usepackage{fancyhdr}

\fancypagestyle{titlestyle}{
	\fancyhf{}
	\fancyfoot[C]{\footnotesize\bigskip\thepage/\pageref{LastPage}}
	\renewcommand{\footrulewidth}{0.4pt}
	\renewcommand{\headrulewidth}{0pt}
}

\fancypagestyle{mainstyle}{
	\fancyhf{}
	\fancyfoot[C]{\footnotesize\bigskip\thepage/\pageref{LastPage}}
	\fancyhead[RO,LE]{\footnotesize \thetitle} %left
	\fancyhead[RE,LO]{\footnotesize \theauthor} %right
	\renewcommand{\footrulewidth}{0.5pt}
	\renewcommand{\headrulewidth}{0.5pt}
}

\pagestyle{mainstyle}


\newcommand{\articletitle}{
	\thispagestyle{titlestyle}
	\hrule
	\section*{\centering \thetitle} % (fold)
	\noindent
	\parbox[b][][c]{0.5\textwidth}{\raggedright{\theauthor}}\hfill\parbox[b][][c]{0.5\textwidth}{\raggedleft{\email}}\\
	\hrule
	\bigskip
}

\newcommand{\transp}[1]{ {#1}^\m{T} }
\newcommand{\inv}[1]{ {#1}^{-1} }
\newcommand{\conj}[1]{ \overline{#1} }
\newcommand{\idmat}{\m{I}}
\newcommand{\define}{\coloneqq}

\begin{document}

	\articletitle

	\section*{Aufgabe 1} % (fold)
	\label{sec:aufgabe_1}
	
		\textbf{(a):}
		Sei $A\in\m{M}_2(\SR)$ mit dem charakteristischen Polynom $\chi$
		\[
			A \define
			\begin{pmatrix}
				3 & 0 & 8 \\
				3 & -1 & 6 \\
				-2 & 0 & -5
			\end{pmatrix}
			\quad \implies \chi(\lambda) = (3-\lambda)(1+\lambda)(5+\lambda) - 16(1+\lambda) = -(1+\lambda)^3
		\]
		$A$ besitzt damit nur den Eigenwert $-1$ mit algebraischer Vielfachheit $3$.
		Für die Eigenvektoren $\transp{(x,y,z)}\in\SR^3$ gilt dann
		\begin{alignat*}{3}
			&\! \begin{drcases}
			3x + 8z &= -x \\
			3x - y + 6z &= -y \\
			-2x - 5z &= -z
			\end{drcases}
			\implies \quad x=-2z \quad \implies \quad \dim \m{E}_{-1}(A) = 2
		\end{alignat*}
		Der Eigenwert besitzt demzufolge die geometrische Vielfachheit $2$.
		Die Jordan-Normalform $J$ von $A$ wird also gerade durch zwei Jordan-Blöcke beschrieben.
		\[
			A\sim \diag\boxb{ J(2,-1), J(1,-1) } =
			\begin{pmatrix}
				-1 & 1 & 0 \\
				0 & -1 & 0 \\
				0 & 0 & -1
			\end{pmatrix}
		\]

		Notiz zur Nebenrechnung:
		Die Transformationsmatrix ist gerade
		\[
			\begin{pmatrix}
				8 & 0 & 0 \\
				6 & 0 & 1 \\
				-4 & 1 & 0
			\end{pmatrix}
		\]

		\textbf{(b):}
		Sei $A\in\m{M}_2(\SR)$ mit dem charakteristischen Polynom $\chi$
		\[
			A \define
			\begin{pmatrix}
				1 & -3 & 0 & 3 \\
				-2 & -6 & 0 & 13 \\
				0 & -3 & 1 & 3 \\
				-1 & -4 & 0 & 8
			\end{pmatrix}
			\quad \implies \chi(\lambda) = (1-\lambda)^4
		\]
		$A$ besitzt damit nur den Eigenwert $\lambda\define 1$ mit algebraischer Vielfachheit $4$.
		Für die Eigenvektoren $\transp{(x,y,z,t)}\in\SR^4$ gilt dann
		\begin{alignat*}{3}
			\begin{drcases}
				x = 3t \\
				y = t
			\end{drcases}
			\implies \quad \dim \m{E}_{1}(A) = 2
		\end{alignat*}
		Der Eigenwert besitzt demzufolge die geometrische Vielfachheit $2$.
		Die Jordan-Normalform $J$ von $A$ wird also gerade durch zwei Jordan-Blöcke beschrieben.
		Für die Größe der Jordan-Blöcke betrachtet man $B\define A-\lambda\idmat$.
		Dann ist $\dim(\ker B) = 2$.
		Durch direktes Rechnen folgt
		\[ B^3 = 0 \quad \implies \quad \dim(\ker B^3) = 4 \]
		$B$ ist also nilpotent.
		Es gilt damit nach bereits bewiesenen Sätzen
		\[ 4 =\dim(\ker B^3) > \dim(\ker B^2) = 3 > \dim(\ker B) =2 \]
		Die Dimensionen nehmen damit immer um $1$ ab.
		Demzufolge besitzt der größte Jordan-Blöcke die Dimension $3$.
		\[
			A\sim \diag\boxb{ J(3,1), J(1,1) } =
			\begin{pmatrix}
				1 & 1 & 0 & 0 \\
				0 & 1 & 1 & 0 \\
				0 & 0 & 1 & 0 \\
				0 & 0 & 0 & 1
			\end{pmatrix}
		\]

		Notiz zur Nebenrechnung:
		Die Transformationsmatrix ist gerade
		\[
			\begin{pmatrix}
				3 & 0 & 1 & 3 \\
				1 & -2 & 0 & 1 \\
				3 & 0 & 0 & 0 \\
				1 & -1 & 0 & 1
			\end{pmatrix}
		\]

	% section aufgabe_1 (end)

	\section*{Aufgabe 2} % (fold)
	\label{sec:aufgabe_2}
	
		Sei $A\define J(r,\lambda)$ für ein $r\in\SN$ und ein $\lambda\in\SR,\lambda\neq 0$.
		Dann besitzt $A$ nur den Eigenwert $\lambda$ mit der algebraischen Vielfachheit $r$ und der geometrischen Vielfachheit $1$.
		Damit besitzt $\inv{A}$ nur den Eigenwert $\inv{\lambda}$ und die selben Eigenvektoren (siehe Lineare Algebra I).
		Der Eigenraum von $\inv{A}$ zum Eigenwert $\inv{\lambda}$ besitzt also auch die Dimension $1$.
		$\inv{\lambda}$ besitzt also bezüglich $\inv{A}$ die gleich algebraische und geometrische Vielfachheit wie $\lambda$ bezüglich $A$.
		Die Jordan-Normalform $J$ von $\inv{A}$ setzt sich demnach aus einem Jordan-Block zusammen.
		\[ \inv{A} \sim J\curvb{ r, \inv{\lambda} } \]

		Für $A^2$ gelten analoge Schlussfolgerungen.
		$A^2$ besitzt nur den Eigenwert $\lambda^2$.
		Der zugehörige Eigenraum besitzt die Dimension $1$.
		Die Jordan-Normalform $J$ von $A^2$ besteht wieder nur aus einem Jordan-Block.
		\[ A^2 \sim J\curvb{ r, \lambda^2 } \]

	% section aufgabe_2 (end)

	\section*{Aufgabe 3} % (fold)
	\label{sec:aufgabe_3}
	
		\textbf{(a):} Sei $A\in\m{M}_2(\SR)$ mit dem charakteristischen Polynom $\chi$
		\[
			A \define
			\begin{pmatrix}
				1 & 1 \\
				-1 & 3
			\end{pmatrix}
			\quad \implies \quad \chi(\lambda) = (1-\lambda)(3-\lambda) + 1 = (\lambda-2)^2
		\]
		$A$ besitzt damit nur den Eigenwert $2$ mit algebraischen Vielfachheit $2$.
		Für die zugehörigen Eigenvektoren $\transp{(x,y)}\in\SR^2$ muss gelten
		\[ x+y=2x,\quad -x+3y=2y \quad \implies \quad x=y \]
		Der Eigenraum besitzt demzufolge die Dimension $1$, was der geometrischen Vielfachheit des Eigenwertes entspricht.
		Die Jordan-Normalform der Matrix besteht also aus einem Jordan-Block.
		Im zweidimensionalen Fall reicht es nun einen Vektor $v\in\SR^2$ zu wählen, welcher kein Eigenvektor ist.
		\[
			v =
			\begin{pmatrix}
				1 \\ 0
			\end{pmatrix}
			\quad \implies \quad u\define (A-\lambda\idmat)v =
			\begin{pmatrix}
				-1 \\ -1
			\end{pmatrix}
		\]
		Es ist dann $C:=(u,v)\in\m{Gl}_n(\SR)$ und $\inv{C}AC$ die Jordan-Normalform von $A$.
		\[
			J\define\inv{C}AC = 
			\begin{pmatrix}
				0 & -1 \\
				1 & -1
			\end{pmatrix}
			\begin{pmatrix}
				1 & 1 \\
				-1 & 3
			\end{pmatrix}
			\begin{pmatrix}
				-1 & 1 \\
				-1 & 0
			\end{pmatrix}
			=
			\begin{pmatrix}
				2 & 1 \\
				0 & 2
			\end{pmatrix}
			= J(2,2)
		\]
		Für $J$ lässt sich nun leicht durch vollständige Induktion für beliebige $k\in\SN$ zeigen
		\[
			J^k =
			\begin{pmatrix}
				2^k & k2^{k-1} \\
				0 & 2^k
			\end{pmatrix}
		\]
		Es gilt nach aus der Vorlesung bekannten Rechenregeln für $k\in\SN$
		\[
			A^k = CJ^k\inv{C} =
			\begin{pmatrix}
				2^k - k2^{k-1} & k2^{k-1} \\
				-k2^{k-1} & 2^k + k2^{k-1}
			\end{pmatrix}
		\]
		\[
			\implies \quad A^{50} =
			\begin{pmatrix}
				-48\cdot 2^{49} & 50\cdot 2^{49} \\
				-50\cdot 2^{49} & 52\cdot 2^{49}
			\end{pmatrix}
		\]

		\textbf{(b):}
		Sei nun
		\[
			A \define
			\begin{pmatrix}
				7 & -4 \\
				14 & -8
			\end{pmatrix}
			\quad \implies \quad \chi(\lambda) = (7-\lambda)(-8-\lambda) + 56 = \lambda(\lambda+1)
		\]
		Die Eigenwerte von $A$ sind damit $\lambda_1\define -1$ und $\lambda_2\define 0$ mit der algebraischen Vielfachheit 1.
		Da $A$ wieder eine $2\times 2$-Matrix ist, besitzen beide Eigenräume die Dimension $1$.
		Die Jordan-Normalform $J$ von $A$ besteht damit aus zwei Jordan-Blöcken der Dimension $1$.
		$J$ ist also eine Diagonalmatrix.
		Es seien nun $u,v\in\SR^2$ zwei Eigenvektoren zu dem Eigenwert $\lambda_1$ beziehungsweise $\lambda_2$.
		\[
			u\define
			\begin{pmatrix}
				1 \\ 2
			\end{pmatrix}
			,\qquad v\define
			\begin{pmatrix}
				4 \\ 7
			\end{pmatrix}
		\]
		Es sei wieder $C\define (u,v)$.
		Dann ist
		\[
			J\define\inv{C}AC = 
			\begin{pmatrix}
				-7 & 4 \\
				2 & -1
			\end{pmatrix}
			\begin{pmatrix}
				7 & -4 \\
				14 & -8
			\end{pmatrix}
			\begin{pmatrix}
				1 & 4 \\
				2 & 7
			\end{pmatrix}
			=
			\begin{pmatrix}
				-1 & 0 \\
				0 & 0
			\end{pmatrix}
		\]
		Es gilt damit für $k\in\SN$
		\[
			J^k = 
			\begin{pmatrix}
				(-1)^k & 0 \\
				0 & 0
			\end{pmatrix}
			\quad \implies \quad J^{64} = 
			\begin{pmatrix}
				1 & 0 \\
				0 & 0
			\end{pmatrix}
		\]
		\[
			A^{64} = CJ^{64}\inv{C} =
			\begin{pmatrix}
				-7 & 4 \\
				-14 & 8
			\end{pmatrix}
		\]

	% section aufgabe_3 (end)

	\section*{Aufgabe 4} % (fold)
	\label{sec:aufgabe_4}
	
		Sei $A\in\m{M}_n(\SC)$ für ein $n\in\SN$.
		Dann existiert die Jordan-Zerlegung von $A$.
		Es gibt also eine Diagonalmatrix $D\in\m{M}_n(\SC)$ und eine nilpotente Matrix $N\in\m{M}_n(\SC)$ mit $N=\diag(J(r_1),\ldots,J(r_m))$ für gewisse Parameter $r_1,\ldots,r_m\in\SN,m\in\SN$ mit $\sum_{i=1}^{m}r_i = n$, sodass
		\[ A\sim D + N,\qquad DN=ND \]
		Es folgt damit aus bekannten Rechenregeln
		\[ \det A = \det(D+N) = \det D,\qquad \tr A = \tr(D+N) = \tr D \]
		Für das Matrixexponential gilt dann unter Verwendung von $DN=ND$
		\[ \exp A \sim \exp(D+N) = \exp D \exp N \]
		\[ \implies \quad \det(\exp A) = \det(\exp D \exp N) = \det(\exp D)\det(\exp N) \]

		$N$ ist ein obere Dreiecksmatrix mit Nullen auf der Diagonalen.
		Demzufolge ist auch $N^k$ für ein beliebiges $k\in\SN$ eine obere Dreiecksmatrix mit Nullen auf der Diagonalen.
		Man wendet nun die Definition des Matrixexponetials an.
		\[ \exp N = \idmat + \sum_{k=1}^\infty \frac{N^k}{k!} \]
		Es muss also $\exp N$ eine obere Dreiecksmatrix mit Einsen auf der Diagonalen sein.
		Damit gilt unmittelbar
		\[ \det (\exp N) = 1 \]

		Sei nun $\diag(d_1,\ldots,d_n)\define D$.
		\begin{alignat*}{3}
			&\implies \quad D^k = \diag\curvb{d_1^k,\ldots,d_n^k} \quad \text{für alle } k\in\SN \\
			&\implies \quad \exp D = \diag\curvb{e^{d_1},\ldots,e^{d_n}} \\
			&\implies \quad \det(\exp D) = \prod_{i=1}^n e^{d_i} = \exp\curvb{\sum_{i=1}^n d_i} = \exp(\tr D) = \exp(\tr A)
		\end{alignat*}
		Damit folgt die Aussage
		\[ \det(\exp A) = \exp(\tr A) \]
		\qedbox

	% section aufgabe_4 (end)

\end{document}