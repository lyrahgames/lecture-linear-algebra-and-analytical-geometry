% init list of math for definitions and theorems
\newcommand{\listofmathcall}{Verzeichnis der Definitionen und Sätze}
\newlistof{math}{mathlist}{\listofmathcall}
% add parentheses around argument
\newcommand{\parent}[1]{ \ifx&#1&\else (#1) \fi }
\definecolor{mathdefback}{rgb}{0.95,0.95,0.98}
% unnumerated mathematical definition environment definiton
\newenvironment{mathdef*}[2]{
	\medskip
	\begin{tcolorbox}[colback=mathdefback, boxrule=0.5pt, colframe=black, boxsep=0pt, enhanced jigsaw, breakable, arc=3pt]
	\noindent
	{ \fontfamily{ppl}\selectfont \textbf{\textsc{#1:}} } ~ #2 
	\par \hfill\\ 
	\fontfamily{lmr}\selectfont \itshape
}{
	\end{tcolorbox}
	\medskip
}
% definitions for numerated mathematical definition environment
\newcounter{mathdefc}[section]
\newcommand*{\mathdefnum}{\thesection.\arabic{mathdefc}}
\renewcommand{\themathdefc}{\mathdefnum}
\newenvironment{mathdef}[2]{
	\refstepcounter{mathdefc}
	\addcontentsline{mathlist}{figure}{\protect{\numberline{\mathdefnum}#1 ~ #2}}
	\begin{mathdef*}{#1 \mathdefnum}{#2}
}{
	\end{mathdef*}
}
% standard mathdef calls
\newcommand{\definitioncall}{Definition}
\newenvironment{definition*}[1][]{ \begin{mathdef*}{\definitioncall}{\parent{#1}} }{ \end{mathdef*} }
\newenvironment{definition}[1][]{ \begin{mathdef}{\definitioncall}{\parent{#1}} }{ \end{mathdef} }

\definecolor{maththeoremframe}{rgb}{0.7,0.7,0.73}

% unnumerated theorem environment definition
\newenvironment{maththeorem*}[2]{
	\medskip
	\begin{tcolorbox}[boxrule=0pt, leftrule=2.5pt, arc=2pt, colback=white, colframe=maththeoremframe, enhanced jigsaw, breakable, vfill before first, top=0mm, bottom=0mm, left=2mm, right=0mm, boxsep=1mm]
	\noindent
	{ \fontfamily{ppl}\selectfont \textbf{\textsc{#1:}} } ~ #2
	\par \hfill\\ 
	\fontfamily{lmr} \fontshape{it} \selectfont
}{ 
	\end{tcolorbox}
	\medskip
}
% definitions for numerated theorem environment
\newcounter{maththeoremc}[section]
\newcommand*\maththeoremnum{\thesection.\arabic{maththeoremc}}
\renewcommand{\themaththeoremc}{\maththeoremnum}
\newenvironment{maththeorem}[2]{
	\refstepcounter{maththeoremc}
	\addcontentsline{mathlist}{figure}{\protect{\qquad\numberline{\maththeoremnum}#1 ~ #2}}
	\begin{maththeorem*}{#1 \maththeoremnum}{#2}
}{
	\end{maththeorem*}
}
% standard maththeorem calls
\newcommand{\theoremcall}{Theorem}
\newenvironment{theorem*}[1][]{ \begin{maththeorem*}{\theoremcall}{\parent{#1}} }{ \end{maththeorem*} }
\newenvironment{theorem}[1][]{ \begin{maththeorem}{\theoremcall}{\parent{#1}} }{ \end{maththeorem} }
\newcommand{\lemmacall}{Lemma}
\newenvironment{lemma*}[1][]{ \begin{maththeorem*}{\lemmacall}{\parent{#1}} }{ \end{maththeorem*} }
\newenvironment{lemma}[1][]{ \begin{maththeorem}{\lemmacall}{\parent{#1}} }{ \end{maththeorem} }
\newcommand{\propositioncall}{Proposition}
\newenvironment{proposition*}[1][]{ \begin{maththeorem*}{\propositioncall}{\parent{#1}} }{ \end{maththeorem*} }
\newenvironment{proposition}[1][]{ \begin{maththeorem}{\propositioncall}{\parent{#1}} }{ \end{maththeorem} }
\newcommand{\corollarycall}{Korollar}
\newenvironment{corollary*}[1][]{ \begin{maththeorem*}{\corollarycall}{\parent{#1}} }{ \end{maththeorem*} }
\newenvironment{corollary}[1][]{ \begin{maththeorem}{\corollarycall}{\parent{#1}} }{ \end{maththeorem} }
% q.e.d. definition
\newcommand{\qed}{ \par \hfill \fontfamily{lmr} \fontshape{it} \selectfont \mbox{q.e.d.} \\}
\newcommand{\qedbox}{ \hfill $\Box$ }
% proof environment definition for theorems
\newenvironment{mathproof}[2]{
	% \par\hfill\\
	\medskip
	% \noindent
	% \par
	% { \fontfamily{ppl}\selectfont \small \textsc{#1:} } ~ \parent{#2} \smallskip\\
	% \begin{adjustwidth}{1em}{}
	\begin{tcolorbox}[title= { \fontfamily{ppl}\selectfont \small \textsc{#1:} } ~ \parent{#2}, boxrule=0pt, colback=white, colframe=white, coltitle=black, breakable, boxsep=0mm, top=2mm, bottom=0mm, right=0mm, left=0mm, before upper={\parindent1em}]%
	\normalfont
	\small
}{ 
	\end{tcolorbox}
	% \end{adjustwidth} 
	% \qedbox
	\medskip
}
% standard mathproof calls
\newcommand{\proofcall}{Beweis}
\newenvironment{proof}[1][]{ \begin{mathproof}{\textbf{\proofcall}}{#1} }{ \qedbox \end{mathproof} }
\newcommand{\proofideacall}{Beweisidee}
\newenvironment{proofidea}[1][]{ \begin{mathproof}{\proofideacall}{#1} }{ \end{mathproof} }
\newcommand{\examplecall}{Beispiel}
\newenvironment{example}[1][]{ \begin{mathproof}{\examplecall}{#1} }{ \end{mathproof} }
