% documentclass: article used for scientific journals, short reports, program documentation, etc
% options: fontsize 11, generate document for double sided printing, a4-paper
\documentclass[10pt, twoside, a4paper, fleqn]{article}

% package for changing page layout
\usepackage{geometry}
\geometry{a4paper, lmargin=40mm, rmargin=45mm, tmargin=40mm, bmargin=45mm}
% set indentation
\setlength{\parindent}{1em}
% set factor for line spacing
% \linespread{1.0}\selectfont
% set (dynamic) additional line spacing
% \setlength{\parskip}{1ex plus 0.5ex minus 0.3ex}

% rigorous formatting (not too much hyphens)
% \fussy
% \sloppy

% package for changing page layout (used to indent whole paragraphs with adjustwidth)
\usepackage{changepage}

% input encoding for special characters (e.g. ä,ü,ö,ß), only for non english text
% options: utf8 as encoding standard, latin1
\usepackage[utf8]{inputenc}
% package for font encoding
\usepackage[T1]{fontenc}
% package for changing used language (especially for more than one language)
% options: ngerman (new spelling) or default: english
\usepackage[ngerman]{babel}
% package for times font
% \usepackage{times}
% package for latin modern fonts
\usepackage{lmodern}

% package for math symbols, functions and environments from ams(american mathematical society)
\usepackage{amsmath}
\usepackage{mathtools}
% package for extended symbols from ams
\usepackage{amssymb}
% package for math black board symbols (e.g. R,Q,Z,...)
\usepackage{bbm}
% package used for calligraphic math symbols
\usepackage{mathrsfs}
% package for extended symbols from stmaryrd(st mary road)
\usepackage{stmaryrd}
% package for more math blackboard symbols
\usepackage{dsfont}

% pack­age im­ple­ments scal­ing of the math ex­ten­sion font cmex; used for scaling math signs
\usepackage{exscale}

% package for including extern graphics plus scaling and rotating
\usepackage{graphicx}
%package for positioning figures
\usepackage{float}
% package for changing color of font and paper
% options: using names of default colors (e.g red, black)
% \usepackage[usenames]{color}
\usepackage[dvipsnames]{xcolor}
\definecolor{shadecolor}{gray}{0.9}
% package for customising captions
\usepackage[footnotesize, hang]{caption}
% package for customising enumerations (e.g. axioms)
\usepackage{enumitem}
% calc package reimplements \setcounter, \addtocounter, \setlength and \addtolength: commands now accept an infix notation expression
\usepackage{calc}
% package for creating framed, shaded, or differently highlighted regions that can break across pages; environments: framed, oframed, shaded, shaded*, snugshade, snugshade*, leftbar, titled-frame
\usepackage{framed}
% package for creating custom "list of"
% options: titles: do not intefere with standard headings for "list of"
\usepackage[titles]{tocloft}
% change enumeration style of equations
% \renewcommand\theequation{\thesection.\arabic{equation}}

% init list of math for definitions and theorems
\newcommand{\listofmathcall}{Verzeichnis der Definitionen und Sätze}
\newlistof{math}{mathlist}{\listofmathcall}
% add parentheses around argument
\newcommand{\parent}[1]{ \ifx&#1&\else (#1) \fi }
% unnumerated mathematical definition environment definiton
\newenvironment{mathdef*}[2]{
	\medskip
	\begin{tcolorbox}[colback=white, boxrule=0.5pt, colframe=black, breakable]
	\noindent
	{ \fontfamily{ppl}\selectfont \textbf{\textsc{#1:}} } ~ #2 
	\par \hfill\\ 
	\fontfamily{lmr}\selectfont \itshape
}{
	\end{tcolorbox}
	\medskip
}
% definitions for numerated mathematical definition environment
\newcounter{mathdefc}[section]
\newcommand*{\mathdefnum}{\thesection.\arabic{mathdefc}}
\renewcommand{\themathdefc}{\mathdefnum}
\newenvironment{mathdef}[2]{
	\refstepcounter{mathdefc}
	\addcontentsline{mathlist}{figure}{\protect{\numberline{\mathdefnum}#1 ~ #2}}
	\begin{mathdef*}{#1 \mathdefnum}{#2}
}{
	\end{mathdef*}
}
% standard mathdef calls
\newcommand{\definitioncall}{Definition}
\newenvironment{definition*}[1][]{ \begin{mathdef*}{\definitioncall}{\parent{#1}} }{ \end{mathdef*} }
\newenvironment{definition}[1][]{ \begin{mathdef}{\definitioncall}{\parent{#1}} }{ \end{mathdef} }
% unnumerated theorem environment definition
\newenvironment{maththeorem*}[2]{
	\begin{leftbar}%[boxrule=0pt, leftrule=3pt, arc=0pt, colback=white, colframe=black, enhanced jigsaw]
	\noindent
	{ \fontfamily{ppl}\selectfont \textbf{\textsc{#1:}} } ~ #2
	\par \hfill\\ 
	\fontfamily{lmr} \fontshape{it} \selectfont
}{ 
	\end{leftbar}
}
% definitions for numerated theorem environment
\newcounter{maththeoremc}[section]
\newcommand*\maththeoremnum{\thesection.\arabic{maththeoremc}}
\renewcommand{\themaththeoremc}{\maththeoremnum}
\newenvironment{maththeorem}[2]{
	\refstepcounter{maththeoremc}
	\addcontentsline{mathlist}{figure}{\protect{\qquad\numberline{\maththeoremnum}#1 ~ #2}}
	\begin{maththeorem*}{#1 \maththeoremnum}{#2}
}{
	\end{maththeorem*}
}
% standard maththeorem calls
\newcommand{\theoremcall}{Theorem}
\newenvironment{theorem*}[1][]{ \begin{maththeorem*}{\theoremcall}{\parent{#1}} }{ \end{maththeorem*} }
\newenvironment{theorem}[1][]{ \begin{maththeorem}{\theoremcall}{\parent{#1}} }{ \end{maththeorem} }
\newcommand{\lemmacall}{Lemma}
\newenvironment{lemma*}[1][]{ \begin{maththeorem*}{\lemmacall}{\parent{#1}} }{ \end{maththeorem*} }
\newenvironment{lemma}[1][]{ \begin{maththeorem}{\lemmacall}{\parent{#1}} }{ \end{maththeorem} }
\newcommand{\propositioncall}{Proposition}
\newenvironment{proposition*}[1][]{ \begin{maththeorem*}{\propositioncall}{\parent{#1}} }{ \end{maththeorem*} }
\newenvironment{proposition}[1][]{ \begin{maththeorem}{\propositioncall}{\parent{#1}} }{ \end{maththeorem} }
\newcommand{\corollarycall}{Korollar}
\newenvironment{corollary*}[1][]{ \begin{maththeorem*}{\corollarycall}{\parent{#1}} }{ \end{maththeorem*} }
\newenvironment{corollary}[1][]{ \begin{maththeorem}{\corollarycall}{\parent{#1}} }{ \end{maththeorem} }
% q.e.d. definition
\newcommand{\qed}{ \par \hfill \fontfamily{lmr} \fontshape{it} \selectfont \mbox{q.e.d.} \\}
\newcommand{\qedbox}{ \par \hfill $\Box$ \\ }
% proof environment definition for theorems
\newenvironment{mathproof}[1]{
	\par\hfill\\
	\noindent
	{ \fontfamily{lmr}\selectfont \small \textsc{#1:}}
	\normalfont
	\small
	\begin{adjustwidth}{1em}{}
	\medskip
}{ 
	\end{adjustwidth} 
	% \qed
	\qedbox
}
% standard mathproof calls
\newcommand{\proofcall}{Beweis}
\newenvironment{proof}{ \begin{mathproof}{\proofcall} }{ \end{mathproof} }
\newcommand{\proofideacall}{Beweisidee}
\newenvironment{proofidea}{ \begin{mathproof}{\proofideacall} }{ \end{mathproof} }

% new displaymath command, so that equations will not be stretched
\newcommand{\D}[1]{\mbox{$ #1 $}}
% make unnumerated equation
\newcommand{\E}[1]{\[ #1 \]}
% command for curly brackets
\newcommand{\curlb}[1]{\left\{ #1 \right\}}
% command for box brackets
\newcommand{\boxb}[1]{\left[ #1 \right]}
% command for parentheses/curved brackets
\newcommand{\curvb}[1]{\left( #1 \right)}
% command for angle brackets
\newcommand{\angleb}[1]{\left\langle #1 \right\rangle}
% command for floor brackets
\newcommand{\floorb}[1]{\left\lfloor #1 \right\rfloor}
% command for ceil brackets
\newcommand{\ceilb}[1]{\left\lceil #1 \right\rceil}
% command for creating sets
\newcommand{\set}[2]{ \left\{ #1 \enspace \middle\vert \enspace #2 \right\} }
% command for absolute value
\newcommand{\abs}[1]{\left\vert #1 \right\vert}
\newcommand{\norm}[1]{\left\| #1 \right\|}
% commands for writing limits
\newcommand{\limit}[3]{\, \longrightarrow \, #1, \ #2 \longrightarrow #3}
\newcommand{\Limit}[2]{\lim_{#1 \rightarrow #2}}
% command for differential
\newcommand{\diff}{\mathrm{d}}
\newcommand{\Diff}{\mathrm{D}}
% command for derivative
\newcommand{\Deriv}[3][]{\Diff_{#2}^{#1}#3}
\newcommand{\deriv}[3][]{\dfrac{\diff^{#1}#2(#3)}{\diff #3^{#1}}}
% command for integral
\newcommand{\integral}[4]{\int_{#1}^{#2} #3\ \diff #4}
\newcommand{\Integral}[4]{\int\limits_{#1}^{#2} #3\ \diff #4}
\newcommand{\iintegral}[2]{\int #1\ \diff #2} % indefinite integral
% mathematical definitions (standard sets)
\newcommand{\SR}{\mathds{R}} % real numbers
\newcommand{\SC}{\mathds{C}} % complex numbers
\newcommand{\SN}{\mathds{N}} % natural numbers
\newcommand{\SZ}{\mathds{Z}} % integral numbers
\newcommand{\SQ}{\mathds{Q}} % rational numbers
\newcommand{\SP}{\mathcal{P}} % power set
\newcommand{\SFP}{\mathds{P}} % polynom functions
\newcommand{\SFC}{\mathrm{C}} % complex valued functions (continous or differentiable)
\newcommand{\SFL}{\mathcal{L}} % space of integrable functions
\newcommand{\SFLL}{\mathrm{L}} % space of integrable function classes
\newcommand{\SH}{\mathcal{H}} % hilbert space
% mathematical standard functions
\DeclareMathOperator{\real}{Re} % real part
\DeclareMathOperator{\imag}{Im} % imaginary part
\DeclareMathOperator{\diag}{diag}
\DeclareMathOperator{\id}{Id}
\newcommand{\FF}{\mathcal{F}} % fourier transform
\newcommand{\FE}{\mathbb{E}} % expectation
\DeclareMathOperator{\var}{var} % variance
\newcommand{\FN}{\mathcal{N}} % normal distribution

\newcommand{\m}[1]{\mathrm{#1}}

% command for physical units
\newcommand{\unit}[1]{\, \mathrm{#1}}


% package for init listings(non-formatted  text) e.g. different source codes
\usepackage{listings}


% definitions for listing colors
\definecolor{codeDarkGray}{gray}{0.2}
\definecolor{codeGray}{gray}{0.4}
\definecolor{codeLightGray}{rgb}{0.94,0.94,0.91}
\definecolor{codeBorder}{rgb}{0.34,0.24,0.21}
% predefinitions for listings
\newcommand{\listingcall}{Listing}
\newlength{\listingframemargin}
\setlength{\listingframemargin}{1em}
\newlength{\listingmargin}
\setlength{\listingmargin}{0.08\textwidth}
% \newlength{\listingwidth}
% \setlength{\listingwidth}{ ( \textwidth - \listingmargin * \real{2} + \listingframemargin * \real{2} ) }
% definitions for list of listings
\newcommand{\listoflistingscall}{\listingcall -Verzeichnis}
\newlistof{listings}{listinglist}{\listoflistingscall}
% style definition for standard code listings
\lstdefinestyle{std}{
	belowcaptionskip=0.5\baselineskip,
	breaklines=true,
	frameround=tttt,
	% frame=false,
	xleftmargin=0em,
	xrightmargin=0em,
	showstringspaces=false,
	showtabs=false,
	% tab=\smash{\rule[-.2\baselineskip]{.4pt}{\baselineskip}\kern.5em},
	basicstyle= \fontfamily{pcr}\selectfont\footnotesize\bfseries,
	keywordstyle= \bfseries\color{MidnightBlue}, %\color{codeDarkGray},
	commentstyle= \itshape\color{codeGray},
	identifierstyle=\color{codeDarkGray},
	stringstyle=\color{BurntOrange}, %\color{codeDarkGray},
	numberstyle=\tiny\ttfamily,
	% numbers=left,
	numbersep = 1em,
	% stepnumber = 1,
	% captionpos=t,
	tabsize=4,
	% backgroundcolor=\color{codebLightGray},
	rulecolor=\color{codeBorder},
	framexleftmargin=\listingframemargin,
	framexrightmargin=\listingframemargin
}
% definition for unnumerated listing
\newcommand{\inputlistingn}[3][]{
	\begin{center}
		\begin{adjustwidth}{\listingmargin}{\listingmargin}
			\centerline{ {\fontfamily{lmr}\selectfont \footnotesize \listingcall:}\quad {\footnotesize #2} }
			\lstinputlisting[style=std, #1]{#3}
		\end{adjustwidth}
	\end{center}
}
% definition for numerated listing
\newcounter{listingc}[section]
\newcommand*\listingnum{\thesection.\arabic{listingc}}
\renewcommand{\thelistingc}{\listingnum}
\newcommand{\inputlisting}[3][]{
	\refstepcounter{listingc}
	\addcontentsline{listinglist}{figure}{\protect{\numberline{\listingnum:} #2 } }
	% \inputlistingn[#1]{#2}{#3}
	\begin{center}
		\begin{adjustwidth}{\listingmargin}{\listingmargin}
			\centerline{ {\fontfamily{lmr}\selectfont \footnotesize \listingcall~\listingnum:}\quad {\footnotesize #2} }
			\lstinputlisting[style=std, #1]{#3}
		\end{adjustwidth}
	\end{center}
}


% package for including csv-tables from file
% \usepackage{csvsimple}
% package for creating, loading and manipulating databases
\usepackage{datatool}

% package for converting eps-files to pdf-files and then include them
\usepackage{epstopdf}
% use another program (ps2pdf) for converting
% !!! important: set shell_escape=1 in /etc/texmf/texmf.cnf (Linux/Ubuntu 12.04) for allowing to use other programs
% !!!			or use the command line with -shell-escape
% \epstopdfsetup{outdir=./}
% \epstopdfDeclareGraphicsRule{.eps}{pdf}{.pdf}{
% ps2pdf -dEPSCrop #1 \OutputFile
% }


% package for reference to last page (output number of last page)
\usepackage{lastpage}
% package for using header and footer
% options: automate terms of right and left marks
% \usepackage[automark]{scrpage2}
% \setlength{\headheight}{4\baselineskip}
% set style for footer and header
% \pagestyle{scrheadings}
% \pagestyle{headings}
% clear pagestyle for redefining
% \clearscrheadfoot
% set header and footer: use <xx>head/foot[]{Text} (i...inner, o...outer, c...center, o...odd, e...even, l...left, r...right)

% use that for mark to last page: \pageref{LastPage}
% set header separation line
% \setheadsepline[\textwidth]{0.5pt}
% set foot separation line
% \setfootsepline[\textwidth]{0.5pt}



\usepackage{tcolorbox}
% \usepackage{tikz}
% \tcbuselibrary{listings}
\tcbuselibrary{many}
\tcbset{fonttitle=\footnotesize}

\usepackage{array}

\allowdisplaybreaks

% \usepackage{epic, eepic}
\usepackage{epic}

\usepackage{natbib}
\bibliographystyle{plain}
\usepackage{url}

\usepackage{indentfirst}

\title{Lineare Algebra und Analytische Geometrie II}
\author{Markus Pawellek}

\ihead[]{\leftmark}
\ohead[]{\rightmark}
\cfoot[]{\newline\newline\newline\pagemark}

\begin{document}

	\maketitle{\thispagestyle{empty}}
	\tableofcontents
	\newpage
	\pagestyle{empty}
	% \listoffigures
	% \listoftables
	\listofmath
	% \lstlistoflistings
	% \listoflistings
	\newpage
	\pagestyle{scrheadings}
	\pagenumbering{arabic}
	
	\section{Wiederholung} % (fold)
	\label{sec:wiederholung}
	
		\begin{definition}[Gruppe]
			Sei $G$ eine Menge und sei \D{\ \star :G\times G\longrightarrow G \ } eine Abbildung mit den folgenden Eigenschaften.
			\begin{enumerate}[label = \normalfont{(G\arabic*)}]
				\item Es gibt ein $e\in G$, sodass für alle $g\in G$ gilt
					\[
						e\star g = g\star e = g \hfill\text{(neutrales Element)}
					\] 
				\item Für alle $g,f,h \in G$ gilt
					\[
						\curvb{g\star f}\star h = g\star\curvb{f\star h} \hfill\text{(Assoziativität)}
					\]
				\item Für alle $g\in G$ gibt es ein $g^{-1}\in G$ mit
					\[
						g\star g^{-1} = g^{-1}\star g = e \hfill\text{(inverses Element)}
					\]
			\end{enumerate}
			Dann nennt man $\curvb{G,\star}$ oder auch $G$ eine Gruppe.
			Gilt darüber hinaus für alle $g,f\in G$ auch
			\[
				f\star g = g\star f \hfill\text{(Kommutativität)}
			\]
			so nennt man $(G,\star)$ auch abelsche Gruppe.
		\end{definition}

		\begin{definition}[Vektorraum]
			Sei $K$ ein Körper. 
			Seien weiterhin $\curvb{V,+}$ eine abelsche Gruppe und \D{\ \star :K\times V\longrightarrow V \ } eine Abbildung mit den folgenden Eigenschaften für alle $u,v\in V$ und alle $a,b\in K$.
			\begin{enumerate}[label = \normalfont{(V\arabic*)}]
				\item $ 1\star v = v \hfill\text{(neutrales Element)} $
				\item $ (ab)\star v = a\star(b\star v) \hfill\text{(Assoziativität)} $
				\item $ (a+b)\star v = a\star v + b\star v \hfill\text{(1.Distributivgesetz)} $
				\item $ a\star (v+u) = a\star v + a\star u \hfill\text{(2.Distributivgesetz)} $
			\end{enumerate}
			Dann nennt man $\curvb{V,+,\star}$ oder auch $V$ einen $K$-Vektorraum.
		\end{definition}

		\begin{definition}[Vektorraumendomorphismus]
			Sei $(V,+,\star)$ ein $K$-Vektorraum.
			Eine Abbildung $\varphi:V\longrightarrow V$ heißt Endomorphismus, wenn $\varphi$ linear ist, das heißt für alle $a\in K$ und $x,y\in V$ gilt
			\[
				\varphi((a\star x)+y) = a\star\varphi(x) + \varphi(y)
			\]
		\end{definition}

		\begin{lemma}[$\m{GI}_n(\SR)$ ist Gruppe]
			Die Menge der reellen invertierbaren Matrizen
			\[
				\m{GI}_n(\SR) := \set{A\in \m{M}_n(\SR)}{\det A \neq 0}
			\]
			ist bezüglich der Matrixmultiplikation eine Gruppe.
		\end{lemma}
		\begin{proof}[Lemma\ \ref{corollary:charac}]
			Es gilt im allgemeinen für alle $A,B\in GI_n(\SR)$
			\[
				\det(AB) = \det A \cdot\det B \neq 0 \ \Rightarrow \ AB\in \m{GI}_n(\SR)
			\]
			Weiterhin folgt
			\[
				\det A \neq 0 \quad \implies \quad \text{es gibt } A^{-1}\in \m{GI}_n(\SR) \text{ mit } A^{-1}A=I
			\]
			Die Identitätsmatrix beschreibt das neutrale Element.
			Die Matrixmultiplikation ist assoziativ.
			$\m{GI}_n$ ist damit gerade eine Gruppe.
		\end{proof}

	% section wiederholung (end)
	
	\section{Skalarprodukt und orthogonale Matrizen} % (fold)
	\label{sub:skalarprodukt_und_orthogonale_matrizen}
	
		Im Folgenden soll es das Ziel sein, eine geeignete Charakterisierung von 2- und 3-dimensionalen Drehungen im euklidischen Raum zu finden.
		Dafür soll als Grundgerüst die euklidische Geometrie verwendet werden.
		Durch Einführung weiterer Konzepte sollen die erhaltenen Ergebnisse dann für ein $n\in\SN$ auf dem $n$-dimensionalen euklidischen Raum verallgemeinert werden.

		\begin{definition}[orthogonale Matrix, $\m{O}_n$]
			Sei $n\in\SN$. 
			Eine reelle invertierbare $n\times n$-Matrix $A\in \m{M}_n(\SR)$ heißt orthogonal genau dann, wenn
			\[
				A\transp = A^{-1}
			\]
			Die Menge der orthogonalen Matrizen sei dann
			\[
				\m{O}_n := \set{A\in \m{M}_n(\SR)}{\det A \neq 0,\ A\transp A=\idmat}
			\]
		\end{definition}

		\begin{lemma}[$\m{O}_n$ ist Untergruppe]
			Sei $n\in\SN$.
			Die Menge der orthogonalen Matrizen $\m{O}_n$ bildet eine Untergruppe von $\m{GI}_n(\SR)$ bezüglich der Matrixmultiplikation.

		\end{lemma}
		\begin{proof}
			Es seien $A,B\in \m{O}_n$. Dann gilt
			\[
				(AB)\transp AB = B\transp A\transp AB = B\transp \idmat B = B\transp B = \idmat \quad \implies \quad AB\in \m{O}_n
			\]
			Weiterhin gilt
			\[
				A^{-1}\curvb{A^{-1}}\transp  = A\transp \curvb{A\transp}\transp = A\transp A = \idmat \quad \implies \quad A^{-1}\in \m{O}_n
			\]
			Damit ist $\m{O}_n$ eine Untergruppe.
		\end{proof}

		\begin{proposition}[Determinante orthogonaler Matrizen]
			Seien $n\in\SN$ und $A\in \m{O}_n$. Dann gilt
			\[
				\abs{\det A} = 1
			\]
		\end{proposition}
		\begin{proof}
			Es gilt $A\transp A=\idmat$. Daraus folgt
			\[
				1 = \det\idmat = \det (A\transp A) = \det A\transp \det A = \curvb{\det A}^2 \quad \implies \quad \abs{\det A} = 1
			\]
		\end{proof}

		Ist die Determinante einer orthogonalen Matrix gerade $-1$, so spiegelt sie den Raum.
		Dies kann man sich am Beispiel von
		\[
			A:=\diag(1,-1)\in\SR^2
		\]
		klar machen.
		Hier wird der Raum nicht gedreht, sondern nur an der Abszisse gespiegelt.
		Damit wird also eine spezielle Teilmenge aus $\m{O}_n$ benötigt, um Drehungen zu charakterisieren.

		\begin{definition}[spezielle Teilmenge $\m{SO}_n$]
			Sei $n\in\SN$.
			Dann sei die Menge der $n$-dimensionalen orthogonalen Matrizen mit Determinante 1
			\[
				\m{SO}_n := \set{A\in \m{O}_n}{\det A = 1}
			\]
		\end{definition}

		\begin{lemma}[$\m{SO}_n$ ist Untergruppe]
			Sei $n\in\SN$.
			Dann bildet $\m{SO}_n$ eine spezielle Untergruppe von $\m{O}_n$.
		\end{lemma}
		\begin{proof}
			\[ A=B \]
		\end{proof}

		Die Drehungen im 2- und 3-dimensionalen sind nun gerade Elemente aus $\m{SO}_2$ beziehungsweise $\m{SO}_3$.
		Dies ist der Grund, weshalb $\m{SO}_n$ auch $n$-dimensionale Drehgruppe genannt wird.

		% \begin{corollary}[mehrere Drehungen sind wieder eine Drehung]
		% 	Die Zusammensetzung von zwei Drehungen um den Nullpunkt im $\SR^3$ ist wieder eine Drehung.
		% \end{corollary}


		\begin{definition}[Standardskalarprodukt und -norm auf $\SR^n$]
			Sei $n\in\SN$. 
			Die folgende Abbildung wird Standardskalarprodukt auf $\SR^n$ genannt.
			\[
				\angleb{\cdot, \cdot}:\SR^n\times\SR^n\longrightarrow\SR,\qquad \angleb{x,y} = \sum_{i=1}^n x_iy_i = x\transp y
			\]
			Die zugehörige induzierte Standardnorm auf $\SR^n$ sei dann
			\[
				\norm{\cdot}:\SR^n\longrightarrow[0,\infty),\qquad \norm{x} = \sqrt{\angleb{x,x}} = \curvb{\sum_{i=1}^n x_i^2}^{\frac{1}{2}}
			\]
			Ein $x\in\SR^n$ mit $\norm{x}=1$ heißt Einheitsvektor oder auch normierter Vektor.
		\end{definition}

		$\norm{x-y}$ stellt gerade den Abstand zwischen $x$ und $y$ dar.
		$\angleb{x,y}$ enthält zusätzlich noch die Information über den Winkel zwischen den Vektoren $x,y$.

		\begin{proposition}[Zusammenhang Skalarprodukt und Winkel]
		\label{prop:skalarproduktregel}
			Seien $x,y\in\SR^n$ für $n\in\SN$ und $\vartheta\in[0,2\pi)$ der Winkel zwischen $x,y$. Dann gilt
			\[
				\angleb{x,y} = \norm{x}\norm{y}\cos \vartheta
			\]
		\end{proposition}
		\begin{proof}
			Aus der Bilinearität und dem Kosinussatz folgt die Aussage direkt.
		\end{proof}

		\begin{corollary}[Charakterisierung Orthogonalität]
		\label{corollary:charac}
			Seien $x,y\in\SR^n$.
			Dann ist
			\[
				x \perp y \equivalent \angleb{x,y} = 0
			\]
		\end{corollary}
		\begin{proof}[\corollarycall  \ref{corollary:charac}]
			Ist $\varphi\in [0,2\pi)$ der Winkel zwischen $x$ und $y$, dann muss $\varphi = \pi/2$ gelten, wenn sie senkrecht aufeinander stehen.
			Nach Proposition \ref{prop:skalarproduktregel} folgt nun direkt
			\[
				\angleb{x,y} = \norm{x}\norm{y}\cos\frac{\pi}{2} = 0
			\]
		\end{proof}

		\begin{theorem}[Charakterisierung orthogonaler Matrizen]
		\label{theorem:charakterisierung orthogonale matrizen}
			Sei $A\in M_n(\SR)$ für $n\in\SN$. 
			Dann sind folgende Aussagen äquivalent.
			\begin{enumerate}[label = \normalfont(\roman*)]
				\item $A$ ist orthogonal.
				\item Für alle $x,y\in\SR^n$ gilt
					\[
						\angleb{Ax,Ay} = \angleb{x,y}
					\]
				\item Die Spaltenvektoren von $A$ sind paarweise orthogonale Einheitsvektoren.
			\end{enumerate}
		\end{theorem}
		\begin{proof}
			(i)$\equivalent$(ii):\\
			Sei $A$ orthogonal. 
			In diesem Falle gilt
			\[
				AA^T=I \ \Leftrightarrow \ \angleb{x,y} = x^Ty = x^TA^TAy = (Ax)^TAy = \angleb{Ax,Ay} \text{ für alle } x,y\in\SR^n
			\]

			(ii)$\Rightarrow$(iii):\\
			Es sei $A=(a_1,\ldots,a_n)$ mit $a_i\in\SR^n$ für $1\leq i\leq n$. Für alle $x,y\in\SR^n$ gilt $\angleb{Ax,Ay} = \angleb{x,y}$.
			Weiterhin seien die $e_i$ für $1\leq i\leq n$ die kartesischen Einheitsvektoren.
			Dann folgt für $i,j\leq n$
			\[
				a_i = Ae_i \ \Rightarrow \ \angleb{a_i,a_j} = \angleb{Ae_i, Ae_j} \stackrel{(Vor.)}{=} \angleb{e_i,e_j} = \delta_{ij}
			\]


			(iii)$\Rightarrow$(i):\\
			Es sei $A=(a_1,\ldots,a_n)$ mit $a_i\in\SR^n$ für $1\leq i\leq n$, sodass $\angleb{a_i,a_j}=\delta_{ij}$.
			Es gelte $A^TA = (c_{ij})$.
			Dann folgt durch Rechnung
			\[
				c_{ij} = \angleb{a_i,a_j} = \delta_{ij} \ \Rightarrow \ A^TA=I
			\]
		\end{proof}

		\begin{definition}[Orthonormalbasis]
			Seien $V$ ein $K$-Vektorraum und $B=\curlb{b_1,\ldots,b_n}$ für $n\in\SN$ eine Basis.
			Dann ist $B$ eine Orthonormalbasis genau dann, wenn für alle $i,j\in\SN$ mit $i,j\leq n$ Folgendes gilt.	
			\[
				\angleb{b_i,b_j}=\delta_{ij} =
				\begin{cases}
					1 &:i=j \\
					0 &:i\neq j
				\end{cases}
			\]
		\end{definition}

		Im Falle $i=j$ bedeutet dies gerade $\angleb{b_i,b_i} = \norm{b_i}^2 = 1$.
		Die Basisvektoren einer Orthonormalbasis stehen also senkrecht aufeinander und haben jeweils die Länge 1.

		\begin{definition}[orthogonaler Endomorphismus]
			Sei $n\in\SN$.
			Sei $\varphi:\SR^n\longrightarrow\SR^n$ ein Endomorphismus.
			$\varphi$ heißt orthogonal, wenn für alle $x,y\in\SR^n$ gilt
			\[
				\angleb{x,y} = \angleb{\varphi(x), \varphi(y)}
			\]
		\end{definition}

		\begin{definition}[Isometrie]
			Eine euklidische Bewegung oder auch Isometrie auf $\SR^n$ ist eine abstandstreue Abbildung $\beta:\SR^n\longrightarrow\SR^n$,
			das heißt für alle $x,y\in\SR^n$ gilt
			\[
				\norm{\beta(x) - \beta(y)} = \norm{x-y}
			\]
			Erfüllt $\beta$ darüber hinaus auch die Eigenschaft
			\[
				\beta(0) = 0
			\]
			so nennt man $\beta$ hier eine homogene Isometrie.
			% Die Menge der Isometrien bildet die Bewegungsgruppe oder auch Isometriegruppe.
		\end{definition}

		\begin{lemma}[Isometrien bilden Gruppe]
			Sei $n\in\SN$. Die Menge der Isometrien auf $\SR^n$ bildet bezüglich der Komposition eine Gruppe.
			Diese Gruppe wird auch Bewegungs- oder Isometriegruppe genannt.
		\end{lemma}
		\begin{proof}
			Seien $\alpha,\beta$ Isometrien auf $\SR^n$.
			\[
				\norm{\alpha(\beta(x))-\alpha(\beta(y))} \stackrel{\text{($\alpha$ Iso.)}}{=} \norm{\beta(x) - \beta(y)} \stackrel{\text{($\beta$ Iso.)}}{=} \norm{x-y}
			\]
			\[
				\Rightarrow \ \alpha\circ\beta \text{ ist Isometrie auf } \SR^n
			\]
			Das neutrale Element wird durch die Identität, welche auch eine Isometrie ist, beschrieben.
			

			Weiterhin ist die Komposition von Funktionen im Allgemeinen assoziativ.
			Gibt es nun zu $\beta$ eine inverse Funktion $\beta^{-1}$ folgert man
			\[
				\norm{\beta(x)-\beta(y)} = \norm{x-y} = \norm{\beta^{-1}(\beta(x)) - \beta^{-1}(\beta(y))}
			\]
			Mit $\tilde{x}:=\beta(x)$ und $\tilde{y}:=\beta(y)$ ist ersichtlich, dass $\beta^{-1}$ eine Isometrie sein muss.
			\[
				\norm{\tilde{x}-\tilde{y}} = \norm{\beta^{-1}(\tilde{x})-\beta^{-1}(\tilde{y})} \ \Leftrightarrow \ \beta^{-1} \text{ ist Isometrie}
			\]
			Zu zeigen ist nun noch, dass $\beta^{-1}$ existiert.
			Seien nun $x,y\in\SR^n$ mit $\beta(x)=\beta(y)$.
			Dann gilt per Definition
			\[
				0 = \norm{\beta(x)-\beta(y)} = \norm{x-y} \ \Leftrightarrow \ x-y = 0 \ \Leftrightarrow \ x = y 
			\]
			\[
				\Leftrightarrow \ \beta \text{ ist bijektiv} \ \Leftrightarrow \ \beta^{-1} \text{ existiert}
			\]
		\end{proof}

		\begin{theorem}[Charakterisierung homogene Isometrie]
			Sei $\beta:\SR^n\longrightarrow \SR^n$ für $n\in\SN$ eine Abbildung.
			Dann sind folgende Aussagen äquivalent.
			\begin{enumerate}[label = \normalfont(\roman*)]
				\item $\beta$ ist eine Isometrie mit $\beta(0)=0$.
				\item Für alle $x,y\in\SR^n$ gilt 
					\[
						\angleb{\beta(x), \beta(y)} = \angleb{x,y}
					\]
				\item $\beta$ ist gegeben durch Linksmultiplikation mit einer orthogonalen Matrix.
			\end{enumerate}

		\end{theorem}
		\begin{proof}
			(i)$\Rightarrow$(ii):\\
			Es seien $\beta$ eine homogene Isometrie und $x,y\in\SR^n$ beliebig.
			\[
				\norm{\beta(x)-\beta(y)}^2 = \angleb{\beta(x)-\beta(y), \beta(x)-\beta(y)} = \norm{x-y}^2 = \angleb{x-y,x-y}
			\]
			Setze nun $y=0$.
			\[
				\Rightarrow \norm{\beta(x)}^2 = \angleb{\beta(x),\beta(x)} = \angleb{x,x} = \norm{x}^2
			\]
			Aus der Bilinearität von $\angleb{\cdot,\cdot}$ folgt wieder für beliebige $y$
			\[
				\norm{x-y}^2 = \norm{x}^2 + \norm{y}^2 - 2\angleb{x,y}
			\]
			\[
				\Rightarrow \ \norm{\beta(x)}^2 + \norm{\beta(y)}^2 - 2\angleb{\beta(x),\beta(y)} = \norm{x}^2 + \norm{y}^2 - 2\angleb{x,y}
			\]
			\[
				\Rightarrow \ \angleb{\beta(x), \beta(y)} = \angleb{x,y}
			\]
			(ii)$\Rightarrow$(iii):\\
			Die einzige Abbildung, welche nun das Skalarprodukt erhält und die Einheitsvektoren auf sich selbst abbildet, ist die Identität.
			\[
				\angleb{x,e_i} = \angleb{\beta(x), \beta(e_i)} = \angleb{\beta(x), e_i}
			\]
			\[
				\Rightarrow \ x_i^\prime = x_i \text{ für alle } i\in\curlb{1,\ldots,n} \ \Leftrightarrow \ x^\prime = x \ \Leftrightarrow \ \beta = \id
			\]
			Für ein allgemeines $\beta$ folgt nun, dass $\set{\beta(e_i)}{1\leq i\leq n}$ ein Orthonormalsystem ist, wegen
			\[
				\delta_{ij} = \angleb{e_i,e_j} = \angleb{\beta(e_i),\beta(e_j)}
			\]
			Da $\beta$ ein Endomorphismus ist, gibt es eine Matrix $A\in M_n(\SR)$, sodass $\beta(x) = Ax$.
			Diese Matrix ist eindeutig bestimmt.
			\[
				A:=\curvb{\beta(e_1),\ldots,\beta(e_n)}
			\]
			Dann gilt nach Theorem \ref{theorem:charakterisierung orthogonale matrizen} über die Charakterisierung orthogonaler Matrizen, dass $A$ orthogonal ist.\smallskip \\
			(iii)$\Rightarrow$(i):\\
			Folgt ebenfalls aus dem Theorem \ref{theorem:charakterisierung orthogonale matrizen}.
		\end{proof}

		\begin{definition}[Translation]
			Sei $n\in\SN$. 
			Für $b\in\SR^n$ sei die Translation $t_b$ die Abbildung
			\[
				t_b:\SR^n\longrightarrow\SR^n,\qquad t_b(x) = x+b
			\]
		\end{definition}

		\begin{lemma}[Translation ist Isometrie]
			Jede Translation ist eine Isometrie.
		\end{lemma}
		\begin{proof}
			Klar.
		\end{proof}

		\begin{theorem}[Charakterisierung Isometrie]
		\label{theorem:charakterisierung isometrie}
			Sei $\beta$ eine Isometrie auf $\SR^n$ für $n\in\SN$.
			Dann gibt es genau ein $A\in O_n$ und genau ein $b\in\SR^n$, sodass für alle $x\in\SR^n$
			\[
				\beta(x) = Ax + b
			\]
		\end{theorem}
		\begin{proof}
			Sei $\beta(0)=b\in\SR^n$. 
			Dann gilt $t_{-b}\circ \beta(0) = 0$.
			Dann ist $t_{-b}\circ \beta$ eine homogene Isometrie und gerade durch die Linksmultiplikation mit einer orthogonalen Matrix $A\in O_n$ gegeben.
			\[
				\Leftrightarrow \ t_{-b}\circ \beta(x)=Ax \ \Leftrightarrow \ \beta(x) = Ax + b
			\]
		\end{proof}

		Nach dem obigen Beweis für Theorem \ref{theorem:charakterisierung isometrie} sind $A$ und $b$ für ein $\beta$ eindeutig bestimmt.
		Die euklidischen Bewegungen teilen sich damit gerade in Spiegelungen, Drehungen und Translationen auf.

		\begin{definition}[Orientierung]
			Sei $A\in O_n$ für $n\in\SN$ ein orthogonaler Endomorphismus.
			$A$ heißt orientierungserhaltend, wenn $\det A = 1$, und orientierungsumkehrend, wenn $\det A = -1$.
			Eine analoge Definition sei für den homogenen Anteil der Isometrien gegeben.
		\end{definition}

		\begin{lemma}[$A\in SO_3$ besitzt 1 als Eigenwert]
			Für alle $A\in SO_3$ gilt $1\in\lambda(A)$.
		\end{lemma}
		\begin{proof}
			\[
				\det (A-I) = \det(A^T)\det(A-I) = \det((E-A)^T) = \det(E-A) = (-1)^n \det(A-I)
			\]
		\end{proof}

		\begin{corollary}[Charakterisierung Drehungen]
			Die Drehungen in $\SR^n$ sind gerade die homogenen orientierungserhaltenden Isometrien.
		\end{corollary}
		\begin{proof}
			Es ist bereits klar, dass jede Drehung eine Isometrie ist.
			Die Determinante des zugehörigen Endomorphismus muss stetig vom Winkel abhängen.
			Die Drehung um den Winkel $0$ entspricht $\det I = 1$.
			Es müssen damit alle Determinanten $1$ sein, da sie sonst nur $-1$ sein könnten.

			Jede Drehung kann auf den zweidimensionalen Fall zurückgeführt werden.
			Bestimme den Unterraum, der auf sich selbst abgebildet wird und berechne die Matrix, welche die Einheitsvektoren in dieser Ebene abbildet.
		\end{proof}

	% subsection skalarprodukt_und_orthogonale_matrizen (end)

	\section{Bilinearformen} % (fold)
	\label{sec:bilinearformen}
	
		Im folgenden sei $K$ ein Körper und $V$ ein $K$-Vektorraum.

		\begin{definition}[Bilinearform]
			Eine Bilinearform $b$ auf V ist eine Abbildung $b:V\times V\longrightarrow K$, sodass für alle $u,v,w\in V$ und alle $\alpha,\beta\in K$ gilt 
			\[
				b(\alpha v + \beta w, u) = \alpha b(v,u) + \beta b(w,u)
			\]
			\[
				b(u, \alpha v + \beta w) = \alpha b(u,v) + \beta b(u,w)
			\]
		\end{definition}

		\textsc{Beispiel:}\\
		Das Standardskalarprodukt ist eine Bilinearform.

		\begin{definition}[symmetrische und schiefsymmetrische Bilinearform]
			Eine Bilinearform $b$ heißt symmetrisch, wenn für alle $v,w\in V$ gilt
			\[
				b(v,w) = b(w,v)
			\]
			und schiefsymmetrisch, wenn für alle $v,w\in V$ gilt
			\[
				b(v,w) = -b(w,v) \text{ oder } b(v,v) = 0
			\]
		\end{definition}

		Für das Skalarprodukt gilt, dass 
		\[
			\angleb{v,v} > 0
		\]
		für alle $v\in V$ mit $v\neq 0$.

		\begin{definition}[positiv definite Bilinearform]
			Eine Bilinearform $b$ heißt positiv definit, wenn
			\[
				b(v,v) > 0
			\]
			für alle $v\in V$ mit $v\neq 0$ gilt.
			(Hierbei muss die Relation $>$ im Körper sinnvoll definiert sein.)
		\end{definition}

		\subsection{Bilinearformen und Matrizen} % (fold)
		\label{sub:bilinearformen_und_matrizen}
		
			Sei nun $V = K^n$ und $B=\curlb{e_1,\ldots,e_n}$ eine Basis von $V$.
			Sei $A\in M_n(K)$. Wir setzen nun
			\[
				b_A(v,w) = v^TAw \in K
			\]
			Dann ist $b_A$ eine Bilinearform auf $V$.
			Es gilt, wenn $A=(a_{ij})$,
			\[
				b(e_i,e_j) = e_i^TAe_j = a_{ij}
			\]

			\textsc{Beispiel:}\\
			\[
				b(\alpha,\gamma) := \alpha_1\gamma_2 - \alpha_2\gamma_1
			\]
			Dann ist
			\[
				A = \curvb{ \begin{matrix}
					0 & 1 \\
					-1 & 0
				\end{matrix} }
			\]
			Sowohl $A$ als auch $b$ sind schiefsymmetrisch.

			\begin{lemma}[Charakterisierung symmetrischer Bilinearformen]
				Die Bilinearform $b_A$ ist genau dann symmetrisch, wenn gilt
				\[
					A=A^T
				\]
				Also wenn $A$ symmetrisch ist.
			\end{lemma}
			\begin{proof}
				$\Rightarrow$:\\
				\[
					a_{ij} = b_A(e_i,e_j) = b(e_j,e_i) = a_{ji}
				\]
				$\Leftarrow$:\\
				\[
					b_A(v,w) = v^TAw = \curvb{v^TAw}^T = w^TA^Tv = w^TAv = b_A(w,v)
				\]
			\end{proof}

			Betrachtet man einen Basiswechsel, so betrachtet man $\tilde{A}=CAC^T$.
			\[
				\tilde{b}(Cv,Cw) = \curvb{Cv}^T \tilde{A} Cw
			\]
			\[
				\tilde{A} = \curvb{C^{-1}}^T A C^{-1}
			\]

			\begin{definition}[Äquivalenz Bilinearformen]
				Zwei Bilinearformen $b_1,b_2$ heißen äquivalent, wenn es ein $C\in GL_n(K)$ gibt, sodass
				\[
					A_2=C^TA_1C
				\]
				$A_1,A_2$ beschreiben die Matrizen von $b_1,b_2$.
			\end{definition}

			\begin{corollary}[Übergangsformel]
				Eine Bilinearform $b_A$ ist genau dann zum Standardskalarprodukt äquivalent, wenn 
				\[
					A=C^TC
				\]
				mit $C\in GL_n(K)$.
			\end{corollary}

			\begin{definition}[positiv definite Matrix]
				Eine Matrix $A\in M_n(K)$ heißt positiv definit, falls für alle $v\in V$ mit $v\neq 0$ gilt
				\[
					v^TAv > 0
				\]
			\end{definition}

			\begin{theorem}[Charakterisierung positiv definit]
				Eine Matrix $A$ ist genau dann positiv definit, wenn $b_a$ positiv definit ist.
			\end{theorem}

			\begin{theorem}[Charakterisierung symmetrisch und positiv definite Matrix]
				Für $A\in M_n(K)$ sind folgende Aussagen äquivalent.
				\begin{enumerate}[label = \normalfont(\roman*)]
					\item $A$ repräsentiert bezüglich einer geeigneten Basis das Standardskalarprodukt.
					\item $A=C^TC$ für ein geeignetes $C\in GL_n(K)$
					\item $A=A^T$ und $A$ ist positiv definit.
				\end{enumerate}

			\end{theorem}
			\begin{proof}
				(i)$\Rightarrow$(ii):\\
				$A=C^TIC$ ist invertierbar.
				
				(ii)$\Rightarrow$(iii):\\
				Gegeben sei $A=C^TC$.
				\[
					A^T = C^TC = A
				\]
				(iii)$\Rightarrow$(i):\\
				Suchen orthonormale Basis für $A$ oder für $b_A$ mithilfe des Gram-Schmidtschen Orthogonalisierungsverfahren.
				Danach wird eine Induktion über die Dimension durchgeführt.
			\end{proof}

			\begin{definition}[Gram-Schmidtsches Orthonormierungsverfahren]
				Sei $B = \curlb{v_1,\ldots,v_n} \subset V$ für $n\in\SN$ eine Menge linear unabhängiger Vektoren.
				Dann ist das Gram-Schmidtsche Orthonormierungsverfahren gerade die Abbildung, welche $B$ auf $U=\curlb{w_1,\ldots,w_n}$ abbildet, sodass für alle $i\in\SN_0$ mit $i<n$ gilt
				\[ w_{i+1} = \frac{v_{i+1} - \sum_{k=1}^i \angleb{v_{i+1}, w_k}w_k}{\norm{v_{i+1} - \sum_{k=1}^i \angleb{v_{i+1}, w_k}w_k}} \]
				$U$ bildet dann ein Orthonormalsystem.
				Hierbei kann $\angleb{\cdot,\cdot}$ eine beliebige symmetrische positive definite Bilinearform darstellen.
				$\norm{\cdot}$ stellt dann die dazu induzierte Norm dar. 
			\end{definition}

			Hier muss ein Beispiel für symmetrische positiv definite 2x2-Matrizen kommen (Charakterisierung).
			

		% subsection bilinearformen_und_matrizen (end)

		\subsection{Indefinite Bilinearformen} % (fold)
		\label{sub:indefinite_bilinearformen}
		
			\textsc{Beispiel:}\quad (Lorentzform)\\
			Es gilt $V = \SR^4$ und für ein $c>0$
			\[ b(x,y) = x_1y_1 + x_2y_2 + x_3y_3 - c^2x_4y_4 \]

			\begin{lemma}
				Sei $b$ eine nicht-triviale symmetrische Bilinearform auf $\SR^n$.
				Dann gibt es ein $v\in\SR^n$, sodass $b(v,v)\neq 0$.

			\end{lemma}
			\begin{proof}
				Es gilt für zwei Vektoren $u,v\in\SR^n$, dass $b(u,v)\neq 0$.
				Ist $b(u,u)\neq 0$ oder $b(v,v)\neq 0$, dann ist die Aussage gezeigt.
				Sonst gilt $b(u,u)=b(v,v)=0$ und für $w=u+v$ gilt dann
				\[ b(w,w) = b(u,u) + 2b(u,v) + b(v,v) = 2b(u,v) \neq 0 \]
			\end{proof} 

			\begin{definition}[orthogonales Komplement]
				Sei $b$ eine Bilinearform auf $V=K^n$ und $W\subset V$ ein Unterraum von $V$.
				Das orthogonale Komplement von $W$ bezüglich $V$ ist der Unterraum
				\[ W_b^\perp = \set{v\in V}{\forall w\in V:\ b(v,w) = 0} \]
			\end{definition}

			Es kann passieren, dass $W\cap W^\perp_b \neq \curlb{0}$.

			\begin{lemma}
				Ist $b(v,v)=0$, so gilt für $W = \angleb{v}$ (lineare Hülle), dass
				\[ W \oplus W^\perp_b = V \]

			\end{lemma}
			\begin{proof}
				Klar, dass $v\notin W_b^\perp$, da $b(v,v)\neq 0$.
				Der Unterraum $W_b^\perp$ wird bestimmt mit Hilfe der linearen Gleichung $b(x,v)=0$.
				Die Lösung dafür lautet
				\[ u-\frac{b(u,v)}{b(v,v)}v \in W_b^\perp \]
				Dieser Raum besitzt also die Dimension $n-1$.
			\end{proof}

			\begin{definition}[Radikal und Nicht-Entartung]
				Seien $n\in\SN$, $K$ ein Körper und $V=K^n$ der zugehörige $n$-dimensionale $K$-Vektoraum. 
				Sei nun $b$ eine symmetrische Bilinearform auf $V$.
				Der Unterraum $V_b^\perp$ heißt das Radikal von $b$.
				Die Bilinearform heißt nicht-entartet, wenn $V_b^\perp = \curlb{0}$
			\end{definition}

			\begin{lemma}
				Sei $b=b_A$ mit $A=A^T$.
				Dann gilt
				\begin{enumerate}[label = \normalfont (\alph*)]
					\item $V_b^\perp = Kern A$
					\item $b_A$ ist nicht-entartet $\ \Leftrightarrow \ \det A \neq 0$
				\end{enumerate}
			\end{lemma}
			\begin{proof}
				(a): Es gilt nach Definition $b_A(v,w) = v^TAw$.
				Sei $\curlb{e_1,\ldots,e_n}$ die Basis in $V$.
				Dann falls $v^TAe_i = 0$ für alle $i$ folgt $v^TA = 0$.
				Andererseits falls $v^TA = 0$ gilt $v^TAw = 0$ für alle $w$.
				Nun $v^TA = (A^Tv)^T = (Av)^T$ und damit $V_b^\perp = Kern A$.\\
				(b): es gilt $Kern A = 0 \ \Leftrightarrow \ \det A \neq 0$ 
			\end{proof}

			\begin{theorem}
				Sei $b$ eine symmetrische Bilinearform. Dann gibt es eine Basis
				$B=\curlb{v_1,\ldots,v_n}$, sodass die Vektoren orthogonal zueinander bezüglich $b$ stehen und $b(v_i,v_i)\in\curlb{1,-1,0}$.\\
				Sei $A\in M_n(\SR)$ symmetrisch. Dann existiert ein $C\in GL_n(\SR)$, sodass $CAC^T$ eine Diagonalmatrix ist mit den Einträgen aus der Menge $\curlb{1,-1,0}$.
			\end{theorem}
			\begin{proof}
				Sei $b$ nicht trivial.
				Dann gibt es ein $v\in V$ mit $b(v,v)\neq 0$.
				Setze
				\[ v_n := \frac{v}{\sqrt{\abs{b(v,v)}}} \]
				Zerlege nun $V$ in $V = \angleb{v_n} \oplus \angleb{v_n}_b^\perp$.
				Die Einschränkung von $b$ auf $\angleb{v_n}_b^\perp$ ist wieder eine symmetrische Bilinearform mit Dimension $n-1$.
				Führe nun eine Induktion durch.
				Die Matrix $CAC^T$ hat die Gestalt
				\[ CAC^T = \diag(\underbrace{1,\ldots,1}_{p}, \underbrace{-1,\ldots,-1}_{q},\underbrace{0,\ldots,0}_{s}) \]
				mit $p+q+s=n$. Das Paar $(p,q)$ heißt Signatur von $b_A$.
			\end{proof}

			\begin{theorem}
				Die Zahlen $p,q,s$ sind durch die Bilinearform $b_A$ eindeutig bestimmt.
				Sie hängen also nicht von der Wahl der Orthogonalbasis ab.
			\end{theorem}
			\begin{proof}
				$s=n-(p+q) = \dim (CAC^T) = \dim A$
				Zeige nun das $p$ eindeutig ist.
				Sei $B^\prime = \curlb{v_1^\prime,\ldots,v_n^\prime}$ eine andere Orthogonalbasis mit $p^\prime < p$.
				Wir zeigen, dass die Vektoren $\curlb{v_1,\ldots, v_p, v^\prime_{p+1},\ldots, v^\prime_n}$ linear unabhängig sind.
				Betrachte eine nicht-triviale Abhängigkeit
				\[ \underbrace{ \alpha_1v_1 + \ldots + \alpha_pv_p }_{=:u} = \underbrace{ \gamma_{p^\prime +1}v^\prime_{p^\prime +1} + \ldots + \gamma_nv^\prime_n }_{=w} \]
				Dann
				\[ b(u,u) = \sum_{i=1}^p \alpha_i^2 \geq 0 \]
				\[ b(w,w) = -\sum_{j=1}^{q^\prime} \gamma^2_{p^\prime +j} \leq 0 \]
				\[ \sum_{i=1}^p \alpha_i^2 = \sum_{j=1}^{q^\prime} \gamma^2_{p^\prime +j} = 0 \ \Rightarrow \ \alpha_i^2 = \gamma_{p^\prime +j} = 0 \text{ für alle } i\in\curlb{1,\ldots,p}, j\in\curlb{1,\ldots,q^\prime} \]
				Alle übrigen Koeffizienten müssen ebenfalls Null sein, da es sich um eine Basis handelt.
				Die Vektoren sind also doch linear unabhängig.
				Das sind insgesamt $p+n-p^\prime$ Vektoren.
				\[ p+n-p^\prime \leq n \ \Rightarrow \ p\leq p^\prime \ \Rightarrow  \]
				Schließlich gilt dann $p^\prime = p$
			\end{proof}

		% subsection indefinite_bilinearformen (end)

	% section bilinearformen (end)

	\section{Vektorräume über dem Körper $\SC$} % (fold)
	\label{sec:der_k_rper_}

		\subsection{Idee komplexes Skalarprodukt} % (fold)
		\label{sub:idee_komplexes_skalarprodukt}
	
			Der Körper $\SC$ kann auch geschrieben werden als
			\[ \SC = \SR \oplus i\SR \quad \Rightarrow \quad \dim_\SR\SC = 2 \]
			Die Addition $+$ auf $\SC$ sei wie auf dem $\SR^2$ eingeführt.
			Die Multiplikation für $a+ib,c+id\in\SC$ ist definiert durch
			\[ (a+ib)(c+id) = (ac-bd) + i(ad+bc) \]
			Es entsteht nun ein Problem, wenn man die Länge eines Vektors berechnen möchte, da die induzierte Norm im Allgemeinen nicht größer gleich Null sein wird.
			Es gibt also $z\in\SC$ mit
			\[ \angleb{z,z} = zz \not\geq 0 \]
			Um dieses Problem zu umgehen, möchte man, dass die Länge in $\SC$ sich gerade aus der Länge in $\SR$ ergibt.
			Man führt ein neues Skalarprodukt, welches positiv definit ist und gerade die Norm aus dem $\SR$ beibehält.
			\[ \angleb{z,z}_\SC := \bar{z}z = \abs{z}^2 \geq 0 \]
			Aufgrund dieser Betrachtung erhält man nun eine Idee für ein neues Skalarprodukt auf dem $\SC^n, n\in\SN$, welches die Länge erhält.

		% subsection idee_komplexes_skalarprodukt (end)
	
		\subsection{Hermitesche Formen} % (fold)
		\label{sub:hermitesche_formen}

			\begin{definition}[hermitisches Standardskalarprodukt]
				Sei $n\in\SN$.
				Die folgende Abbildung stellt das hermitesche Standardskalarprodukt auf dem $\SC^n$ dar.
				\[ \angleb{\cdot,\cdot}_\SC:\SC^n\times\SC^n\longrightarrow\SC,\qquad \angleb{x,y}_\SC = \angleb{\bar{x},y} = \bar{x}\transp y \]
				Im Folgenden wird diese Abbildung auch Skalarprodukt auf $\SC^n$ genannt oder als $\angleb{\cdot,\cdot}$ geschrieben.
			\end{definition}

			Es ist nun angebracht eine Verallgemeinerung für das hermitesche Standardskalarprodukt analog zu den Bilinearformen zu konstruieren.
			Dafür betrachtet man die grundlegendsten Eigenschaften des Skalarproduktes. 

			\begin{definition}[Sesquilinearform]
				Seien $V$ ein $\SC$-Vektorraum und eine Abbildung $s:V\times V\longrightarrow \SC$ gegeben.
				Dann ist $s$ eine Sesquilinearform genau dann, wenn für alle $\alpha,\beta\in\SC$ und alle $u,v,w\in V$ gilt
				\begin{enumerate}[label = \normalfont (S\arabic*)]
					\item $s(u, \alpha v + \beta w) = \alpha s(u,v) + \beta s(u,w) \hfill \text{(Linearität)}$
					\item $s(\alpha u + \beta v, w) = \bar{\alpha} s(u,w) + \bar{\beta}s(v,w) \hfill \text{(Semilinearität)}$
				\end{enumerate}
			\end{definition}

			Hier wird gefordert, dass die Linearität im zweiten Argument und die Semilinearität im ersten Argument gilt.
			Es ist allerdings auch gebräuchlich dies in umgekehrter Weise zu definieren.

			\begin{definition}[hermitesche Form]
				Eine hermitesche Form auf einem $\SC$-Vektorraum $V$ ist eine Sesquilinearform $H:V\times V\longrightarrow\SC$, sodass für alle $x,y\in V$ gilt
				\[ H(x,y) = \overline{H(y,x)} \hfill \text{(hermitesche Symmetrie)} \]
				Eine hermitesche Form wird auch hermitesches Skalarprodukt oder auch hermitesche Sesquilinearform genannt.
			\end{definition}

			Damit ist $\angleb{\cdot,\cdot}$ eine hermitesche Form auf dem $\SC^n$.
			Eine der im Folgenden betrachten Haupteigenschaften von solchen durch Matrizen induzierten Formen wird die Hermitizität oder auch hermitesche Symmetrie sein.

			\begin{definition}[hermitesche Matrix]
				Sei $A\in\m{M}_n(\SC)$ für ein $n\in\SN$.
				Dann ist $A$ hermitesch genau dann, wenn
				\[ \bar{A} = A\transp \]
				Hierbei ist $\bar{A} := (\bar{a}_{ij})$ die komplex konjugierte Matrix zu $A$.
				Es sei weiterhin $A^\dagger := \bar{A}\transp$.
			\end{definition}

			Das Analogon einer orthogonalen Matrix für hermitesche Formen ist die unitäre Matrix.

			\begin{definition}[unitäre Matrix, $\m{U}_n$]
				Sei $n\in\SN$.
				Ein $C\in\m{M}_n(\SC)$ heißt unitär genau dann, wenn
				\[ C^\dagger C = \idmat \]
				Damit bleibt das Standardskalarprodukt erhalten.
				Die Menge der unitären Matrizen sei dann
				\[ \m{U}_n := \set{A\in \m{M}_n(\SC)}{ A^\dagger A = \idmat } \]
			\end{definition}

			Gilt $A\in\m{M}_n(\SR)$, dann ist orthogonal äquivalent zu unitär.
			Analoges gilt für hermitesche und symmetrische Matrizen.
			Für die Übergangsformel gilt
			\[ \tilde{A} = C^\dagger A C \]

			\begin{lemma}[$\m{U}_n$ bildet Gruppe]
				$\m{U}_n$ bildet für $n\in\SN$ bezüglich der Matrixmultiplikation eine Gruppe.
			\end{lemma}
			\begin{proof}
				Klar.
			\end{proof}

			Analog lässt sich auch die Orthogonalität von Vektoren übertragen.
			Diese wird vor Allem gebraucht, um Orthonormalbasen im komplexen Vektorraum zu definieren.

			\begin{definition}[Orthogonalität komplexer Vektoren]
				Sei $V$ ein $\SC$-Vektorraum und $H:V\times V\longrightarrow\SC$ eine hermitesche Form auf $V$.
				Dann stehen zwei Vektoren $v,w\in V$ orthogonal zueinander genau dann, wenn 
				\[ H(v,w) = 0 \]
				In diesem Falle schreibt man auch $v\perp_H w$ oder auch nur $v\perp w$.
			\end{definition}

			Für das vollständige Konzept einer Orthonormalbasis wird nun noch eine analoge Eigenschaft zur positiv Definitheit benötigt, um so die Normierung von Vektoren zu ermöglichen.

			\begin{definition}[positiv definite hermitesche Form]
				Sei $V$ ein $\SC$-Vektorraum und $H:V\times V\longrightarrow\SC$ eine hermitesche Form auf $V$.
				$H$ heißt positiv definit, wenn für alle $v\in V\setminus\curlb{0}$ gilt
				\[ H(v,v) > 0 \]
				Diese Eigenschaft ist wohldefiniert, da $H(v,v) = \overline{H(v,v)}$ und damit $H(v,v)\in\SR$ folgt.
			\end{definition}

			Diese Eigenschaft lässt sich so nur für hermitesche Formen formulieren.
			Dies war bei positiv definiten Bilinearformen nicht der Fall.

			\begin{theorem}[Charakterisierung positiv Definitheit]
				Eine hermitesche Form $H$ ist genau dann positiv definit, wenn es eine Orthonormalbasis bezüglich $H$ gibt.
			\end{theorem}

			\begin{theorem}[Spektralsatz]
				Sei $V$ ein $\SC$-Vektorraum.
				\begin{enumerate}[label = \normalfont (\alph*)]
					\item Sei $f:V\longrightarrow V$ eine hermitesche lineare Abbildung.
						Dann gibt es eine Orthonormalbasis von $V$, die aus Eigenvektoren von $f$ zu reellen Eigenwerten besteht.

					\item Sei eine hermitesche Matrix $M$ gegeben.
						Dann gibt es ein $C\in\m{U}_n$, sodass
						\[ C^\dagger MC \in\m{M}_n(\SR) \]
						eine Diagonalmatrix ist.
				\end{enumerate}
			\end{theorem}
			\begin{proof}
				(a):
				Wir benutzen, dass $\SC$ algebraisch abgeschlossen ist (siehe Algebra II).
				Sei
				\[ \chi_M := \det (X\idmat - M) \]
				das charakteristische Polynom zu $M$.
				Dieses Polynom besitzt komplexe Koeffizienten.
				Aufgrund der algebraischen Abgeschlossenheit folgt dann 
			\end{proof}

			Der Spektralsatz begründet noch einmal, weshalb man ein neues Skalarprodukt für Vektorräume über den komplexen Zahlen konstruiert.
			Diese Betrachtung ermöglicht es nun hermitesche Matrizen zu diagonalisieren.

			\begin{corollary}[hermitesche Matrizen sind diagonlisierbar]
				Jede hermitesche Matrix ist diagonalisierbar.
			\end{corollary}

			\begin{theorem}
				Seien $H$ eine hermitesche Form auf einem $\SC$-Vektorraum $V$ und $M$ die zugehörige Matrix.
				Dann sind alle Eigenwerte von $M$ reell, das heißt
				\[ \lambda(M)\subset\SR \]
				Außerdem gibt es ein $C\in\m{U}_n$, sodass $C^\dagger M C$ Diagonalform hat und $\SR\subset\SC$ als ein Teilkörper abgeschlossen ist.
			\end{theorem}

			\begin{corollary}
				Sei $A\in\m{M}_n(\SR)$ für $n\in\SN$ hermitesch.
				Dann sind alle Eigenwerte von $A$ reell und $A$ ist in $\m{M}_n(\SC)$ diagonalisierbar.
			\end{corollary}

			Seien $A\in\m{M}_n(\SR)$, $\lambda\in\SC$ ein Eigenwert von $A$ und $v_\lambda\in\SC^n$ ein zugehöriger Eigenvektor.
			Dann folgt 
			\[ \lambda\in\SR \quad \equivalent \quad v_\lambda \in \SR \]

			Warum besitzt nun eine symmetrische reelle Matrix einen reellen Eigenwert?
			Dies ist eigentlich eine Frage der Analysis.
			Sei $n\in\SN$.
			Es ist dann
			\[ S^{n-1} := \set{v\in\SR^n}{\angleb{v,v}=1} \]
			eine kompakte Teilmenge des $\SR^n$.
			Sei nun $A\in\m{M}_n(\SR)$ eine symmetrische Matrix.
			\[ F_A:\SR^n \longrightarrow \SR,\qquad F_A(v):=v\transp A v \]
			$F_A$ ist nun eine stetige kontinuierliche Funktion.
			$F_A|_{S^{n-1}}$ ist also begrenzt und die nimmt maximale und minimale Werte an.
			Sei $L$ nun ein maximaler Wert.
			\[ L=F_A(v_0),\qquad v_0\transp v_0 = 1 \]
			\[ \diff F_A(v_0) \text{ ist Null auf } T_{v_0}S^{n-1} \]
			Das heißt also, dass für alle $w\in\SR^n$
			\[ w\transp Av_0 + v_0\transp Aw = 0 \quad \implies \quad w\transp v_0 = 0 \]
			\[ \implies \quad v_0\transp Aw = 0 \quad \implies \quad v_0^\perp \subset (Av_0)^\perp \quad \implies \quad \angleb{Av_0}\subset\angleb{v_0} \]
			\[ Av_0 \in \set{rv_0}{r\in\SR} \]
			$v_0$ ist also ein Eigenvektor von $A$.
			Wenn nun $Av_0 = rv_0$, dann folgt
			\[ \implies \quad F_A(v_0) = L = r \]

			\begin{theorem}[Spektralsatz im reellen Fall]
				Sei $n\in\SN$.
				\begin{enumerate}[label = \normalfont (\alph*)]
					\item Sei $f:\SR^n\longrightarrow\SR^n$ eine lineare Abbildung mit $angleb{f(v),w} = \angleb{v,f(w)}$ für alle $v,w\in\SR^n$.
						Dann existiert eine Orthonormalbasis in $\SR^n$, wo alle Vektoren Eigenvektoren von $f$ sind.
						Alle Eigenvektoren von $f$ sind reell und $f$ ist diagonalisierbar.
					\item Sei $A\in\m{M}_n(\SR)$ symmetrisch.
						Dann existiert ein $C\in\m{O}_n$, sodass $C\transp AC$ eine Diagonalmatrix ist.
				\end{enumerate}

			\end{theorem}
			\begin{proof}
				Es sei hier $f(v)=Av$.
				$f$ ist damit auch symmetrisch (da es selbstadjungiert ist).
				Die beiden getroffenen Aussagen sind äquivalent.

				(a)$\implies$(b):\\
				zu zeigen: Es gibt ein $C\in\m{O}_n$, sodass $C\transp A C = \diag (\lambda_1,\ldots,\lambda_n)$.
				Es gibt ein $v\in\SR^n$ mit $Av=\lambda v$ für $\lambda\in\SR$ mit $v\neq 0$.
				\[ v_1 := \frac{v}{\norm{v}},\qquad \norm{v}=\sqrt{v\transp v}> 0 \]
				\[ \SR^n = \angleb{v_1}\oplus v^\perp \]
				Aus $\angleb{v,f(w)}=\angleb{f(v),w}=\angleb{\lambda v, w}$ folgt es, dass $v^\perp$ stabil ist.
				\[ \angleb{v,w}=0 \quad \implies \quad \angleb{v,f(w)}=0 \]
				Die Induktion über $n$ bringt uns dann eine Orthonormalbasis.
			\end{proof}

			\textsc{Beispiel:}\\
			Sei $n=2$ und
			\[ A = \begin{pmatrix}
				a & b \\ b & d
			\end{pmatrix},\qquad \chi_A(x) := (x-a)(x-b) - b^2 \]
			Die Nullstellen liegen hier in $\SR$, wie man sich durch eine Parabel leicht veranschaulichen kann.

		% subsection hermitesche_formen (end)

		\subsection{Schiefsymmetrische Bilinearformen} % (fold)
		\label{sub:schiefsymmetrische_bilinearformen}
		
			Im Körper $\mathds{F}_2$ ist $2=0$ und damit auch $-1=1$.
			Führt man eine Bilinearform $b$ auf einem $\mathds{F}_2$-Vektorraum ein, so bedeutet Schiefsymmetrie nichts weiter als
			\[ b(v,w) = -b(w,v) = b(w,v) \]
			$b$ wäre also auch symmetrisch.
			Um dieses Problem zu umgehen, führt man eine etwas abgewandelte Definition ein.

			\begin{definition}[schiefsymmetrische Bilinearformen]
				Sei $n\in\SN$.
				Eine Bilinearform $b$ auf $V=K^n$, wobei $K$ ein Körper ist, heißt schiefsymmetrisch, falls für alle $v\in V$ gilt
				\[ b(v,v) = 0 \]
			\end{definition}

			Hierbei muss das Folgende beachtet werden.
			\[ 0 = b(v+w,v+w) = b(v,v) + b(v,w) + b(w,v) + b(w,w) = b(v,w) + b(w,v) \]
			\[ \implies \quad b(v,w) = -b(w,v) \]
			Die oben geforderte Eigenschaft bedeutet also gerade Schiefsymmetrie.
			Für einen Körper $K$, indem $0\neq 2$ gilt, folgt dann auch
			\[ b(v,w) = -b(w,v) \quad \implies \quad b(v,v)=-b(v,v) \quad \implies \quad b(v,v) = 0 \]
			Sei $b = b_A$ schiefsymmetrisch.
			Dann gilt $A\transp = -A$ und damit auch $a_{ii}=0$ für alle $i\in\SN, i\leq n$.
			\[ v \perp_b w \quad \implies \quad b(v,w) = 0 \]
			\[ b(v,w) = 0 \quad \equivalent \quad b(w,v) = 0 \]
			\[ b \text{ nicht entartet} \quad \equivalent \quad \set{v}{b(v,w)=0} = \curlb{0} \quad \equivalent \quad \det A \neq 0 \]

		% subsection schiefsymmetrische_bilinearformen (end)

	% section der_k_rper_ (end)

	\subsection{Der orientierte Winkel} % (fold)
	\label{sub:der_orientierte_winkel}
	
		Seien $v,w\in\SR^2\setminus\curlb{0}$ linear unabhängig und $\gamma:=\angle_\pm(v,w)$.
		\[ \cos\gamma := \begin{cases} \cos\angle(v,w) & :\det(v,w)>0 \\ -\cos\angle(v,w) & :\mathrm{sonst} \end{cases} \]

	% subsection der_orientierte_winkel (end)

	\subsection{Kreuzprodukt} % (fold)
	\label{sub:kreuzprodukt}
	
		\begin{definition}[Kreuzprodukt]
			Das Kreuzprodukt ist eine bilineare Verknüpfung $\times:\SR^3\times\SR^3\longrightarrow\SR^3$, sodass für alle $v,w,u\in\SR^3$ gilt
			\begin{enumerate}[label=\normalfont(K\arabic*)]
				\item $v\times w = -w\times v$ \hfill (alternierend)
				\item $(v,w,u)\in\m{SO}_3 \quad \implies \quad v\times w = u$
			\end{enumerate}
		\end{definition}

		\begin{theorem}[Existenz und Formel des Kreuzprodukts]
			Das Kreuzprodukt existiert und ist für $v,w\in\SR^3$ durch die folgende Formel eindeutig bestimmt.
			\[
				v\times w =
				\begin{pmatrix}
					v_1 \\ v_2 \\ v_3
				\end{pmatrix}
				\times
				\begin{pmatrix}
					w_1 \\ w_2 \\ w_3
				\end{pmatrix}
				=
				\begin{pmatrix}
					v_2w_3 - v_3w_2 \\
					v_3w_1 - v_1w_3 \\
					v_1w_2 - v_2w_1
				\end{pmatrix}
			\]
		\end{theorem}
		\begin{proof}
			Es sei $\transp{(c_1,c_2,c_3)}:=v\times w$.
			\[ \implies \quad \det(v,w,v\times w) = c_1^2 + c_2^2 + c_3^2 = \norm{v\times w}^2 \]
			Wir betrachten nun
			\[ \angleb{v\times w, v} = \det(v,w,v) = 0 = \det(v,w,w) = \angleb{v\times w, w} \]
			Sei jetzt $(v,w,u)\in\m{SO}_3$.
			\[ \norm{v\times w}^2 = \det(v,w,v\times w) = 1 \]

			Eindeutigkeit:
			\begin{alignat*}{3}
				v\times w &= \sum_{i,j=1}^3 x_iy_j e_i\times e_j
				= \sum_{\substack{i,j=1 \\ i\neq j}}^3 x_iy_j e_i\times e_j
				= \sum_{\substack{i,j=1 \\ i < j}}^3 (x_iw_j - x_jw_i) e_i\times e_j
			\end{alignat*}
		\end{proof}

		(geometrische Bedeutung einfügen)

	% subsection kreuzprodukt (end)

	\section{Jordansche Normalform} % (fold)
	\label{sec:jordansche_normalform}
	
		\begin{definition}[Hauptraum von $f$ zu $\lambda$]
			Seien $K$ ein Körper, $V$ ein $K$-Vektorraum und $f\colon V\funcarr V$ eine lineare Abbildung.
			Für ein $\lambda\in K$ heißt die folgende Menge der Hauptraum von $f$ zu $\lambda$.
			\[ \m{Hau}(f,\lambda) := \set{v\in V}{ \exists n\in\SN:\quad (f-\lambda\id_V)^n(v) = 0 } \]
		\end{definition}

		\begin{lemma}[Hauptraum ist Unterraum]
			Seien $K$ ein Körper, $V$ ein $K$-Vektorraum, $f\colon V\funcarr V$ eine lineare Abbildung und $\lambda\in K$.
			Dann ist $\m{Hau}(f,\lambda)$ ein linearer Unterraum von $V$.
		\end{lemma}
		\begin{proof}
			Es ist $\m{Hau}(f,\lambda) \neq \emptyset$, da $0\in\m{Hau}(f,\lambda)$.
			Seien $v,w\in\m{Hau}(f,\lambda)$ und $\alpha\in K$, dann gibt es $n,m\in\SN$, sodass
			\[ (f-\lambda\id)^n(v) = (f-\lambda\id)^m(w) = 0 \]
			Setzt man nun $p:=\max\curlb{n,m}$, folgt
			\[ (f-\lambda\id)^p(v+\alpha w) = (f-\lambda\id)^p(v) + \alpha(f-\lambda\id)^p(w) = 0 \]
			\[ \implies \quad v+\alpha w \in \m{Hau}(f,\lambda) \]
		\end{proof}

		\begin{lemma}[Charakterisierung nicht-trivialer Hauptraum]
			Seien $K$ ein Körper, $V$ ein $K$-Vektorraum, $f\colon V\funcarr V$ eine lineare Abbildung und $\lambda\in K$.
			\[ \m{Hau}(f,\lambda) \neq \curlb{0} \quad \equivalent \quad \lambda \in \sigma(f) \]
		\end{lemma}
		\begin{proof}
			$\Longleftarrow$:
			Sei $\lambda$ ein Eigenwert von $f$.
			In diesem Fall gibt es ein $v\in V$ mit $f(v)=\lambda v$.
			Damit gilt dann auch 
			\[ (f-\lambda\id)(v) = 0 \quad \implies \quad v\in\m{Hau}(f,\lambda) \]

			$\implies$:
			Sei nun $v\in\m{Hau}(f,\lambda)$ mit $v\neq 0$.
			Sei nun $n\in\SN$ die kleinste Zahl mit
			\[ 0 = (f-\lambda\id)^n(v) \]
			Ist $n=1$, folgt direkt, dass $\lambda$ ein Eigenwert ist.
			Für $n>1$ definiert man
			\[ u:= (f-\lambda\id)^{n-1}(v) \neq 0 \quad \implies \quad (f-\lambda\id)(u) = 0 \]
			$u$ ist damit ein Eigenvektor zum Eigenwert $\lambda$.
		\end{proof}

		\begin{definition}[nilpotenter Endomorphismus]
			Seien $K$ ein Körper und $V$ ein $K$-Vektorraum.
			Eine lineare Abbildung $\func{f}{V}{V}$ heißt nilpotent, wenn es ein $n\in\SN$ gibt, sodass
			\[ f^n = 0 \]
			Analoges sei für die zugehörige Matrix definiert.
		\end{definition}

		\begin{theorem}[Charakterisierung nilpotente Abbildung]
			Seien $K$ ein Körper und $V=K^n$ für ein $n\in\SN$.
			Für einen Endomorphismus $\func{f}{V}{V}$ mit der zugehörigen Matrix $A\in\m{M}_n(K)$ sind folgende Aussagen äquivalent.
			\begin{enumerate}[label=\normalfont(\roman*)]
				\item $f$ ist nilpotent
				\item $\m{Hau}(f,0)=V$
				\item $A$ ist triagonalisierbar und alle Eigenwerte sind Null.
			\end{enumerate}
		\end{theorem}
		\begin{proof}
			(iii)$\equivalent$(i) wurde bereits in Linearer Algebra I gezeigt.

			(i)$\implies$(ii):
			Ist $f$ nilpotent, dann gibt es ein $k\in\SN$, sodass für alle $v\in V$ gilt
			\[ f^k(v)=0 \quad \implies \quad v\in\m{Hau}(f,0) \]
			Damit folgt direkt $\m{Hau}(f,0) = V$, da $\m{Hau}(f,0)\subset V$.

			(ii)$\implies$(i):
			Sei nun also $\m{Hau}(f,0)=V$.
			Dann gibt es für alle $v\in V$ ein $k_v\in\SN$, sodass $f^{k_v}(v)=0$ gilt.
			Setze nun
			\[ \tilde{k} := \max\set{k_v}{v\in V} \]
			Sei nun $v\in V$.
			\begin{itemize}
				\item Fall $k_v=\tilde{k}$:\\
					Aus der Definition ergibt sich direkt $f^{\tilde{k}}(v) = 0$.
				\item Fall $k_v<\tilde{k}$:\\
					Es gibt nun ein $m\in\SN$, sodass $\tilde{k}=k_v+m$.
					Dafür folgt nun
					\[ f^{\tilde{k}}(v) = f^{m+k_v}(v) = f^m\curvb{ f^{k_v}(v) } = f^m(0) = 0 \]
			\end{itemize}
			Die Funktion $f$ ist also nilpotent.
		\end{proof}

		\begin{lemma}[Hauptraum ist stabil unter kommutierenden Abbildungen]
			Seien $K$ ein Körper, $V$ ein $K$-Vektorraum und $\func{f,h}{V}{V}$ lineare Abbildungen, welche kommutieren (das heißt $f\circ h = h\circ f$).
			Dann ist für alle $\lambda\in K$
			\[ h\curvb{ \m{Hau}(f,\lambda) } \subset \m{Hau}(f,\lambda) \]
			$\m{Hau}(f,\lambda)$ heißt dann auch $h$-stabil.
		\end{lemma}
		\begin{proof}
			Aus Rechenregeln ergibt sich
			\[ h\circ (f-\lambda\id) = h\circ f - \lambda h = f\circ h - \lambda h = \curvb{ f - \lambda\id }\circ h \]
			\[ \implies \quad h\circ (f-\lambda\id)^k = (f-\lambda\id)^k\circ h \qquad \text{für } k\in\SN \]
			Sei nun $v\in\m{Hau}(f,\lambda)$. Dann gibt es ein $k\in\SN$, sodass
			\[ (f-\lambda\id_V)^k(v) = 0 = h\circ\curvb{f-\lambda\id_V}^k(v) = (f-\lambda\id_V)^k(h(v)) \]
			\[ \implies \quad h(v)\in\m{Hau}(f,\lambda) \]
		\end{proof}

		\begin{corollary}[Hauptraum ist stabil unter eigener Abbildung]
			Seien $K$ ein Körper, $V$ ein $K$-Vektorraum und $\func{f}{V}{V}$ eine lineare Abbildungen.
			Für alle $\lambda\in K$ ist $\m{Hau}(f,\lambda)$ ein $f$-stabiler Unterraum.
		\end{corollary}

		\begin{theorem}[Haupträume als direkte Summe]
			Seien $K$ ein Körper und für ein $r\in\SN$ die paarweise verschiedenen Werte $\lambda_1,\ldots,\lambda_r\in K$ gegeben.
			Dann gilt
			\[ \sum_{i=1}^r \m{Hau}(f,\lambda_i) = \bigoplus_{i=1}^r \m{Hau}(f,\lambda_i) \]
		\end{theorem}
		\begin{proof}
			Man führt hier eine Induktion über $r$ durch.
			Für $r=1$ gibt es nichts zu zeigen.

			$r\implies r+1$:
			Wir nehmen an, dass
			\[ S:=\m{Hau}(f,\lambda_{r+1}) \cap \sum_{i=1}^r \m{Hau}(f,\lambda_i) \neq \curlb{0} \]
			Man findet also einen Vektor $u\in S$ mit $u\neq 0$.
			Dann gibt es eindeutige $v_i\in\m{Hau}(f,\lambda_i)$ für $i\in\SN,i\leq r$ und ein kleinstes $n\in\SN$ mit
			\[ u = \sum_{i=1}^r v_i,\qquad h(u):=(f-\lambda_{r+1}\id)^n(u)=0 \]
			\[ \implies \quad 0 = h(u) = h\curvb{\sum_{i=1}^r v_i} = \sum_{i=1}^r h(v_i) \]
			Da die Summe der ersten $r$ Haupträume direkt ist muss $h(v_i)=0$ sein für alle $i\in\SN,i\leq r$.
			Es gibt nun ein $i\in\SN,i\leq r$, sodass $v_i\neq 0$ und der folgende Vektor ein Eigenvektor von $f$ zum Eigenwert $\lambda_{r+1}$ ist.
			\[ \tilde{h}(v_i) := (f-\lambda_{r+1}\id)^{n-1}(v_i) \neq 0 \]
			Weiterhin gibt es ein kleinstes $k\in\SN$ mit
			\[ g(v_i) := (f-\lambda_i\id)^k(v_i) = 0 \quad \implies \quad \tilde{g}(v_i):=(f-\lambda_i\id)^{k-1}(v_i)\neq 0 \]
			$\tilde{g}(v_i)$ stellt also auch einen Eigenvektor von $f$ zum Eigenwert $\lambda_i$ dar.
			Man macht sich leicht klar, dass $h,\tilde{h},g,\tilde{g}$ kommutierende Abbildungen sind.
			Aus diesem Grund ist
			\[ \tilde{h}\circ\tilde{g}(v_i) = \tilde{g}\circ\tilde{h}(v_i) \]
		\end{proof}

		\begin{theorem}[Fitting-Zerlegung]
			Seien $K$ ein Körper, $V=K^n$ für ein $n\in\SN$ und $\func{f}{V}{V}$ eine lineare Abbildung.
			Dann besitzt $\m{Hau}(f,0)$ stets genau ein $f$-stabiles Komplement $U$.
			\[ V = \m{Hau}(f,0)\oplus U \] 
		\end{theorem}
		\begin{proof}
			
		\end{proof}


		\begin{definition}[nilpotente Jordan-Block der Größe $r$]
			Sei $r\in\SN$.
			Dann sei der nilpotente Jordan-Block der Größe $r$ durch eine $r\times r$-Matrix $J(r)\in\m{M}_r(\SR)$ definiert, sodass für alle $i,j\in\SN$ mit $i,j\leq r$ gilt
			\[ J(r)_{ij} = \delta_{i+1\, j} \]
		\end{definition}

		\textsc{Beispiel:}\\
		\[
			J(1) = 0,\qquad
			J(2) =
			\begin{pmatrix}
				0 & 1 \\
				0 & 0 \\
			\end{pmatrix}
			,\qquad J(3) =
			\begin{pmatrix}
				0 & 1 & 0 \\
				0 & 0 & 1 \\
				0 & 0 & 0
			\end{pmatrix}
		\]


		\begin{theorem}[Normalform nilpotenter Abbildungen]
			Seien $K$ ein Körper und $V$ ein Vektorraum mit $V=K^n$ für $n\in\SN$.
			Sei $f:V\longrightarrow V$ eine nilpotente lineare Abbildung.
			Dann gibt es eine Basis $B$ von $V$, sodass 
			\[ \boxb{f} = \diag{J(r_1),\ldots,J(r_t)} \]
			Die Zahlen $r_1,\ldots,r_n$ sind hierbei von $f$ eindeutig bis auf ihre Reihenfolge bestimmt.
		\end{theorem}
		\begin{proof}
			Eindeutigkeit:
			\[ t = \dim(\ker f) \]
			\[ f(e_i) = 0 \quad \implies \quad i=1 \text{ oder } i = 1+r_1 \text{ oder } i = 1 + r_1 +\ldots +r_s \]
			mit $s< t$.
			\[ \dim(\ker f) \geq t \]
			\[ \dim(\im f) \geq \dim V - t \]
			\[ \im f = \angleb{e_i \vert 1\leq i\leq n} \]
			\[ e_j = f(e_i) \quad \equivalent \quad i\notin\curlb{r_1,r_1+r_2,\ldots,r_1+\ldots+r_t} \]
		\end{proof}

		\begin{corollary}
			Seien $A,B\in\m{M}_n(K)$ nilpotent.
			Dann existieren $r_1,\ldots,r_t$, sodass
			\[ A\sim\diag(J(r_1),\ldots,J(r_t)) \]
			\[ A\sim B\quad \equivalent \quad B\sim\diag(J(r_1),\ldots,J(r_t)) \]
		\end{corollary}

		\begin{lemma}
			Sei $f:V\longrightarrow V$ nilpotent mit $\dim V < \infty$.
			Dann existiert eine Basis $B$ von $V$, sodass $B\cup \curlb{0}$ stabil ist und jedes Element von $B$ unter $f$ höchstens ein Urbild in $B$ hat.
		\end{lemma}
		\begin{proof}
			Sei $B$ eine Basis des Lemmas von $f$.
			Nicht jedes Element von $B$ hat ein Urbild unter $f$ in $B$.
			($f$ wäre sonst nicht nilpotent).
			Wir sehen auch, dass es für alle $b\in B$ ein $b^\prime\in B$ und ein $k\geq 0$ gibt, sodass $b^\prime$ kein Urbild in $B$ hat und $b=f^k(b^\prime)$.
			$B$ zerfällt in Ketten unter $f$.
			Jede dieser Ketten entspricht dann einem Jordan-Block $J(r_i)$ der jeweiligen Größe $r_i$.

			Führe eine Induktion über die die Dimension des Vektorraums durch.
			\[ \im f \subset V \]
			\[ f \text{ nilpotent} \quad \implies \quad \ker f \neq \curlb{0} \quad \implies \quad \im f\neq V \]
			$\dim(\im f)\leq \dim V$.
			$\im f$ ist ein $f$-stabiler Unterraum.
			\[ \implies \quad \im f \text{ hat eine Jordan-Basis } J \]
			\[ \angleb{B_0}\oplus\curvb{ \im f \cap \ker f } = \ker f \]
		\end{proof}

		\begin{definition}[Jordan-Basis]
			Die Basis des vorherigen Lemmas wird auch Jordan-Basis von $f$ genannt.
		\end{definition}

		\begin{lemma}[Jedes $f$ besitzt eine Jordan-Basis]
			Eine Jordan-Basis existiert für alle linearen Abbildungen $f\colon V \funcarr V$.
		\end{lemma}
		\begin{proof}
			Führe hier eine Induktion über $n$ aus.
			Falls $n=0$ oder $n=1$ ist jede Basis von $V$ eine Jordan-Basis.
			\[ \ker f \neq \curlb{0} \quad \implies \quad \dim(\im f)<n \]
			$\im f$ ist ein $f$-stabiler Unterraum.
			$\im f$ besitzt eine Jordan-Basis $\mathcal{J}$ von $f$.
			\[ \mathcal{J}_0 := \set{t\in\mathcal{J}}{f(t)=0} \]
			$\mathcal{J}_0$ ist eine Basis von $ \im f \cap \ker f $.
			Sei $ \mathcal{J}_0 \cup B_0 $ eine Basis von $\ker f = (\ker f\cap \im f)\oplus\angleb{B_0}$.
			$f(\mathcal{J}) \subset \mathcal{J}\cup\curlb{0} $ weil $\mathcal{J} $ eine Jordan-Basis ist.
			Seien $s_1,\ldots,s_r\in\mathcal{J}$ genau die Elemente, die nicht in $f(\mathcal{J})$ liegen.
			Jedes Element $s_i$ liegt in $\im f$ (das heißt $s_i$ hat ein Urbild $b_i\in V$).
			\[ B:= \curlb{ b_1,\ldots,b_r }\cup B_0 \cup \mathcal{J} \]
			Wir zeigen, dass $B$ eine Basis von $V$ und dass $B$ eine Jordan-Basis ist.
			\[ B=(B_0 \sqcup \mathcal{J}_0)\sqcup \curvb{ \curlb{b_1,\ldots,b_r}\sqcup \set{t\in\mathcal{J}}{ f(t)\neq 0 } } \]
			$f(B)\setminus \curlb{0}$ ist eine Basis von $\im f$.
			Es folgt, dass $B$ linear unabhängig ist und dass $\#B=\dim V$.
			Hier fehlt noch was.
		\end{proof}

		\begin{proposition}
			Wenn $A^n=$ für $A\in\m{M}_n(K)$, dann
			\[ A\sim \diag(J(r_1),\ldots,J(r_m)) \]
		\end{proposition}

		\begin{proposition}[Eigenschaften]
			Seien $A,B\in\m{M}_n(\SC)$ für ein $n\in\SN$.
			\begin{enumerate}[label=\normalfont(\alph*)]
				\item Sei $C\in\m{Gl}_n(\SC)$. Dann gilt
					\[ \exp(CA\inv{C}) = C\exp(A)\inv{C} \]
				\item 
					\[ AB=BA \quad \implies \quad \exp(A+B)=\exp(A)\exp(B) \]
				\item Sei $A=A_s + A_{nl}$ die Jordan-Zerlegung von $A$.
					Hier ist $A_s$ diagonalisierbar, $A^n_{nl}=0$ und $A_sA_{nl}=A_{nl}A_s$.
					\[ A_s\sim \diag(\gamma_1,\ldots,\gamma_n),\qquad \exp(A_s)\sim \diag(e^{\gamma_1},\ldots,e^{\gamma_n}) \]
					\[ \exp(A) = \exp(A_s)\exp(A_{nl}) \]
					\[ A_{nl} = \diag(J(r_1),\ldots,J(r_m)) \]
					\[ \exp(A_{nl}) = \diag(\exp(J(r_1)),\ldots,\exp(J(r_m))) \]
			\end{enumerate}
		\end{proposition}
		\begin{proof}
			(a): Es gilt für ein $k\in\SN$
			\[ \curvb{CA\inv{C}}^k = CA^k\inv{C} \]
			(b): Direktes Rechnen.
		\end{proof}

		\begin{theorem}[Cayley-Hamilton]
			Sei $A\in\m{M}_n(\SC)$ für ein $n\in\SN$.
			Dann gilt
			\[ \chi_A(A) = 0 \]
		\end{theorem}
		Der Satz gilt auch, wenn die Jordan-Normalform nicht existiert.
		Das wird hier aber nicht gezeigt.
		\begin{proof}
			Sei 
			\[ \chi_A(x) = \prod_{i=1}^m \curvb{x-\lambda_i}^{\alpha_i} \quad \text{für } \lambda_i\neq\lambda_j,i\neq j \]
			\[ A\sim \diag\curvb{ J(r_1^{(1)}, \lambda_1),\ldots,J(r_{t(1)}^{(1)},\lambda_1), J(r_1^{(2)}, \lambda_2),\ldots,J(r_{t(m)}^{(m)},\lambda_m) } \]
			\[ \chi_A(A) = \prod_{i=1}^m \curvb{A-\lambda_i\idmat}^{\alpha_i} \quad \text{für } \lambda_i\neq\lambda_j,i\neq j \]
		\end{proof}

	% section jordansche_normalform (end)

	\section{Kegelschnitte und Quadriken} % (fold)
	\label{sec:kegelschnitte_und_quadriken}
	
		Wir betrachten den $\SR^2$ und eine Funktion $\func{h}{\SR^2}{\SR}$
		\[ h(x,y) = a_{11}x^2 + 2a_{12}xy + a_{22}y^2 + c_1x + c_2y + d \]
		Wir untersuchen
		\[ \set{(x,y)\in\SR^2}{ h(x,y)=0 } \]
		Wir lösen also eine quadratische Gleichung.

		\begin{definition}[quadratische Form]
			Eine Funktion $\func{q}{\SR^2}{\SR}$ heißt quadratische Form, genau dann wenn
			\[ q(x,y) = a_{11}x^2 + 2a_{12}xy + a_{22}y^2 \]
			für bestimmte Koeffizienten $a_{11},a_{12},a_{22}\in\SR$.
			Man schreibt dann auch für alle $v\in\SR^2$
			\[ q(v) = v\transp Av \]
			mit $A:= (a_{ij})\in\m{M}_2(\SR)$ symmetrisch.
		\end{definition}

		Auf $\SR^n$ sehen die quadratischen Formen so aus:
		\[ q(x_1,\ldots,x_n) = \sum_{i=1}^n a_{ii}x_i^2 + \sum_{j=i+1}^n a_{ij}x_ix_j \]
		Ist $A\in\m{M}_n(\SR)$ also eine symmetrische Matrix, so ist $q$ mit
		\[ q(v)=v\transp A v \]
		gerade eine quadratische Form.

		Die quadratischen Formen auf dem $\SR^n$ sind isomorph zu den Bilinearformen auf $\SR^n$.
		\[ q\mapsto b(v,w) = \tfrac{1}{2} \boxb{ q(v+w) - q(v) - q(w) } \]
		\[ h(x_1,\ldots,x_n) = q(x_1,\ldots,x_n) + \sum_{i=1}^n c_ix_i + d \]
		\[ \implies \quad h(v) = v\transp A v  + \angleb{c,v} + d \]

		\begin{definition}[Quadrike]
			Die Quadrike einer so definierten Funktion $h$ ist gerade die Lösungsmenge
			\[ \set{v\in\SR^n}{h(v)=0} \]
		\end{definition}

		\begin{definition}[Kegelschnitt]
			Ein Kegelschnitt ist gerade eine Quadrike aus dem $\SR^2$.
		\end{definition}

		Beispiel Kreis.

		\begin{definition}[entarteter Kegelschnitt]
			Ein Kegelschnitt heißt entartet, falls er ein Paar von Geraden, eine Gerade, einen Punkt oder die leere Menge ist.
		\end{definition}

		Beispiel

		\begin{theorem}[Kongruenz nicht entarteter Kegelschnitte]
			Jeder nicht entartete Kegelschnitt ist kongruent zu einem der folgenden Typen, wobei jeweils $a_11,a_22>0$.
			\begin{enumerate}[label=\normalfont(\roman*)]
				\item Ellipse:
					\[ a_11x_1^2 + a_22x_2^2 - 1 = 0 \]
				\item Hyperbel:
					\[ a_{11}x_1^2 - a_22x_2^2 - 1 = 0 \]
				\item Parabel:
					\[ a_{11}x_1^2 - x_2 = 0 \]
			\end{enumerate}
		\end{theorem}
		\begin{proof}
			Sei $A\in\m{M}_2(\SR)\neq 0$.
			Dann gibt es ein $C\in\m{O}_2(\SR)$, sodass
			\[
				C\transp AC =
				\begin{pmatrix}
					\lambda_1 & 0 \\ 0 & \lambda_2
				\end{pmatrix}
			\]
			($h$ und $-h$ besitzen die gleichen Lösungsmengen.)
			Wir nehmen an, dass $\lambda_1>0$.
			\[ \lambda_1 x_1^2 + c_1x_1 = \lambda_1\curvb{ x_1 + \frac{c_1}{2\lambda_1} }^2 - \frac{c_1^2}{4\lambda_1} \]
			\[ x_1^\prime := x_1 + \frac{c_1}{2\lambda_1} \]
			Wenn $\lambda_i\neq 0$, können wir den Koeffizienten $c_i$ eliminieren.

			Für den Fall $\lambda_2 >0$
			\[ \lambda_1\curvb{x_1^\prime}^2 + \lambda_2\curvb{ x_2^\prime }^2 + d^\prime = 0 \]
			Sei 
			\[ H(y_1,y_2) = \lambda_1y_1^2 + \lambda_2y_2^2 + C_1y_1 + C_2y_2 + D \]

			Wenn $d\geq 0$, dann ist der Kegelschnitt entartet.
			Dann folgt die Äquivalenz zu
			\[ \frac{\lambda_1}{-d}\curvb{x_1^\prime}^2 + \frac{\lambda_2}{-d}\curvb{x_2^\prime}^2 - 1 = 0 \]

			Sollte $\lambda_2 < 0$ sein, so gilt:
			\[ \lambda_1\curvb{x_1^\prime}^2 - \abs{\lambda_2}\curvb{x_2^\prime}^2 + d = 0 \]
			Wenn $d=0$, dann ist $h$ entartet.
			Wenn $d>0$: So bekommen wir $d<0$.

			Sei nun $\lambda_2 = 0$.
			Dann muss $c_2 \neq 0$.
			\[ h\sim \frac{\lambda_1}{-c_2}\curvb{ x_1^\prime }^2 - x_2 \]
		\end{proof}

		\textsc{Beispiel:}\\
		\[ h(x_1,x_2) = x_1^2 + x_1x_2 + x_2^2 + 4x_1 + 3x_2 + 4 \]
		\[
			A =
			\begin{pmatrix}
				1 & \tfrac{1}{2} \\ \tfrac{1}{2} & 1
			\end{pmatrix}
			,\qquad
			v_{\lambda_1 = 3/2} = 
			\begin{pmatrix}
				1 \\ 1
			\end{pmatrix}
			,\quad
			v_{\lambda_2 = 1/2} =
			\begin{pmatrix}
				1 \\ -1
			\end{pmatrix}
		\]
		\[
			C = \frac{1}{\sqrt{2}}
			\begin{pmatrix}
				1 & 1 \\ 1 & -1
			\end{pmatrix}
		\]
		\[ h(x_1,x_2) = \frac{1}{4}(x_1-x_2)^2 + \frac{3}{4}(x_1+x_2)^2 + \frac{1}{2}(x_1-x_2) + \frac{7}{2}(x_1+x_2) + 4 \]
		Dies ist eine Ellipse.


		Ist nun $n\geq 3$, so gibt es nach dem Spektralsatz für eine orthogonale Matrix, sodass $A$ eine Diagonalmatrix mit Eigenwerten auf der Diagonalen.
		Ist dann $\lambda_i\neq 0$, so können wir den Koeffizienten $c_i$ eliminieren.
		Eine Normalform:
		\begin{itemize}
			\item $\det A \neq 0$
				\[ \sum_{i=1}^n \lambda_ix_i^2 + d = 0 \]
			\item $\det A = 0$
				\[ \sum_{i=1}^k \lambda_ix_i^2 + C_{k+1}x_{k+1} = 0 \]
		\end{itemize}
		Beispiel Ellipsoide, einschalige Hyperboloide, zweischalige Hyperboloide, hyperbolische Paraboloide.

	% section kegelschnitte_und_quadriken (end)

\end{document}